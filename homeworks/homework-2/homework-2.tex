\documentclass[12pt]{article}
\usepackage{amsmath,fullpage,graphicx,fancyhdr,enumerate,amsthm,amssymb,tikz,tikz-cd,color,pgfplots,tabularx,amsfonts,amscd}
\usepackage{enumitem}
\usetikzlibrary{arrows,chains,matrix,positioning,scopes}
\pgfplotsset{compat=1.13}
\setlength{\headheight}{15pt}
\setlength{\headsep}{7pt}
\newcommand{\bsnl}{\bigskip\newline}
\pagestyle{fancy}
\usepackage[paper=a4paper, margin = 1in]{geometry}
\usepackage{hyperref}% http://ctan.org/pkg/hyperref
\usepackage[capitalise, noabbrev]{cleveref}

\hypersetup{ % colors hyperlinks with colors to make noticeable but is not an ugly green box like default
    colorlinks,
    linkcolor={red!50!black},
    citecolor={blue!50!black},
    urlcolor={blue!80!black}
}
\usepackage[mathscr]{euscript}\usepackage{pifont}
\usepackage{cleveref}
\usepackage{parskip}
\usepackage[makeroom]{cancel}


\newcommand{\cmark}{\ding{51}}%
\newcommand{\xmark}{\ding{55}}%

\renewcommand{\aa}{\mathbf{a}}
\newcommand{\bb}{\mathbf{b}}
\newcommand{\cc}{\mathbf{c}}
\newcommand{\dd}{\mathbf{d}}
\newcommand{\FF}{\mathbf{F}}
\newcommand{\ii}{\mathbf{i}}
\newcommand{\jj}{\mathbf{j}}
\newcommand{\rr}{\mathbf{r}}
\newcommand{\kk}{\mathbf{k}}
\newcommand{\uu}{\mathbf{u}}
\newcommand{\vv}{\mathbf{v}}
\newcommand{\ww}{\mathbf{w}}
\newcommand{\yy}{\mathbf{y}}
\newcommand{\R}{\mathbb{R}}
\newcommand{\T}{\mathcal{T}}
\newcommand{\N}{\mathbb{N}}
\newcommand{\C}{\mathscr{C}}
\newcommand{\Z}{\mathbb{Z}}
\newcommand{\B}{\mathcal{B}}
\newcommand{\Q}{\mathbb{Q}}
\newcommand{\Sph}{\mathbb{S}}
\newcommand{\D}{\mathbb{D}}

\def\dim{\mathop{\rm dim}\nolimits}
\def\image{\mathop{\rm Im}\nolimits}  
\def\interior{\mathop{\rm Int}\nolimits}
\def\kernel{\mathop{\rm Ker}\nolimits}
\def\cokernel{\mathop{\rm Coker}\nolimits}
\def\bd{\mathop{\rm Bd}\nolimits}
\def\ext{\mathop{\rm Ext}\nolimits}
\def\num{\mathop{\#}\limits}
\def\cl{\mathop{\rm Cl}\nolimits}
\def\lub{\mathop{\rm lub}\nolimits}


\newcommand{\proj}{\operatorname{proj}}
\newcommand{\ds}{\displaystyle}
\newcommand{\pa}{\partial}
\newcommand{\ol}{\overline{}}


\newcommand{\degree}{^{\circ}}
\renewcommand{\epsilon}{\varepsilon}

\pagestyle{fancy}
\lhead{Math 6421: Measure Theory}
\chead{\bf Jose Nino: Homework \#2}
\rhead{September 14, 2023}
\cfoot{Page \thepage}
% \setlength{\headheight}{10pt}

\theoremstyle{definition}
\newtheorem*{thm}{Theorem}
\newtheorem*{exercise}{Exercise}
\newtheorem*{definition}{Definition}
\newtheorem{problem}{Problem}
\newtheorem{lemma}{Lemma}
\newtheorem*{cor}{Corollary}


\begin{document}

\begin{problem}[3.5]

    Let \( A \) be thee set of rational numbers between \( 0 \) and \( 1 \), and let \( \{ I_n \} \) be a \textbf{finite} collection of open intervals covering A. Then \( \displaystyle \sum_{i=1}^{n} l(I_n) \geq 1 \).

        \begin{proof}
            
        \end{proof}
     
\end{problem}

\begin{problem}[3.10]

    Show that if \( E_1 \) and \( E_2 \) are measure, then 
        \[
            m(E_1 \cap  E_2) + m(E_1 + E_2) = mE_1 + mE_2.
        \]
    Recall that if \( E \) is measurable, then \( mE := m^{*} E \).

        \begin{proof}
            
        \end{proof}
    
\end{problem}

\begin{problem}[3.13]

    Prove Proposition 15 by the following steps:
        \begin{enumerate}[label = \alph{*}.]
            \item Show that for \( m^{*} E < \infty \), \( (i) \Rightarrow (ii) \Leftrightarrow (vi) \).
            \begin{proof}
                To show that the following are equivalent, we will need to create a chain of implications starting with the outline above. 

                \begin{enumerate}
                    \item[\((i) \Rightarrow (ii)\)]
                    \item[\((ii) \Rightarrow (vi)\)]
                    \item[\((vi) \Rightarrow (ii)\)]   
                \end{enumerate}
            \end{proof}
            \item Use part a. to show that for arbitrary sets, \( (i) \Rightarrow (ii) \Rightarrow (iv) \Rightarrow (i) \).
            \begin{proof}
                Finally, to complete the chain of implications, we will use the outline above. 

                \begin{enumerate}
                    \item[\((i) \Rightarrow (ii)\)]
                    \item[\((ii) \Rightarrow (vi)\)]
                    \item[\((iv) \Rightarrow (i)\)]   
                \end{enumerate}
            \end{proof}
        \end{enumerate}
    
\end{problem}

\end{document}