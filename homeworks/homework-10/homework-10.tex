\documentclass[12pt]{article}
\usepackage{amsmath,fullpage,graphicx,fancyhdr,enumerate,amsthm,amssymb,tikz,tikz-cd,color,pgfplots,tabularx,amsfonts,amscd}
\usepackage{enumitem}
\usetikzlibrary{arrows,chains,matrix,positioning,scopes}
\pgfplotsset{compat=1.13}
\setlength{\headheight}{15pt}
\setlength{\headsep}{7pt}
\newcommand{\bsnl}{\bigskip\newline}
\pagestyle{fancy}
\usepackage[paper=a4paper, margin = 1in]{geometry}
\usepackage{hyperref}% http://ctan.org/pkg/hyperref
\usepackage[capitalise, noabbrev]{cleveref}

\hypersetup{ % colors hyperlinks with colors to make noticeable but is not an ugly green box like default
    colorlinks,
    linkcolor={red!50!black},
    citecolor={blue!50!black},
    urlcolor={blue!80!black}
}
\usepackage[mathscr]{euscript}\usepackage{pifont}
\usepackage{cleveref}
\usepackage{parskip}
\usepackage[makeroom]{cancel}


\newcommand{\cmark}{\ding{51}}%
\newcommand{\xmark}{\ding{55}}%

\renewcommand{\aa}{\mathbf{a}}
\newcommand{\bb}{\mathbf{b}}
\newcommand{\cc}{\mathbf{c}}
\newcommand{\dd}{\mathbf{d}}
\newcommand{\FF}{\mathbf{F}}
\newcommand{\ii}{\mathbf{i}}
\newcommand{\jj}{\mathbf{j}}
\newcommand{\rr}{\mathbf{r}}
\newcommand{\kk}{\mathbf{k}}
\newcommand{\uu}{\mathbf{u}}
\newcommand{\vv}{\mathbf{v}}
\newcommand{\ww}{\mathbf{w}}
\newcommand{\yy}{\mathbf{y}}
\newcommand{\R}{\mathbb{R}}
\newcommand{\T}{\mathcal{T}}
\newcommand{\N}{\mathbb{N}}
\newcommand{\C}{\mathscr{C}}
\newcommand{\Z}{\mathbb{Z}}
\newcommand{\B}{\mathcal{B}}
\newcommand{\Q}{\mathbb{Q}}
\newcommand{\Sph}{\mathbb{S}}
\newcommand{\D}{\mathbb{D}}

\def\dim{\mathop{\rm dim}\nolimits}
\def\image{\mathop{\rm Im}\nolimits}  
\def\interior{\mathop{\rm Int}\nolimits}
\def\kernel{\mathop{\rm Ker}\nolimits}
\def\cokernel{\mathop{\rm Coker}\nolimits}
\def\bd{\mathop{\rm Bd}\nolimits}
\def\ext{\mathop{\rm Ext}\nolimits}
\def\num{\mathop{\#}\limits}
\def\cl{\mathop{\rm Cl}\nolimits}
\def\lub{\mathop{\rm lub}\nolimits}


\newcommand{\proj}{\operatorname{proj}}
\newcommand{\ds}{\displaystyle}
\newcommand{\pa}{\partial}
\newcommand{\ol}{\overline{}}


\newcommand{\degree}{^{\circ}}
\renewcommand{\epsilon}{\varepsilon}
\newcommand{\toM}{\overset{m}{\to}}


\newcommand{\upint}[2]{
  \overline{\int_{#1}^{#2}}
}
\newcommand{\lowint}[2]{
  \underline{\int_{#1}^{#2}}
}
\newcommand{\dif}{\, \mathrm{d}}
\newcommand{\norm}[1]{\left\lVert #1 \right\rVert}
\newcommand{\abs}[1]{\left\lvert #1 \right\rvert}


\pagestyle{fancy}
\lhead{Math 6421: Measure Theory}
\chead{\bf Jose Nino: Homework \#10}
\rhead{November 16, 2023}
\cfoot{Page \thepage}
% \setlength{\headheight}{10pt}

\theoremstyle{definition}
\newtheorem*{thm}{Theorem}
\newtheorem*{exercise}{Exercise}
\newtheorem*{definition}{Definition}
\newtheorem{problem}{Problem}
\newtheorem{lemma}{Lemma}
\newtheorem*{cor}{Corollary}
\newtheorem*{prop}{Proposition}


\begin{document}

\begin{problem}[11.10]

    Prove Proposition 11.7 which is stated follows:
\begin{prop}[11.7]
Let \( f \) be an nonnegative measurable function. 
Then there is a sequence \( \{\phi_n\} \) of simple functions with \( \phi_{n+1} \geq \phi_n \) such that \( \displaystyle f = \lim_{n \to \infty} \phi_n \) at each point of \( X \). If \( f \) is defined on a  \( \sigma \)-finite measure space, then we may choose the functions \( \phi_n \) so that each vanishes outside a set of finite measure.
    \begin{proof}
        Let \( f \) be a nonnegative measurable function. Per the hint, for every pair of integers \( (n,k) \), let 
            \[
                E_{n,k} = \{ x: k 2^{-n} \leq f(x) < (k+1)2^{-n} \}, \, \text{and set} \ \phi_n = 2^{-n} \sum_{k=0}^{2^{2n}} k \chi_{E_{n,k}}.
            \]
        Let \( (n,k) \) be any arbitrary pair of each integers. 
        Then because \( f \) is measurable, each \( E_{n,k} \) is a measurable set and so \( \phi_n \) is a simple function defined on each \( E_{n,k} \).
            First, we will note that
            \[
                E_{n,k} = E_{n+1, 2k} \cup E_{n+1, 2k+1}.
            \]  
            Let \( x \in E_{n,k} \). This means \( \phi_n(x) = k 2^{-n} \). 
            Now suppose \( x \in E_{n+1, 2k} \).
            Then we know that 
                \[
                    \phi_{n+1}(x) = (2k)2^{-(n+1)} = k2^{-n} = \phi_n(x). 
                \]
            Lastly, suppose that \( x \in E_{n+1, 2k+1} \). Then we know that
                \[
                    \phi_{n+1}(x) = (2k+1) 2^{-(n+1)} > (2k)2^{-(n+1)} \phi_n(x). 
                \]
            Thus, in all cases, \( \phi_n(x) \leq \phi_{n+1}(x) \).
            
            To prove pointwise convergence, let \( x \in X \) be any point. This brings two cases: either (i) \( f(x) < \infty\) or (ii) \( f(x) = \infty \).
            First, assume that \( f(x) < \infty \). Because of how we defined \( \phi_n \) and \( E_{n,k} \), we know that
                \[
                    \abs{f(x) - \phi_n(x)} \leq 2^{-n}  
                \]
            will always exist with \( n \in \N \) large enough. 
            But because \( (n,k) \) are chosen arbitrarily, we have that \( \displaystyle f = \lim_{n \to \infty} \phi_n \).
            Now, suppose that \( f(x) = \infty \). Then 
                \[
                    \phi_n(x) = (2^{2n} + 1)2^{-n} = 2^{n} + \frac{1}{2^{n}} > 2^{n}. 
                \]
            So as \( n \to \infty\), \( \phi_n \to \infty \) as well and so we still have \( \displaystyle f = \lim_{n \to \infty} \phi_n \). Therefore, in all cases, we have pointwise convergence. 
            

        Suppose \( f \) is defined on a \( \sigma \)-finite measure space. 
        Then \( \displaystyle X = \bigcup_{n} X_n \) with \( \mu(X_n) < \infty \) for all \( n \in \N \).
        Define \( E_{n,k} \) the same as above but define \( \phi_n \) on the set \( \displaystyle E_{n,k} \cap \bigcup_{m=1}^{n} X_m \) i.e.,
            \[
                \phi_n = 2^{-n} \sum_{k=0}^{2^{2n}} k \chi_{E_{n,k} \cap \bigcup_{m=1}^{n} X_m}.
            \]
        So, by a similar argument to above, \( \phi_{n+1} \geq \phi_n \) and \( \displaystyle f = \lim_{n \to \infty} \phi_n \). 
        However, each simple function will vanish outside of the set of finite measure, \( \displaystyle \bigcup_{m=1}^{n} X_m \).  This completes the proof.
    \end{proof}
\end{prop}
\end{problem}

\begin{problem}[11.22]

    \begin{enumerate}[label = (\alph{*})]
        \item Let \( (X, \B, \mu) \) be a measure space and \( g \) a nonnegative measurable function on \( X \). Set \( \displaystyle \nu(E) = \int_{E} g \dif \mu \). 
        Show that \( \nu \) is a measure on \( \B \).

            \begin{proof}
                Let \( g \) be a nonnegative measurable function on the measure space \( (X, \B, \mu) \). Set \( \displaystyle \nu(E) = \int g \dif  \). Let \( E = \emptyset \). Then certainly
                    \[
                        \int_{E} g  = 0
                    \]
                and so \( \nu(\emptyset) = 0 \).

                To prove countable additivity, let \( \{E_n\} \) be a sequence of sets with \( E_i \cap E_j = \emptyset \) for any \( i \neq j \). Thus, we have then that
                    \begin{align*}
                        \nu \left( \bigcup_{n=1}^{\infty} E_n \right) = \int_{\bigcup_{n=1}^{\infty} E_n} g \dif \mu &= \int g \chi_{\bigcup_{n=1}^{\infty} E_n} \dif \mu \\ 
                        &= \int \sum_{n=1}^{\infty} g \chi_{E_n} \dif \mu \\
                        &= \sum_{n=1}^{\infty} \int g \chi_{E_n} \dif \mu \\
                        &= \sum_{n=1}^{\infty} \int_{E} g \dif \mu \\
                        &= \sum_{n=1}^{\infty} \nu \left( E_n \right)
                    \end{align*}
                which completes this proof. 
            \end{proof}
        \item Let \(f \) be a nonnegative measurable function on \( X \). Then 
            \[
                \int f \dif v = \int fg \dif \mu.  
            \]

            \begin{proof}
                Let \(f \) be a nonnegative measurable function on \( X \).
                We will work through two cases: (i) \( f \) is a simple function and (ii) \( f \) is any other measurable function. 
                Suppose \( f \) is a simple function i.e., 
                    \[
                        f = \sum_{n=1}^{\infty}  c_i \chi_{E_{i}}.
                    \]
                Using properties of simple function, we can show the following:
                    \begin{align*}
                        \int f \dif \mu = \sum_{i=1}^{n} c_i \nu(E_i) &= \sum_{i=1}^{n} c_i \int_{E_i} g \dif \mu  \\
                        &= \sum_{i=1}^{n} c_i \int g \chi_{E_{i}} \dif \mu \\
                        &= \int_{E} \sum_{i=1}^{n} c_i  g \chi_{E_{i}} \dif \mu \\
                        &= \int_{E} fg.
                    \end{align*}
                Now, suppose \( f \) is any measurable but not simple function. 
                Because \( f \) is nonnegative, there exists an increasing sequence of simple functions \( \{\phi_n\}\) such that \( \displaystyle f = \lim_{n \to \infty} \phi_n \).
                Now take the sequence \( \{\phi_n g\} \) at each point on \( X \).
                We have \( g \) as nonnegative and so \( \{\phi_n g \} \) is also an increasing sequence of functions and converges with \( \displaystyle fg = \lim_{n \to \infty} \phi_n g \).
                Thus, having satisfied the properties of the Monotone Convergence Theorem, we have that
                    \[
                        \int fg \dif \mu = \lim_{n \to \infty} \int \phi_n g \dif \mu = \lim_{n \to \infty} \int \phi_n \dif \nu = \int f \dif \nu. 
                    \]
                Therefore, having exhausted all cases, this completes the proof. 
            \end{proof}
    \end{enumerate}


\end{problem}

\end{document}