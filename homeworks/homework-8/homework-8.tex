\documentclass[12pt]{article}
\usepackage{amsmath,fullpage,graphicx,fancyhdr,enumerate,amsthm,amssymb,tikz,tikz-cd,color,pgfplots,tabularx,amsfonts,amscd}
\usepackage{enumitem}
\usetikzlibrary{arrows,chains,matrix,positioning,scopes}
\pgfplotsset{compat=1.13}
\setlength{\headheight}{15pt}
\setlength{\headsep}{7pt}
\newcommand{\bsnl}{\bigskip\newline}
\pagestyle{fancy}
\usepackage[paper=a4paper, margin = 1in]{geometry}
\usepackage{hyperref}% http://ctan.org/pkg/hyperref
\usepackage[capitalise, noabbrev]{cleveref}

\hypersetup{ % colors hyperlinks with colors to make noticeable but is not an ugly green box like default
    colorlinks,
    linkcolor={red!50!black},
    citecolor={blue!50!black},
    urlcolor={blue!80!black}
}
\usepackage[mathscr]{euscript}\usepackage{pifont}
\usepackage{cleveref}
\usepackage{parskip}
\usepackage[makeroom]{cancel}


\newcommand{\cmark}{\ding{51}}%
\newcommand{\xmark}{\ding{55}}%

\renewcommand{\aa}{\mathbf{a}}
\newcommand{\bb}{\mathbf{b}}
\newcommand{\cc}{\mathbf{c}}
\newcommand{\dd}{\mathbf{d}}
\newcommand{\FF}{\mathbf{F}}
\newcommand{\ii}{\mathbf{i}}
\newcommand{\jj}{\mathbf{j}}
\newcommand{\rr}{\mathbf{r}}
\newcommand{\kk}{\mathbf{k}}
\newcommand{\uu}{\mathbf{u}}
\newcommand{\vv}{\mathbf{v}}
\newcommand{\ww}{\mathbf{w}}
\newcommand{\yy}{\mathbf{y}}
\newcommand{\R}{\mathbb{R}}
\newcommand{\T}{\mathcal{T}}
\newcommand{\N}{\mathbb{N}}
\newcommand{\C}{\mathscr{C}}
\newcommand{\Z}{\mathbb{Z}}
\newcommand{\B}{\mathcal{B}}
\newcommand{\Q}{\mathbb{Q}}
\newcommand{\Sph}{\mathbb{S}}
\newcommand{\D}{\mathbb{D}}

\def\dim{\mathop{\rm dim}\nolimits}
\def\image{\mathop{\rm Im}\nolimits}  
\def\interior{\mathop{\rm Int}\nolimits}
\def\kernel{\mathop{\rm Ker}\nolimits}
\def\cokernel{\mathop{\rm Coker}\nolimits}
\def\bd{\mathop{\rm Bd}\nolimits}
\def\ext{\mathop{\rm Ext}\nolimits}
\def\num{\mathop{\#}\limits}
\def\cl{\mathop{\rm Cl}\nolimits}
\def\lub{\mathop{\rm lub}\nolimits}


\newcommand{\proj}{\operatorname{proj}}
\newcommand{\ds}{\displaystyle}
\newcommand{\pa}{\partial}
\newcommand{\ol}{\overline{}}


\newcommand{\degree}{^{\circ}}
\renewcommand{\epsilon}{\varepsilon}
\newcommand{\toM}{\overset{m}{\to}}


\newcommand{\upint}[2]{
  \overline{\int_{#1}^{#2}}
}
\newcommand{\lowint}[2]{
  \underline{\int_{#1}^{#2}}
}
\newcommand{\dif}{\, \mathrm{d}}
\newcommand{\norm}[1]{\left\lVert #1 \right\rVert}
\newcommand{\abs}[1]{\left\lvert #1 \right\rvert}


\pagestyle{fancy}
\lhead{Math 6421: Measure Theory}
\chead{\bf Jose Nino: Homework \#8}
\rhead{November 2, 2023}
\cfoot{Page \thepage}
% \setlength{\headheight}{10pt}

\theoremstyle{definition}
\newtheorem*{thm}{Theorem}
\newtheorem*{exercise}{Exercise}
\newtheorem*{definition}{Definition}
\newtheorem{problem}{Problem}
\newtheorem{lemma}{Lemma}
\newtheorem*{cor}{Corollary}
\newtheorem*{prop}{Proposition}


\begin{document}

\begin{problem}[5.10]

  \begin{enumerate}[label = (\alph{*})]
    \item  Let \( f \) be defined  by
      \[
          f(x) = \begin{cases}
            0 & x = 0 \\
            x^2 \sin \left( \frac{1}{x^2} \right) & x \neq 0.
          \end{cases}
      \]
      Is \( f \) of bounded variation on \( [-1,1] \)?

          \begin{proof}
            It will suffice to show that the function is not of bounded variation on \( [0,1] \) as the function is symmetric across the y-axis and thus the same argument holds for \( [-1, 1] \).
            Consider the following subdivision/partition \( \mathcal{P} \) of \( [0, 1] \):
              \[
                   0 < \sqrt{\frac{1}{\pi n}} < \cdots < \sqrt{\frac{2}{\pi(1 + 4n)}} < 1
              \]
            for any \( n \in \N \). Note that we picked these points because \(\)
              \[
                  f\left( \sqrt{\frac{1}{\pi n}}  \right) = 0 \quad \text{and} \quad f \left(  \sqrt{\frac{2}{\pi(1 + 4n)}}\right) = 1
              \]
            since \( \displaystyle \sin \left( \frac{1}{x^2}\right) = 1\) when \( \displaystyle x = \sqrt{\frac{1}{\pi n}} \) and \( \displaystyle \sin \left( \frac{1}{x^2}\right) = 1 \) when \( \displaystyle x = \frac{2}{\pi(1 + 4n)} \).

             Additionally, note that the range of \(f\) is \( [0,1] \).
             When a point \( x \in [0, 1] \) can be written as \( \displaystyle \sqrt{\frac{1}{\pi n}}\) for some \( n \in \N \), the variation of \( f \) across all of these points is \( 0 \). The maximum of \( f \) is \( 1 \) and so this means the total variation of \( f \) is determined by \( x^2 \) where \( \displaystyle \sin \left( \frac{1}{x^2}  \right) = 1 \). So we have that 
              \[
                  T_f = \sum_{n=1}^{k} \abs{\left(\sqrt{\frac{2}{\pi(1 + 4n)}}\right)^{2}} = \sum_{n=1}^{k} \frac{2}{\pi(1 + 4n)} = \frac{2}{\pi} \sum_{i=1}^{k} \frac{1}{1 + 4n}
.              \]
            The series on the right-hand side is of a form similar to the harmonic series so as \( k \to \infty \), this means \( T_f \to \infty \). Therefore, the function \( f \) is not of bounded variation. 
          \end{proof}
    \item  Not assigned. 
  \end{enumerate}

    
\end{problem}

\begin{problem}[5.15]

  The Cantor ternary function (Problem 2.48) is continuous and monotone but not absolutely continuous. 
    \begin{proof}
      By Problem 2.48, the Cantor function is continuous and monotone on \( [0,1] \).
      Thus, we must show that the Cantor function \( f \) is not absolutely continuous. 
      By way of contradiction, suppose that \( f \) is indeed absolutely continuous on \( [0,1] \).
      By Theorem 5.14, for any \( x \in [0,1] \), \( f \) can be written as an indefinite integral i.e.,
        \[
            f(x) = \int_{0}^{x} f'(t) \dif t + f(0)
        \]
      or, equivalently,
        \[
          f(x) - f(0) = \int_{0}^{x} f'(t) \dif t.
        \]
      First, we will show that the Cantor set has measure \( 0 \). 
      Let \( \{C_n\} \) represent the sequence of Cantor sets where \( \displaystyle C_0 = [0,1] \), \( \displaystyle C_1 = \left[ 0, \frac{1}{3} \right] \cup \left[ \frac{2}{3}, 1 \right] \), and so on. 
      Note that the Cantor is represented by \( \displaystyle C = \bigcap_{n=1}^{\infty} C_n \).
      We can see that the measure of the \( n^{\text{th}} \) cantor set is \( \displaystyle m \left(C_n\right) = \left( \frac{2}{3} \right)^{n} \), and that \( \{C_n\} \) is a decreasing sequence of measurable sets.
       So we can apply Proposition 3.14 and say that 
        \[
              \lim_{n \to \infty} m(C_n) = 0 = m \left( \bigcap_{n=1}^{\infty} C_n \right) = m(C)
        \]
      and so the measure of the Cantor set is \( 0 \). By definition of the function, \( f \) is constant if an element is not in the Cantor set. 
      Thus \( f \) is constant on \( [0,1] \) almost everywhere and so \( f'(x) = 0 \) almost everywhere as well (i.e., for all \( x \in [0,1] \setminus C \)). However, suppose that \( x = 1 \). Then we have that 
        \[
          f(1) - f(0) = 1 \neq \int_{0}^{x} f'(t) \dif t = 0
        \]        
      which contradicts Theorem 5.14. Thus the Cantor function \( f \) is not absolutely continuous. 
    \end{proof}
\end{problem}

\begin{problem}[5.20] 
  A function \( f \) is said to satisfy a Lipschitz condition on an interval if there is a constant \( M \) such that \( \abs{f(x) - f(y)} \leq M \abs{x - y} \) for all \( x \) and \( y \) in the interval. 
  \begin{enumerate}[label = (\alph{*})]
    \item Show that a function satisfying a Lipschitz condition is absolutely continuous. 
      \begin{proof}
        Not assigned.
      \end{proof}
    \item  Show that an absolutely continuous function \( f \) satisfies a Lipschitz condition if and only \( \abs{f'} \) is bounded. 
    
      \begin{proof}
        We will show a forward and reverse implication. Let \( f \) be an absolutely continuous function on an interval \( I \subset \R \).
          \begin{enumerate}
            \item[(\(\Rightarrow\))] First, suppose that \( f \) satisfies a Lipschitz condition.  
            Note that the definition of the derivative says that 
              \[
                  f'(x) = \lim_{y \to x}  \frac{f(y) - f(x)}{y - x}.
              \]
            Because \( f \) satisfies a Lipschitz condition, there exists \( M \in \R \) such that 
              \[
                 \abs{f(x) - f(y)} \leq M \abs{x - y} 
              \]
            or 
              \[
                 \frac{\abs{f(x) - f(y)}}{\abs{x - y}}  \leq M.
              \]
            It can be shown readily that 
              \[
                 \abs{f'(x)} = \lim_{y \to x}  \abs{\frac{f(y) - f(x)}{y - x}}
              \]
            and so we can conclude that 
              \[
                  \abs{f'(x)} \leq M
              \]
            and therefore \( f' \) is bounded.
            \item[(\(\Leftarrow\))] We proceed by contraposition, and so first suppose that \( f \) does not satisfy the Lipschitz condition. So for any \( M > 0\), there exists \( x,y \in I \) such that \( \abs{f(x) - f(y)} > M \abs{x - y} \).
            equivalently, 
              \[  
                  \abs{\frac{f(x) - f(y)}{x - y}}> M.
              \]
            Because \( f \) is absolutely continuous and therefore continuous on \( I \), we can apply the Mean Value Theorem. 
            Thus, there exists \( c \in I \) such that \( \abs{f'(c)} > M \) which completes the proof.
          \end{enumerate}
  
      \end{proof}
  \end{enumerate}
    
\end{problem}

\end{document}