\documentclass[12pt]{article}
\usepackage{amsmath,fullpage,graphicx,fancyhdr,enumerate,amsthm,amssymb,tikz,tikz-cd,color,pgfplots,tabularx,amsfonts,amscd}
\usepackage{enumitem}
\usetikzlibrary{arrows,chains,matrix,positioning,scopes}
\pgfplotsset{compat=1.13}
\setlength{\headheight}{15pt}
\setlength{\headsep}{7pt}
\newcommand{\bsnl}{\bigskip\newline}
\pagestyle{fancy}
\usepackage[paper=a4paper, margin = 1in]{geometry}
\usepackage{hyperref}% http://ctan.org/pkg/hyperref
\usepackage[capitalise, noabbrev]{cleveref}

\hypersetup{ % colors hyperlinks with colors to make noticeable but is not an ugly green box like default
    colorlinks,
    linkcolor={red!50!black},
    citecolor={blue!50!black},
    urlcolor={blue!80!black}
}
\usepackage[mathscr]{euscript}\usepackage{pifont}
\usepackage{cleveref}
\usepackage{parskip}
\usepackage[makeroom]{cancel}


\newcommand{\cmark}{\ding{51}}%
\newcommand{\xmark}{\ding{55}}%

\renewcommand{\aa}{\mathbf{a}}
\newcommand{\bb}{\mathbf{b}}
\newcommand{\cc}{\mathbf{c}}
\newcommand{\dd}{\mathbf{d}}
\newcommand{\FF}{\mathbf{F}}
\newcommand{\ii}{\mathbf{i}}
\newcommand{\jj}{\mathbf{j}}
\newcommand{\rr}{\mathbf{r}}
\newcommand{\kk}{\mathbf{k}}
\newcommand{\uu}{\mathbf{u}}
\newcommand{\vv}{\mathbf{v}}
\newcommand{\ww}{\mathbf{w}}
\newcommand{\yy}{\mathbf{y}}
\newcommand{\R}{\mathbb{R}}
\newcommand{\T}{\mathcal{T}}
\newcommand{\N}{\mathbb{N}}
\newcommand{\C}{\mathscr{C}}
\newcommand{\Z}{\mathbb{Z}}
\newcommand{\B}{\mathcal{B}}
\newcommand{\Q}{\mathbb{Q}}
\newcommand{\Sph}{\mathbb{S}}
\newcommand{\D}{\mathbb{D}}

\def\dim{\mathop{\rm dim}\nolimits}
\def\image{\mathop{\rm Im}\nolimits}  
\def\interior{\mathop{\rm Int}\nolimits}
\def\kernel{\mathop{\rm Ker}\nolimits}
\def\cokernel{\mathop{\rm Coker}\nolimits}
\def\bd{\mathop{\rm Bd}\nolimits}
\def\ext{\mathop{\rm Ext}\nolimits}
\def\num{\mathop{\#}\limits}
\def\cl{\mathop{\rm Cl}\nolimits}
\def\lub{\mathop{\rm lub}\nolimits}


\newcommand{\proj}{\operatorname{proj}}
\newcommand{\ds}{\displaystyle}
\newcommand{\pa}{\partial}
\newcommand{\ol}{\overline{}}


\newcommand{\degree}{^{\circ}}
\renewcommand{\epsilon}{\varepsilon}

\pagestyle{fancy}
\lhead{Math 6421: Measure Theory}
\chead{\bf Jose Nino: Homework \#3}
\rhead{September 21, 2023}
\cfoot{Page \thepage}
% \setlength{\headheight}{10pt}

\theoremstyle{definition}
\newtheorem*{thm}{Theorem}
\newtheorem*{exercise}{Exercise}
\newtheorem*{definition}{Definition}
\newtheorem{problem}{Problem}
\newtheorem{lemma}{Lemma}
\newtheorem*{cor}{Corollary}


\begin{document}

\begin{problem}[3.23] Prove Proposition 3.22 by the following lemmas:

    \begin{enumerate}[label = \alph{*}.]
        \item Given a measurable function \( f \) on \( [a,b] \) that takes the values \( \pm \infty \) only on a set of measure zero, and given \( \epsilon > 0 \), there is an \( M \in \R \) such that \( |f| \leq M \) except on a set of measure less than \( \displaystyle \frac{\epsilon}{3} \).
            \begin{proof}

                Suppose \( f \) is a measurable function  on \( [a,b] \) and that \( f(x) = \pm \infty \) only on a set of measure zero. Let \( \epsilon > 0 \) be chosen. Define the set 
                    \[
                        E_n = \{ x \in [a,b]: |f(x)| > n \} \ \text{for all} \ n \in \N.  
                    \]
                Because the function \( f \) is measurable, by definition, this means that each \( E_i \) is a measurable set as well. Note that by construction of \( E_n \), we have that \( E_{i} \subset E_{i+1} \) and so \( \{ E_n \} \) is a decreasing sequence. Since \( E_1 \) is a subset of the inverse image of \( f \) which is itself a subset of \( [a,b] \) i.e., \( E_1 \subset [a,b] \), we have that
                    \[
                        m(E_1) < m([a,b]) = b - a < \infty.  
                    \]
                Again, by the construction of \( E_n \), we have that 
                    \[
                        \bigcap_{n=1}^{\infty} E_n = \emptyset   
                    \]
                implying that 
                    \[
                        m\left( \bigcap_{n=1}^{\infty} E_n \right) = 0.
                    \]
                But having satisfied the conditions of Proposition 3.14, this is the same as saying \( E_n \to 0 \) as \( n \to \infty \) or
                    \[
                        m\left( \bigcap_{n=1}^{\infty} E_n \right) = \lim_{n \to \infty} E_n = 0.
                    \]
                Thus, because \( \epsilon \) is fixed, we can always find \( M \in \N \) such that
                    \[
                        m(E_M) = m\{x \in [a,b]: |f(x)| > M\} < \frac{\epsilon}{3}.
                    \]
                But this necessarily implies that \( |f(x)| \leq M \) for all \( x \in [a,b] \) thereby completing the proof.  
            \end{proof}
        \item Let \( f  \) be a measurable function on \( [a,b] \). Given \( \epsilon > 0 \) and \( M > 0 \), there is a simple function \( \phi \) such that \( |f(x) - \phi(x)| < \epsilon \) except where \( |f(x)| \geq M \). If \( m \leq f \leq M \), then we may take \( \phi \) so that \( m \leq \phi \leq M \).
            \begin{proof}

                Suppose \( f \) is a measurable function on \( [a,b] \). Let \( \epsilon > 0 \) and \( M > 0 \) be chosen. Because \( \epsilon \) and \( M \) are fixed, by the Archimedes principle, we can choose \( N \in \N \) large enough so that \( \displaystyle \frac{M}{N} < \epsilon \). From this, let us define the set   
                    \[
                        E_k =  \left\{ k \frac{M}{N} \leq f(x) \leq (k+1) \frac{M}{N} \right\}  
                    \]
                for \( k \in [-N, N] \) (integer-valued). Since \( f \) is a measurable function, each \( E_i \) is a measurable set as well. Let us define the function \( \phi \) by 
                    \[
                        \phi(x) = \sum_{k = -N}^{N} k \left( \frac{M}{N} \right) \chi_{E_{k}}
                    \]
                with \( \displaystyle a_i = k \frac{M}{N} \in \R \) for each \( k \in [-N, N] \). So because \( \phi \) is a linear combination of characteristic functions of \( E_i \) and each \( E_i \) is a measurable set (in fact, the \(E_i\)'s are pairwise disjoint), \( \phi \) is a simple function. Suppose that \( |f(x)| < M \). Because \( E_i \)'s are pairwise disjoint, then for all \( x \in [a, b] \), \( x \in E_k \) for some \( k \in [-N, N] \) which implies that 
                    \[
                        k \frac{M}{N} \leq f(x) \leq (k+1) \frac{M}{N}.
                    \]
                Thus, \( \displaystyle \phi(x) = k \frac{M}{N} \) which gives us that 
                    \begin{align*}
                        |f(x) - \phi(x) &= \left| f(x) - k \frac{M}{N} \right| \\
                        &< \frac{M}{N} \\
                        &< \epsilon.
                    \end{align*}
                Now suppose that \( f(x) \in [m, M] \) for all \( x \in [a,b] \) (i.e., \( f\) is a bounded function.) Then the same argument holds as before but instead we have that   
                    \begin{align*}
                        |f(x) - \phi(x) &= \left| f(x) - k \frac{M - m}{N} \right| \\
                        &< \frac{M - m}{N} \\
                        &< \epsilon
                    \end{align*}
                meaning for all \( x \in [a, b] \), we have \( \displaystyle \phi(x) = k \frac{M-m}{N} \) implying that \( \phi(x) \in [m, M] \).
            \end{proof}
        \item Given a simple function \( \phi \) on \( [a,b] \), there is a step function \( g \) on \( [a,b] \) such that \( g(x) =  \phi(x) \) except on a set of measure less than \( \displaystyle \frac{\epsilon}{3} \).
            \begin{proof}

                Let \( \phi \) by the simple function defined by
                    \[
                        \phi(x)  = \sum_{i=1}^{n} a_i \chi_{E_{i}}
                    \]
                for measurable, disjoints sets \( E_1, \ldots, E_n \) and \( a_i \in \R \) for \( i = 1, \ldots n \). Let \( \epsilon > 0 \) be chosen. Because each \( E_i \) is a measurable set, by Proposition 3.15, for each \( i = 1, \ldots n \), there exists a finite union \( U_i \) of open intervals \( I_{i}\) such that \( \displaystyle m(E_i \Delta U_i) < \frac{\epsilon}{3n}  \) with  
                    \[
                        U_i = \sum_{k = 1}^{N_i} I_{i, k}.\footnote{This is mostly for myself, but \( k \) is the index for the number of intervals \( N_i \) associated with each \( E_i \).}
                    \]
                Let \( \displaystyle A_i = U_i \setminus \left( \bigcup_{j=1}^{i - 1} U_j \right) \).\footnote{Again, mostly for myself, but for each \( U_i \) associated with \( E_i \), throw out the preceding \( U_i 's \).} For any \( x \in [a ,b] \), define the function
                    \[
                        g(x) = \sum_{i=1}^{n} a_i \chi_{A_{i}}.    
                    \]
                Because the \( E_i \)'s are measurable and the difference of measurable sets is measurable, the set \( \{ A_1, \ldots, A_n\} \) is a set of measurable sets. The \( A_i \)'s are a subdivision of \( [a, b] \) and so \( g \) is a step function per the definition on page 76 of Royden. We claim that this function is equal to \( \phi(x) \) except on a set of measure less than \( \displaystyle \frac{\epsilon}{3} \). To that end, fix \( x \in [a, b] \) so that \( \phi(x) \neq g(x) \). Because \( \phi \) and \( g \) are linear combinations with the same coefficients, this brings two cases: (i) there is some \( i  = 1, \ldots n \) so that \( g(x) = a_i \) but \( \phi(x) \neq a_i \) or (ii) there is some \( i = 1, \ldots, n \) so that \( g(x) \neq a_i \) but \( \phi(x) = a_i \). 
                
                For case (i), this means that \( x \in A_i \subset U_i \setminus E_i \) for some \( i = 1, \ldots, n \). For case (ii), we must have that \( x \in E_i \subset E_i \setminus U_i \) for some \( i = 1, \ldots n \). So, combining both results, 
                    \begin{align*}
                        \{ x \in [a,b]: \phi(x) \neq g(x) \} &\subset \bigcup_{i=1}^{n} U_i \setminus E_i; \\
                        \{ x \in [a,b]: \phi(x) \neq g(x) \} &\subset \bigcup_{i=1}^{n} E_i \setminus U_i
                    \end{align*}
                and thereby implies that 
                    \[
                        \{ x \in [a,b]: \phi(x) \neq g(x) \} \subset \bigcup_{i=1}^{n} E_i \Delta U_i.
                    \]
                Finally, this allows us to show that 
                    \begin{align*}
                        m \left( \{ x \in [a,b]: \phi(x) \neq g(x) \} \right) &\leq m \left( \bigcup_{i=1}^{n} E_i \Delta U_i \right) \\
                        &= \sum_{i=1}^{n} m \left( E_i \setminus U_i \right) \\
                        &< \sum_{i=1}^{n} \frac{\epsilon}{3n} \\
                        &= n \cdot \frac{\epsilon}{3n} \\
                        &= \epsilon
                    \end{align*}
                giving us the desired result.
            \end{proof}
    \end{enumerate}

\begin{problem}[3.31]

    Prove Lusin's Theorem: Let \( f \) be a measurable real-valued function on an interval \( [a,b] \). Then for all \( \delta > 0 \), there is a continuous function \( \phi \) on \( [a,b] \) such that \( m \{x: f(x) \neq \phi(x) \} < \delta \). 

    \begin{proof}

        Let \( \delta > 0 \) be chosen. Suppose \( f \) is a measurable real-valued function on an interval \( [a,b] \). Then by Proposition 3.22, there exists a continuous function \( h_n \) for all \( n \in \N \) such that 
            \[
                    |f - h_n| < \frac{\delta}{2^{n+2}}
            \]
        with \( \displaystyle m \left\{  x \in [a,b] : |f - h_n| \geq \frac{\delta}{2^{n+2}} \right\} < \frac{\delta}{2^{n+2}} \). For convenience, define the sets
            \[
                E_n =  \left\{  x \in [a,b] : |f - h_n| \geq \frac{\delta}{2^{n+2}} \right\}
            \]
        and 
            \[
                E = \bigcup_{n=1}^{\infty} E_n.    
            \]
        Note for a fixed \( x \in [a,b] \setminus E_n \) and any \( n \in \N \), we know by how we defined \( E_n \) that 
            \[
                 |f - h_n|  < \frac{\delta}{2^{n+2}}.
            \]
        Thus, since \( E \) is the union of the \( E_n \)'s, we have that
            \begin{align*}
                m(E) &= m\left( \bigcup_{n=1}^{\infty} E_n \right) \\
                     &\leq \sum_{n=1}^{\infty} m(E_n) \\
                     &< \sum_{n=1}^{\infty} \frac{\delta}{2^{n+2}} \\
                     &= \frac{\delta}{4}.
            \end{align*}
        So then on the set \( [a,b] \setminus E \), the sequence of continuous, and thereby, measurable functions \( \{ h_n\} \) converges to \( f \). Having satisfied the conditions of Egoroff's theorem, there exists a set \( A \subset [a,b] \setminus E \) with \( \displaystyle m(A) < \frac{\delta}{4} \) such that \( h_n \) converges uniformly on \( ([a,b] \setminus E) \setminus A = [a,b] \setminus (E \cup A )\). Since the uniform limit of continuous functions is a continuous function, the function \( f \) is continuous on \( [a,b] \setminus (E \cup A) \). Because \( \displaystyle m(E) \) and \( m(A) \) are less than \( \displaystyle \frac{\delta}{4} \), \(\displaystyle m(E \cup A) < \frac{\delta}{2} \).

        Using Proposition 3.15 part (ii), there exists an open set \( O \supset (E \cup A ) \) with 
            \[ 
                m(O \setminus (E \cup A)) < \frac{\delta}{2}.
            \]
        Because \( [a,b] \setminus (E \cup A) \supset [a,b] \setminus O \) and \( [a,b] \setminus O = [a,b] \cap O^{\C} \) (i.e., a closed set), \( f \) is continuous on the closed set \( [a,b] \setminus O \). Then for any \( x \in [a,b] \setminus O \), by Problem 2.40, there exists a continuous function \( \phi \) so that \( f(x) = \phi(x) \). But then the set \( O \) represents the set of points where \( \phi \) and \( g \) are not equal. In particular, we can show that 
            \begin{align*}
                m \{x \in [a,b]: f(x) \neq \phi(x) \} &= m(O) \\
                                            &= m( (O \setminus (E \cup A)) \cup (E \cup A)) \\
                                            &= m(O \setminus (E \cup A)) + m(E \cup A) \\
                                            &< \frac{\delta}{2} + \frac{\delta}{2} \\
                                            &= \delta
            \end{align*}
        which finally completes the proof.
    \end{proof}
    
\end{problem}
    
\end{problem}
\end{document}