\documentclass[12pt]{article}
\usepackage{amsmath,fullpage,graphicx,fancyhdr,enumerate,amsthm,amssymb,tikz,tikz-cd,color,pgfplots,tabularx,amsfonts,amscd}
\usepackage{enumitem}
\usetikzlibrary{arrows,chains,matrix,positioning,scopes}
\pgfplotsset{compat=1.13}
\setlength{\headheight}{15pt}
\setlength{\headsep}{7pt}
\newcommand{\bsnl}{\bigskip\newline}
\pagestyle{fancy}
\usepackage[paper=a4paper, margin = 1in]{geometry}
\usepackage{hyperref}% http://ctan.org/pkg/hyperref
\usepackage[capitalise, noabbrev]{cleveref}

\hypersetup{ % colors hyperlinks with colors to make noticeable but is not an ugly green box like default
    colorlinks,
    linkcolor={red!50!black},
    citecolor={blue!50!black},
    urlcolor={blue!80!black}
}
\usepackage[mathscr]{euscript}\usepackage{pifont}
\usepackage{cleveref}
\usepackage{parskip}
\usepackage[makeroom]{cancel}


\newcommand{\cmark}{\ding{51}}%
\newcommand{\xmark}{\ding{55}}%

\renewcommand{\aa}{\mathbf{a}}
\newcommand{\bb}{\mathbf{b}}
\newcommand{\cc}{\mathbf{c}}
\newcommand{\dd}{\mathbf{d}}
\newcommand{\FF}{\mathbf{F}}
\newcommand{\ii}{\mathbf{i}}
\newcommand{\jj}{\mathbf{j}}
\newcommand{\rr}{\mathbf{r}}
\newcommand{\kk}{\mathbf{k}}
\newcommand{\uu}{\mathbf{u}}
\newcommand{\vv}{\mathbf{v}}
\newcommand{\ww}{\mathbf{w}}
\newcommand{\yy}{\mathbf{y}}
\newcommand{\R}{\mathbb{R}}
\newcommand{\T}{\mathcal{T}}
\newcommand{\N}{\mathbb{N}}
\newcommand{\C}{\mathscr{C}}
\newcommand{\Z}{\mathbb{Z}}
\newcommand{\B}{\mathcal{B}}
\newcommand{\Q}{\mathbb{Q}}
\newcommand{\Sph}{\mathbb{S}}
\newcommand{\D}{\mathbb{D}}

\def\dim{\mathop{\rm dim}\nolimits}
\def\image{\mathop{\rm Im}\nolimits}  
\def\interior{\mathop{\rm Int}\nolimits}
\def\kernel{\mathop{\rm Ker}\nolimits}
\def\cokernel{\mathop{\rm Coker}\nolimits}
\def\bd{\mathop{\rm Bd}\nolimits}
\def\ext{\mathop{\rm Ext}\nolimits}
\def\num{\mathop{\#}\limits}
\def\cl{\mathop{\rm Cl}\nolimits}
\def\lub{\mathop{\rm lub}\nolimits}


\newcommand{\proj}{\operatorname{proj}}
\newcommand{\ds}{\displaystyle}
\newcommand{\pa}{\partial}
\newcommand{\ol}{\overline{}}


\newcommand{\degree}{^{\circ}}
\renewcommand{\epsilon}{\varepsilon}
\newcommand{\toM}{\overset{m}{\to}}


\newcommand{\upint}[2]{
  \overline{\int_{#1}^{#2}}
}
\newcommand{\lowint}[2]{
  \underline{\int_{#1}^{#2}}
}
\newcommand{\dif}{\, \mathrm{d}}
\newcommand{\norm}[1]{\left\lVert #1 \right\rVert}
\pagestyle{fancy}
\lhead{Math 6421: Measure Theory}
\chead{\bf Jose Nino: Homework \#6}
\rhead{October 19, 2023}
\cfoot{Page \thepage}
% \setlength{\headheight}{10pt}

\theoremstyle{definition}
\newtheorem*{thm}{Theorem}
\newtheorem*{exercise}{Exercise}
\newtheorem*{definition}{Definition}
\newtheorem{problem}{Problem}
\newtheorem{lemma}{Lemma}
\newtheorem*{cor}{Corollary}
\newtheorem*{prop}{Proposition}


\begin{document}

\begin{problem}[6.2]
    Let \( f \) be a bounded measurable function on \( [0, 1] \). Then
    \( \displaystyle \lim_{p \to \infty} \lVert f \rVert_{p} = \lVert f \rVert_{\infty} \).

        \begin{proof}
            First, I will note that \(\displaystyle \lim_{p \to \infty} \lVert f \rVert_{p} \leq \lVert f \rVert_{\infty}  \) follows pretty readily from the definition of \( \lVert \cdot \rVert \). This is because
                \[
                \lVert f \rVert_{p} = \left\{ \int_{0}^{1} |f|^p \right\}^{1/p}  \leq  \left\{ \int_{0}^{1} \lVert f \rVert^p_{\infty} \right\}^{1/p} = \lVert f \rVert_{\infty}
                \]
            and so as we take the limit of \( \lVert f \rVert_{p} \) as \( p \to \infty \), we get \(\displaystyle \lim_{p \to \infty} \lVert f \rVert_{p} \leq \lVert f \rVert_{\infty}  \).
            Now we must also show that \(\displaystyle \lim_{p \to \infty} \lVert f \rVert_{p} \geq \lVert f \rVert_{\infty}  \).
            To that end, let \( \epsilon > 0 \) be chosen. Define the set \( A = \{ x \in [0,1]: |f(x)| > \lVert f \rVert_{\infty} - \epsilon \} \).  
            Then we have that 
                \begin{align*}
                    \lVert f \rVert_{p} = \left\{ \int_{0}^{1} |f|^p \right\}^{1/p} &\geq \left\{ \int_{A} |f|^p \right\}^{1/p} \\
                     &\geq  \left\{ \int_{A} (\lVert f \rVert_{\infty} - \epsilon)^{p} \right\}^{1/p} \\ 
                     &= (\lVert f \rVert_{\infty} - \epsilon)^{p} \cdot m(A).
                \end{align*}
            This implies that 
                \[
                    \lVert f \rVert_{\infty} - \epsilon \cdot (m(A))^{p} \leq \lVert f \rVert_{p}.
                \]
            Because \( \lVert f \rVert_{\infty} \) is the essential supremum (i.e. the smallest, greatest value not on a set of measure zero), we know that \( m(A) >  0 \). 
            Thus, taking the limit of both sides as \( p \to \infty \), we get that 
                \[
                  \lim_{p \to \infty} \lVert f \rVert_{\infty} - \epsilon \cdot (m(A))^{p} = \lVert f \rVert_{\infty} - \epsilon \leq \lim_{p \to \infty} \lVert f \rVert_{p},
                \]
            Since \( \epsilon \) is arbitrary, then \( \displaystyle \lim_{p \to \infty} \lVert f \rVert \) is a superior bound i.e., \( \displaystyle \lVert f \rVert_{\infty} \leq \lim_{p \to \infty} \lVert f \rVert  \). Thus we get \( \displaystyle \lVert f \rVert_{\infty} = \lim_{p \to \infty} \lVert f \rVert\).
        \end{proof}
\end{problem}


\begin{problem}[6.8] Young's Inequality

    \begin{enumerate}[label = (\alph{*})]
        \item Let \( a, b \geq 0 \), \( 1 < p < \infty \), \( \displaystyle \frac{1}{p} + \frac{1}{q} = 1 \). Establish Young's inequality 
            \[
                ab \leq \frac{a^p}{p} + \frac{b^q}{q}.
            \]
            \begin{proof}
                Not assigned. 
            \end{proof}
        \item Use Young's inequality to give a proof of the H\"{o}lder inequality.
            \begin{proof}
                Let \( p \) and \( q \) be nonnegative extended real numbers such that
                    \[
                        \frac{1}{p} + \frac{1}{q} = 1.  
                    \]
                Suppose \( f \in L^p \) and \( g \in L^q \). Without loss of generality, assume that \( \norm{f}, \norm{g} \geq 0 \). 
                With \( \displaystyle a = \frac{|f|}{\norm{f}_p} \) and \( \displaystyle b = \frac{|g|}{\norm{g}_q} \), by Young's Inequality from part (a), we have that
                    \begin{align*}
                        \frac{|f|}{\norm{f}_p} \frac{|g|}{\norm{g}_q} \leq \frac{|f|^p}{p\norm{f}^{p}_{p}} \frac{|g|^q_p}{q\norm{g}^{q}_{q}}.
                    \end{align*}
                From the monotonocity of integrals, this implies that
                    \[
                        \int \frac{|f|}{\norm{f}_p} \frac{|g|}{\norm{g}_q} \leq \int \frac{|f|^p}{p\norm{f}_{p}^{p}} + \frac{|g|^q}{q\norm{g}^{q}_{q}} = \frac{1}{p \norm{f}^{p}_{p}} \int |f|^p + \frac{1}{q\norm{g}^{q}_{q}} \int |g|^q.
                    \]
                However, the the integral of \( |f|^p \) is the same as \( \norm{f}^p_p \) and the same argument for \( |g|^p \). 
                So by cancelling out \( \norm{f}^p_p \) and \( \norm{g}^q_q \), we get that
                    \[
                        \int \frac{|f|}{\norm{f}_p} \frac{|g|}{\norm{g}_q} \leq \frac{1}{p} + \frac{1}{q} = 1.
                    \]
                Multiplying both sides by \( \norm{f}_p \cdot \norm{g}_q \) and so
                    \[
                        \int |fg| \leq \norm{f}_p \cdot \norm{g}_q.
                    \]
                Young's inequality is equality if and only \( a^p = b^q \) and so the H\"{o}lder inequality is equality if and only if \( \displaystyle \frac{|f|^p}{\norm{f}^p_p} =  \frac{|g|^q}{\norm{g}_q^q} \). 
                Thus there exists \( \alpha, \beta \neq 0 \) such that \( \alpha |f|^p = \beta |g|^q \) almost everywhere and so this completes the proof.
            \end{proof}
    \end{enumerate}
    
\end{problem}


\begin{problem}[6.10]
    Let \( \{f_n\} \) be a sequence of functions in \( L^{\infty} \). Prove that \( \{f_n\} \) converges to \( f \) in \( L^{\infty} \) 
    if and only if there is a set \( E \) of measure zero such that \( f_n \) converges to \( f \) uniformly on \( E^{\C} \).

        \begin{proof}
            We will need to complete two directions and so let \( \{f_n\} \) be a sequence of functions in \( L^{\infty} \).  
                \begin{enumerate}
                    \item[\((\Rightarrow)\)] First, suppose that \( \{f_n\} \to f \), and let \( \epsilon > 0 \) be chosen. Because \( f_n \to f \) in \( L^{\infty} \), there exists \( N \in \N \) such that for all \( n \geq N \),
                        \[
                            \norm{f_n - f}_{\infty} = \inf \left\{ M: m\left\{ t : |f_n(t) - f(t| > M \right\} = 0 \right\} < \epsilon.  
                        \]
                    Let \( \displaystyle E = \left\{ t: |f_n(t) - f| \geq \epsilon \right\} \).
                    Per the above expression, for any \( n \geq N \), we have that \( m(E) = 0 \) and \( \norm{f_n - f }_{\infty} < \epsilon \) on the set \( L^{\infty} \setminus E = E^{\C} \). Thus, since \( \epsilon > 0 \) is arbitrary, \( f_n \) converges uniformly to \( f \) on \( E^{\C} \).
                    \item[(\(\Leftarrow\))] 
                    Conversely, suppose there exists a set \( E \) with \( m(E) = 0 \) such that \( f_n \to f \) uniformly on \( E^{\C} \). 
                    Let \( \epsilon > 0 \) be chosen. Since \( f_n \to f \) uniformly on \( E^{\C}\), there exists \( N \in \N \) such that for all \( n \geq N \) and \( t \in E^{\C}\),
                        \[
                             |f_n(t) - f(t)| < \frac{\epsilon}{2}.
                        \]
                    But then this means that the set \( \displaystyle \left\{t: f_n(t) - f(t) > \frac{\epsilon}{2}  \right\} \subset E \). By the definition of the infimum, for our fixed \( \epsilon > 0 \) and any \( n \geq N \), \[ \inf \left\{ M : m \left\{ t : |f_n(t) - f(t)| > M \right\}  = 0 \right\} < \epsilon. \]
                    This is the essential supremum and so this means \( \norm{f_n - f}_{\infty} < \epsilon \). 
                    Therefore, since \( \epsilon \) is arbitrary, \( \norm{f_n - f} < \epsilon \) and which implies that \( f_n \to f \) pointwise on \( L^{\infty} \).
                \end{enumerate}
            Thus, having completed the forward and backwards implication, this completes the proof.
        \end{proof}

\end{problem}


\end{document}