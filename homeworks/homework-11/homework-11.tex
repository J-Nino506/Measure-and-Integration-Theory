\documentclass[12pt]{article}
\usepackage{amsmath,fullpage,graphicx,fancyhdr,enumerate,amsthm,amssymb,tikz,tikz-cd,color,pgfplots,tabularx,amsfonts,amscd}
\usepackage{enumitem}
\usetikzlibrary{arrows,chains,matrix,positioning,scopes}
\pgfplotsset{compat=1.13}
\setlength{\headheight}{15pt}
\setlength{\headsep}{7pt}
\newcommand{\bsnl}{\bigskip\newline}
\pagestyle{fancy}
\usepackage[paper=a4paper, margin = 1in]{geometry}
\usepackage{hyperref}% http://ctan.org/pkg/hyperref
\usepackage[capitalise, noabbrev]{cleveref}

\hypersetup{ % colors hyperlinks with colors to make noticeable but is not an ugly green box like default
    colorlinks,
    linkcolor={red!50!black},
    citecolor={blue!50!black},
    urlcolor={blue!80!black}
}
\usepackage[mathscr]{euscript}\usepackage{pifont}
\usepackage{cleveref}
\usepackage{parskip}
\usepackage[makeroom]{cancel}


\newcommand{\cmark}{\ding{51}}%
\newcommand{\xmark}{\ding{55}}%

\renewcommand{\aa}{\mathbf{a}}
\newcommand{\bb}{\mathbf{b}}
\newcommand{\cc}{\mathbf{c}}
\newcommand{\dd}{\mathbf{d}}
\newcommand{\FF}{\mathbf{F}}
\newcommand{\ii}{\mathbf{i}}
\newcommand{\jj}{\mathbf{j}}
\newcommand{\rr}{\mathbf{r}}
\newcommand{\kk}{\mathbf{k}}
\newcommand{\uu}{\mathbf{u}}
\newcommand{\vv}{\mathbf{v}}
\newcommand{\ww}{\mathbf{w}}
\newcommand{\yy}{\mathbf{y}}
\newcommand{\R}{\mathbb{R}}
\newcommand{\T}{\mathcal{T}}
\newcommand{\N}{\mathbb{N}}
\newcommand{\C}{\mathscr{C}}
\newcommand{\Z}{\mathbb{Z}}
\newcommand{\B}{\mathcal{B}}
\newcommand{\Q}{\mathbb{Q}}
\newcommand{\Sph}{\mathbb{S}}
\newcommand{\D}{\mathbb{D}}

\def\dim{\mathop{\rm dim}\nolimits}
\def\image{\mathop{\rm Im}\nolimits}  
\def\interior{\mathop{\rm Int}\nolimits}
\def\kernel{\mathop{\rm Ker}\nolimits}
\def\cokernel{\mathop{\rm Coker}\nolimits}
\def\bd{\mathop{\rm Bd}\nolimits}
\def\ext{\mathop{\rm Ext}\nolimits}
\def\num{\mathop{\#}\limits}
\def\cl{\mathop{\rm Cl}\nolimits}
\def\lub{\mathop{\rm lub}\nolimits}
\def\sgn{\mathop{\rm sgn}\nolimits}



\newcommand{\proj}{\operatorname{proj}}
\newcommand{\ds}{\displaystyle}
\newcommand{\pa}{\partial}
\newcommand{\ol}{\overline{}}


\newcommand{\degree}{^{\circ}}
\renewcommand{\epsilon}{\varepsilon}
\newcommand{\toM}{\overset{m}{\to}}


\newcommand{\upint}[2]{
  \overline{\int_{#1}^{#2}}
}
\newcommand{\lowint}[2]{
  \underline{\int_{#1}^{#2}}
}
\newcommand{\dif}{\, \mathrm{d}}
\newcommand{\norm}[1]{\left\lVert #1 \right\rVert}
\newcommand{\abs}[1]{\left\lvert #1 \right\rvert}


\pagestyle{fancy}
\lhead{Math 6421: Measure Theory}
\chead{\bf Jose Nino: Homework \#11}
\rhead{November 30, 2023}
\cfoot{Page \thepage}
% \setlength{\headheight}{10pt}

\theoremstyle{definition}
\newtheorem*{thm}{Theorem}
\newtheorem*{exercise}{Exercise}
\newtheorem*{definition}{Definition}
\newtheorem{problem}{Problem}
\newtheorem{lemma}{Lemma}
\newtheorem*{cor}{Corollary}
\newtheorem*{prop}{Proposition}


\begin{document}

\begin{problem}[11.34] Let \( \mu \),  \( \nu \), and \( \lambda \) be \( \sigma \)-finite. Show that the Radon-Nikodym derivative \( [\dif\nu/ \dif\mu] \) has the following properties:
    \begin{enumerate}[label = \alph{*}.]
        \item  If \( \nu \ll \mu \) and \( f \) is nonnegative measurable function, then 
            \[
                \int f \dif \nu = \int f \left[ \frac{\dif \nu}{\dif \mu}       \right] \dif \mu.    
            \]
                \begin{proof}
                    Let \( v \ll u \) and suppose \( f \) is a nonnegative measurable function over a set \( E \). We can break this down into two cases: (i) \( f \) is a simple function or (ii) \( f \) is a non-simple, measurable function. 
                    If \( f \) is a simple function, then
                        \[
                            f = \sum_{i=1}^{n} a_i \chi_{E_{i}}
                        \]
                    for \( a_i \in \R \). Then, using properties of simple functions and integration, we have the following:
                        \begin{align*}
                            \int_{E} f \dif \nu &= \sum_{i=1}^{n} a_i \nu(E_i)\\
                            &= \sum_{i=1}^{n} a_i \left( \int_{E_i} f \dif \mu \right) \\
                            &= \sum_{i=1}^{n} a_i \left( \int_{E_i} \left[ \frac{\dif \nu}{\dif \mu} \right] \dif \mu \right) \\
                            &= \int_{E} \sum_{i=1}^{n} a_i \chi_{E_{i}} \left[ \frac{\dif \nu}{\dif \mu} \right] \dif \mu \\
                            &= \int f \left[ \frac{\dif \nu}{\dif \mu} \right] \dif \mu.
                        \end{align*}
                Turning to case (ii), suppose that \( f \) is a non-simple measurable function.
                Then there exists a sequence of increasing simple functions \( \{ \phi_n\} \) such that \( \phi_n \to f \) pointwise. 
                Using the Monotone Convergence Theorem, we have 
                        \begin{align*}
                            \int_{E} f \dif \nu &= \lim_{n \to \infty} \int_{E} \phi_n \dif \nu \\
                            &= \lim_{n \to \infty} \int_{E} \phi_n \left[ \frac{\dif \nu}{\dif \mu} \right] \dif \mu \\
                            &= \int_{E} f \left[ \frac{\dif \nu}{\dif \mu} \right] \dif \mu.
                        \end{align*}
                Having exhausted all cases, this completes the proof. 
                \end{proof}
        \item  Not assigned.
        \item If \( \nu \ll \mu \ll \lambda \), then
            \[
                \left[ \frac{\dif \nu}{\dif \lambda} \right] = \left[ \frac{\dif \nu}{\dif \mu} \right] \left[ \frac{\dif \mu}{\dif \lambda} \right]. 
            \]
                \begin{proof}
                    Let \( \nu \ll \mu \ll \lambda \), and let \( E \) any measurable set.
                    Then, by definition, 
                        \[
                            \nu(E) = \int_{E} \left[ \frac{\dif \nu}{\dif \mu} \right] \dif \mu. 
                        \]
                    Because \( \mu \ll \lambda \), we also have that (from part (a))
                        \[
                            \int_{E} f \dif \mu = \int_{E} f \left[ \frac{\dif \mu}{\dif \lambda} \right] \dif \lambda.  
                        \]
                    Combining these two, we have that
                        \begin{align*}
                            \nu(E) &= \int_{E} \left[ \frac{\dif \nu}{\dif \mu} \right] \dif \mu.  \\
                            &= \int_{E} \left[ \frac{\dif \nu}{\dif \mu} \right] \left[ \frac{\dif \mu}{\dif \lambda} \right] \dif \lambda
                        \end{align*}
                    and so it follows that
                        \[
                            \left[ \frac{\dif \nu}{\dif \lambda} \right] = \left[ \frac{\dif \nu}{\dif \mu} \right] \left[ \frac{\dif \mu}{\dif \lambda} \right]
                        \]
                    which completes the proof. 
                \end{proof}
    \end{enumerate}
    
\end{problem}

\begin{problem}[11.45]
    For \( g \in L^q \), let \( F \) be the linear functional on \( L^p \) defined by 
        \[
            F(f) = \int fg \dif \mu.  
        \]
    Show that \( \norm{F} = \norm{g}_{q} \). 

        \begin{proof}
            Let \( g \in L^q \), and let \( F \) be the linear functional on \( L^p \) be defined by
                \[
                    F(f) = \int fg \dif \mu.  
                \]
            From the H\"{o}lder inequality, we have
                \begin{align*}
                    \abs{F(f)} &= \abs{\int fg \dif \mu} \\
                    &\leq \int \abs{fg} \dif \mu \\
                    &\leq \norm{f}_p \cdot \norm{g}_q.
                \end{align*}
            So dividing over by \( \displaystyle \norm{f}_p \), we have 
                \[
                    \frac{\abs{F(f)}}{\norm{f}_p} \leq \norm{g}_q  
                \]
            which, by taking the supremum of the left-hand side over all \( f \in L^p \), implies that
                \[
                    \norm{F} \leq \norm{g}_q.  
                \]\

            To show the other inequality, for \( p \in (1, \infty) \), we have three cases: (i) \( p \in (1, \infty) \) and \( q \in (1, \infty) \); (ii) \( p = \infty \) and \( q = 1 \); or (iii) \( p = 1 \) and \( q = \infty \).
            
            First, suppose \( p, q \in (1, \infty)\). Define the function \( f \) by
                \[
                    f =  \abs{g}^{q/p} \cdot \sgn g.
                \]
            Then \( \abs{f}^p = \abs{f}^q = fg \) meaning \( f \in L^p \). This implies
                \begin{align*}
                    \norm{f}_p = \norm{g}^{p/q}_{p}
                \end{align*}
            and so
                \begin{align*}
                    \abs{F(f)} &= \abs{\int fg \dif \mu} \\ 
                    &= \int \abs{g}^q \dif \mu  \\
                    &= \norm{g}^q_q  \\
                    &= \norm{g}_q \norm{f}_p
                \end{align*}
            and by definition of the norm of the linear functional,
                \[
                    \norm{F} \geq \norm{g}_q.  
                \]
            Suppose \( p = \infty \) and \( q = 1 \) meaning \( \norm{g}_1 \). 
            Without loss of generality, assume \( \norm{g}_1 \). 
            Let \( f = \sgn g \). 
            Then \( f \in L^{\infty} \) and so \( \norm{f}_{\infty} = 1 \) from how we defined \( f \). 
            So
                \begin{align*}
                    \abs{F(f)} &= \abs{\int fg \dif \mu} \\
                    &= \int \abs{g} \dif \mu \\
                    &= \norm{g}_1 \\
                    &= \norm{g}_1 \norm{f}_{\infty}
                \end{align*}
            and by a similar argument as in case (i),
                \[
                    \norm{F} \geq \norm{g}_q.  
                \]
            Finally, suppose \( p  = 1 \) but \( q = \infty \), and let \( \epsilon > 0 \) be chosen. 
            Define the set \( E = \{ x: g(x) > \norm{g}_{\infty} - \epsilon \} \) and define \( f = \chi_{E} \). 
            This means \( f \in L^1 \) and
                \[
                    \norm{f}_1 = \int \abs{f} \dif \mu  = \mu(E).
                \]  
            Also,
                \begin{align*}
                    \norm{F(f)} &= \abs{\int fg \dif \mu } \\
                    &= \abs{\int_E g \dif \mu} \\
                    &\leq \left( \norm{g}_{\infty} - \epsilon \right) \norm{f}_1
                \end{align*}
            and by moving \( \norm{f}_1 \) to the other side, we have that
                \[
                    \norm{F(f)} \geq \norm{g}_1.  
                \]
            Therefore, having exhausted all possible cases, this completes the proof. 
        \end{proof}
\end{problem}

\end{document}