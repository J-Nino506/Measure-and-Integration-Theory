\documentclass[12pt]{article}
\usepackage{amsmath,fullpage,graphicx,fancyhdr,enumerate,amsthm,amssymb,tikz,tikz-cd,color,pgfplots,tabularx,amsfonts,amscd}
\usepackage{enumitem}
\usetikzlibrary{arrows,chains,matrix,positioning,scopes}
\pgfplotsset{compat=1.13}
\setlength{\headheight}{15pt}
\setlength{\headsep}{7pt}
\newcommand{\bsnl}{\bigskip\newline}
\pagestyle{fancy}
\usepackage[paper=a4paper, margin = 1in]{geometry}
\usepackage{hyperref}% http://ctan.org/pkg/hyperref
\usepackage[capitalise, noabbrev]{cleveref}

\hypersetup{ % colors hyperlinks with colors to make noticeable but is not an ugly green box like default
    colorlinks,
    linkcolor={red!50!black},
    citecolor={blue!50!black},
    urlcolor={blue!80!black}
}
\usepackage[mathscr]{euscript}\usepackage{pifont}
\usepackage{cleveref}
\usepackage{parskip}
\usepackage[makeroom]{cancel}


\newcommand{\cmark}{\ding{51}}%
\newcommand{\xmark}{\ding{55}}%

\renewcommand{\aa}{\mathbf{a}}
\newcommand{\bb}{\mathbf{b}}
\newcommand{\cc}{\mathbf{c}}
\newcommand{\dd}{\mathbf{d}}
\newcommand{\FF}{\mathbf{F}}
\newcommand{\ii}{\mathbf{i}}
\newcommand{\jj}{\mathbf{j}}
\newcommand{\rr}{\mathbf{r}}
\newcommand{\kk}{\mathbf{k}}
\newcommand{\uu}{\mathbf{u}}
\newcommand{\vv}{\mathbf{v}}
\newcommand{\ww}{\mathbf{w}}
\newcommand{\yy}{\mathbf{y}}
\newcommand{\R}{\mathbb{R}}
\newcommand{\T}{\mathcal{T}}
\newcommand{\N}{\mathbb{N}}
\newcommand{\C}{\mathscr{C}}
\newcommand{\Z}{\mathbb{Z}}
\newcommand{\B}{\mathcal{B}}
\newcommand{\Q}{\mathbb{Q}}
\newcommand{\Sph}{\mathbb{S}}
\newcommand{\D}{\mathbb{D}}

\def\dim{\mathop{\rm dim}\nolimits}
\def\image{\mathop{\rm Im}\nolimits}  
\def\interior{\mathop{\rm Int}\nolimits}
\def\kernel{\mathop{\rm Ker}\nolimits}
\def\cokernel{\mathop{\rm Coker}\nolimits}
\def\bd{\mathop{\rm Bd}\nolimits}
\def\ext{\mathop{\rm Ext}\nolimits}
\def\num{\mathop{\#}\limits}
\def\cl{\mathop{\rm Cl}\nolimits}
\def\lub{\mathop{\rm lub}\nolimits}


\newcommand{\proj}{\operatorname{proj}}
\newcommand{\ds}{\displaystyle}
\newcommand{\pa}{\partial}
\newcommand{\ol}{\overline{}}


\newcommand{\degree}{^{\circ}}
\renewcommand{\epsilon}{\varepsilon}

\newcommand{\upint}[2]{
  \overline{\int_{#1}^{#2}}
}
\newcommand{\lowint}[2]{
  \underline{\int_{#1}^{#2}}
}
\newcommand{\dif}{\, \mathrm{d}}

\pagestyle{fancy}
\lhead{Math 6421: Measure Theory}
\chead{\bf Jose Nino: Homework \#5}
\rhead{October 5, 2023}
\cfoot{Page \thepage}
% \setlength{\headheight}{10pt}

\theoremstyle{definition}
\newtheorem*{thm}{Theorem}
\newtheorem*{exercise}{Exercise}
\newtheorem*{definition}{Definition}
\newtheorem{problem}{Problem}
\newtheorem{lemma}{Lemma}
\newtheorem*{cor}{Corollary}
\newtheorem*{prop}{Proposition}


\begin{document}

\begin{problem}[4.14] 
    Some sequence and integral convergence problems.
    \begin{enumerate}[label = (\alph{*})]
        \item Show that under the hypotheses of Theorem 4.17 we have
            \[
                \int \left| f_n - f \right| \to 0.  
            \]
        \item Let \( \{f_n\} \) be a sequence of integrable functions such that \( f_n \to f \) almost everywhere with \( f \) integrable. Then \( \displaystyle \int |f - f_n| \to 0 \) if and only if \( \displaystyle \int |f_n| \to \int |f| \).
    \end{enumerate}


\end{problem}

\begin{problem}[4.16]
    Establish the \emph{Riemann-Lebesgue Theorem}: If \( f \) is an integrable function on 
    \( -\infty, \infty \), then \( \displaystyle \lim_{n \to \infty} \int_{-\infty}^{\infty} f(x) \cos(nx) \dif x = 0 \). [Hint: The theorem is easy if \( f \) is a step function. Use Problem 15.]


\end{problem}  

\begin{problem}[4.25]

    A sequence \( \{f_n\} \) of measurable functions is said to be a Cauchy sequence in measure if given \( \epsilon > 0 \), there is \( N \in \N \) such that for all \( m, n \geq N \) we have 
        \[
            m \left\{ x: \left| f_n(x) - f_m(x) \right| \geq \epsilon \right\} < \epsilon.
        \]
    Show that if \( \{f_n\} \) is a Cauchy sequence in measure, 
    then there is a function \( f \) to which the sequence \( \{f_n\} \) converges in measure. 
    
\end{problem}


\begin{problem}

    Compute \( \displaystyle \lim_{n \to \infty} \int_{0}^{1} (1 + nx^{2})(1 + x^{2})^{-2} \dif x \). Justify your answer.
    
\end{problem}

\end{document}