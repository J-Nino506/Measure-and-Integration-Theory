\documentclass[12pt]{article}
\usepackage{amsmath,fullpage,graphicx,fancyhdr,enumerate,amsthm,amssymb,tikz,tikz-cd,color,pgfplots,tabularx,amsfonts,amscd}
\usepackage{enumitem}
\usetikzlibrary{arrows,chains,matrix,positioning,scopes}
\pgfplotsset{compat=1.13}
\setlength{\headheight}{15pt}
\setlength{\headsep}{7pt}
\newcommand{\bsnl}{\bigskip\newline}
\pagestyle{fancy}
\usepackage[paper=a4paper, margin = 1in]{geometry}
\usepackage{hyperref}% http://ctan.org/pkg/hyperref
\usepackage[capitalise, noabbrev]{cleveref}

\hypersetup{ % colors hyperlinks with colors to make noticeable but is not an ugly green box like default
    colorlinks,
    linkcolor={red!50!black},
    citecolor={blue!50!black},
    urlcolor={blue!80!black}
}
\usepackage[mathscr]{euscript}\usepackage{pifont}
\usepackage{cleveref}
\usepackage{parskip}
\usepackage[makeroom]{cancel}


\newcommand{\cmark}{\ding{51}}%
\newcommand{\xmark}{\ding{55}}%

\renewcommand{\aa}{\mathbf{a}}
\newcommand{\bb}{\mathbf{b}}
\newcommand{\cc}{\mathbf{c}}
\newcommand{\dd}{\mathbf{d}}
\newcommand{\FF}{\mathbf{F}}
\newcommand{\ii}{\mathbf{i}}
\newcommand{\jj}{\mathbf{j}}
\newcommand{\rr}{\mathbf{r}}
\newcommand{\kk}{\mathbf{k}}
\newcommand{\uu}{\mathbf{u}}
\newcommand{\vv}{\mathbf{v}}
\newcommand{\ww}{\mathbf{w}}
\newcommand{\yy}{\mathbf{y}}
\newcommand{\R}{\mathbb{R}}
\newcommand{\T}{\mathcal{T}}
\newcommand{\N}{\mathbb{N}}
\newcommand{\C}{\mathscr{C}}
\newcommand{\Z}{\mathbb{Z}}
\newcommand{\B}{\mathcal{B}}
\newcommand{\Q}{\mathbb{Q}}
\newcommand{\Sph}{\mathbb{S}}
\newcommand{\D}{\mathbb{D}}

\def\dim{\mathop{\rm dim}\nolimits}
\def\image{\mathop{\rm Im}\nolimits}  
\def\interior{\mathop{\rm Int}\nolimits}
\def\kernel{\mathop{\rm Ker}\nolimits}
\def\cokernel{\mathop{\rm Coker}\nolimits}
\def\bd{\mathop{\rm Bd}\nolimits}
\def\ext{\mathop{\rm Ext}\nolimits}
\def\num{\mathop{\#}\limits}
\def\cl{\mathop{\rm Cl}\nolimits}
\def\lub{\mathop{\rm lub}\nolimits}


\newcommand{\proj}{\operatorname{proj}}
\newcommand{\ds}{\displaystyle}
\newcommand{\pa}{\partial}
\newcommand{\ol}{\overline{}}


\newcommand{\degree}{^{\circ}}
\renewcommand{\epsilon}{\varepsilon}

\newcommand{\upint}[2]{
  \overline{\int_{#1}^{#2}}
}
\newcommand{\lowint}[2]{
  \underline{\int_{#1}^{#2}}
}
\newcommand{\dif}{\, \mathrm{d}}

\pagestyle{fancy}
\lhead{Math 6421: Measure Theory}
\chead{\bf Jose Nino: Homework \#4}
\rhead{September 28, 2023}
\cfoot{Page \thepage}
% \setlength{\headheight}{10pt}

\theoremstyle{definition}
\newtheorem*{thm}{Theorem}
\newtheorem*{exercise}{Exercise}
\newtheorem*{definition}{Definition}
\newtheorem{problem}{Problem}
\newtheorem{lemma}{Lemma}
\newtheorem*{cor}{Corollary}
\newtheorem*{prop}{Proposition}


\begin{document}

\begin{problem}[4.2] 



    \begin{enumerate}[label = (\alph{*})]
    
        \item Let \( f \) be a bounded function on \( [a,b] \), and let \( h \) be the upper envelope of \( f \) (cf. Problem 2.51). Then \( \displaystyle R \upint{a}{b} f = \int_{a}^{b} h \).
        
            \begin{proof}
                Let \( f \) be a bounded function on \( [a,b] \) with \( \displaystyle h(y) = \inf_{\delta > 0} \sup_{|x-y| < \delta} f(x) \) for all \( x \in [a,b] \) be the upper envelope of \( f \). 
                Because \( f \) is bounded, by Problem 2.51 part (b), \( h \) is lower semicontinuous. 
                To show equality, we will show that 
                    \[
                        R \upint{a}{b} f \leq \int_{a}^{b} h \quad \text{and} \quad R \upint{a}{b} f \geq \int_{a}^{b} h.
                    \]
                
                Let \( \phi \) be a step function on \( [a,b] \) such that \( \phi \geq f \).
                Then for any \( x \in [a,b] \), \( h(x) \leq f(x) \leq \phi(x) \), except at the defined partition points of \( \phi \). 
                Thus, by using the definition of Lebesgue integration, we have that 
                    \[
                        \int_{a}^{b} h(x) \dif x \leq \int_{a}^{b} f(x) \dif x = \inf \int_{a}^{b} \phi(x) \dif x \leq R \upint{a}{b} f(x) \dif x.
                    \]
                For the other inequality, we note that because \( h \) is upper semicontinuous, by Problem 2.51 part (g), there exists a monotonically decreasing sequence of step functions \( \{\phi_n \} \) such that \( \phi_n \to h \) pointwise. 
                Because \( f \) is bounded, we have that for all \( x \in [a,b] \), there exists some \( M > 0 \) such that 
                    \[
                        |\phi_n| \leq |h| \leq |f| \leq M \ \text{for all} \ n \in \N.   
                    \]
                Thus using the Bounded Convergence Theorem and properties of the upper Riemann integral, 
                    \[
                          \lim_{n \to \infty} \int_{a}^{b} \phi_n(x) \dif x = \int_{a}^{b} h(x) \dif x \leq R \int_{a}^{b} f(x) \dif x \leq R \upint{a}{b} f(x) \dif x.
                    \]
                Therefore, 
                    \[
                        R \upint{a}{b} f = \int_{a}^{b} h
                    \]
                which is the desired result.
            \end{proof}
        \item Use part (a) to prove Proposition 7 which is stated as follows
            \begin{prop}[\textbf{4.7}]
                A bounded function \( f \) on \( [a,b] \) is Riemann integrable if and only if the set of points at which \( f \) is discontinuous has measure zero. 

                \begin{proof}
                    Let \( f \) be a bounded function on \( [a,b] \). We will need to show a forward and backwards implication to complete this proof. For simplicity, define \( E \) to be the set of discontinuities of \( f \). Additionally, let \( \displaystyle g(y) = \sup_{\delta > 0} \inf_{|x-y| < \delta} f(x) \) be the lower envelope of \( f \).

                        \begin{enumerate} 
                            \item[(\(\Leftarrow\))] First, suppose \( m(E) =  0 \). Since \( g \) is the lower envelope of \( f \), there exists a monotonically increasing sequence of step functions \( \{ \phi_n\} \to g \) pointwise. Thus, by a similar but reverse argument to part (a),
                            \begin{equation}
                                \label{4.1_eq-1}
                                \int_{a}^{b} g(x) \dif x = \lowint{a}{b} f(x) \dif x.  
                            \end{equation}
                            So because \( f \) is continuous everywhere except on the set \( E \)---namely, continuous on \( [a,b] \setminus E\)--- by Problem 2.51, \( g(x) = h(x) \) is continuous on the set \( [a,b] \setminus E \). But since \( m(E) = 0 \), this means \( g = h \) almost everywhere and thus, using part (a) of this problem and \cref{4.1_eq-1}, 
                                \[
                                    \lowint{a}{b} f(x) \dif x = \int_{a}^{b} g(x) \dif x = \int_{a}^{b} h(x) \dif x = \upint{a}{b} f(x) \dif x.
                                \]
                            Thus, \( f \) is Riemann integrable. 
                            \item[(\(\Rightarrow\))] Now suppose \( f \) is Riemann integrable. Thus, the lower and upper integrals of \( f \) are equal to each other and so using \cref{4.1_eq-1}
        
                                \[
                                    \int_{a}^{b} h(x) \dif x = \int_{a}^{b} g(x) \dif x.
                                \]
                            Consider the set \( \displaystyle A_n = \left\{ x:  | g(x) - h(x)| > \frac{1}{n} \right\} \) for all \( n \in \N \). Because the integrals of \( g \) and \( h \) are equal,    
                                \[
                                    \int_{a}^{b} |h(x)- g(x)| \dif x = 0.
                                \]
                            So for any fixed \( n \in \N \),    
                                \[
                                    \int_{a}^{b} |h(x)- g(x)| \dif x \geq m(A_n).
                                \]
                            So \( h(x) = g(x) \) almost everywhere and so by Problem 2.51 part(a), we must that \( f \) is continuous almost everywhere as well. But then this means that the measure of discontinuities is zero---that is, \( m(E) = 0 \) which is what we wanted to show. 
                        \end{enumerate}
                    Having completed both directions, we have shown the desired equivalence. 
                \end{proof}
                
            \end{prop}
    \end{enumerate}


\end{problem}

\begin{problem}[4.3]

    Let \( f \) be a nonnegative measurable function. Show that \( \displaystyle \int f = 0 \) implies \( f = 0 \) almost everywhere. 

        \begin{proof}
            Let \( f \geq 0 \) be a measurable function, and suppose that \(\displaystyle \int f = 0 \). We want to show that the set \( E = {x: f(x) \neq 0} = \{x: f(x) > 0 \} \) has measure \( 0 \). Define the set
                \[
                    E_n = \left\{ x: f(x) \geq \frac{1}{n} \right\} \text{for all} \ n \in \N.  
                \]
            Note that \( \displaystyle \bigcup_{n=1}^{\infty} E_n = E \). Fix \( n \in \N \). Because the integral of \( f \) is equal to \( 0 \), 
                \[
                    0 = \int_{E_{n}} \geq \int_{E_{n}}  \frac{1}{n} = m(E_n) \cdot \frac{1}{n} \geq 0.
                \]
            Thus, because \( n \in \N \) was fixed, \( \displaystyle m\left( \bigcup_{n=1}^{\infty} E_n \right) =  0 = m(E) \), and therefore \( f = 0 \) almost everywhere.
        \end{proof}

\end{problem}  

\begin{problem}[4.8]

    Prove the following generalization of Fatuo's Lemma: If \( {f_n}\) is a sequence of nonnegative functions then 
        \[
            \int \varliminf_{n \to \infty} f_n \leq \varliminf_{n \to \infty} \int f_n.  
        \]

        \begin{proof}
            Let \( \{ f_n\} \geq 0 \) for each \( n \in \N \) on any set \( E \).
            Define \( \displaystyle h_n = \inf_{k \geq n} f_k \) for all \( n \in \N \). Note that as \( n \to \infty\), \( \displaystyle h_n \to \varliminf_{n \to \infty} f_n \) (i.e., \( h_n\) converges pointwise on \( E \) to the limit inferior of \( f_n\)). Thus, by Fatou's lemma, we have that
                \[
                    \int_{E} \varliminf_{n \to \infty} f \leq \int_{E} \varliminf_{n \to \infty} h_n.    
                \]
            But since \( h_n \) is the infimum of the \( f_n \)'s, this implies that \( h_n  \leq f_n \) for all \( n \in \N \) and so
                \[
                  \int_{E} h_n \leq \int_{E} f_n 
                \]
            and thus
                \[
                    \varliminf_{n \to \infty} \int_{E} h_n \leq \varliminf_{n \to \infty} \int_{E} f_n. 
                \]
            By combining all three inequalities, we get that
                \[
                    \int_{E} \varliminf_{n \to \infty} f \leq \varliminf_{n \to \infty} \int_{E} h_n \leq \varliminf_{n \to \infty} \int_{E} f_n
                \]
            which then completes the proof.           
        \end{proof}
    
\end{problem}

\end{document}