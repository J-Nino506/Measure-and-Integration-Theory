\documentclass[12pt]{article}
\usepackage{amsmath,fullpage,graphicx,fancyhdr,enumerate,amsthm,amssymb,tikz,tikz-cd,color,pgfplots,tabularx,amsfonts,amscd}
\usepackage{enumitem}
\usetikzlibrary{arrows,chains,matrix,positioning,scopes}
\pgfplotsset{compat=1.13}
\setlength{\headheight}{15pt}
\setlength{\headsep}{7pt}
\newcommand{\bsnl}{\bigskip\newline}
\pagestyle{fancy}
\usepackage[paper=a4paper, margin = 1in]{geometry}
\usepackage{hyperref}% http://ctan.org/pkg/hyperref
\usepackage[capitalise, noabbrev]{cleveref}

\hypersetup{ % colors hyperlinks with colors to make noticeable but is not an ugly green box like default
    colorlinks,
    linkcolor={red!50!black},
    citecolor={blue!50!black},
    urlcolor={blue!80!black}
}
\usepackage[mathscr]{euscript}\usepackage{pifont}
\usepackage{cleveref}
\usepackage{parskip}
\usepackage[makeroom]{cancel}


\newcommand{\cmark}{\ding{51}}%
\newcommand{\xmark}{\ding{55}}%

\renewcommand{\aa}{\mathbf{a}}
\newcommand{\bb}{\mathbf{b}}
\newcommand{\cc}{\mathbf{c}}
\newcommand{\dd}{\mathbf{d}}
\newcommand{\FF}{\mathbf{F}}
\newcommand{\ii}{\mathbf{i}}
\newcommand{\jj}{\mathbf{j}}
\newcommand{\rr}{\mathbf{r}}
\newcommand{\kk}{\mathbf{k}}
\newcommand{\uu}{\mathbf{u}}
\newcommand{\vv}{\mathbf{v}}
\newcommand{\ww}{\mathbf{w}}
\newcommand{\yy}{\mathbf{y}}
\newcommand{\R}{\mathbb{R}}
\newcommand{\T}{\mathcal{T}}
\newcommand{\N}{\mathbb{N}}
\newcommand{\C}{\mathscr{C}}
\newcommand{\Z}{\mathbb{Z}}
\newcommand{\B}{\mathcal{B}}
\newcommand{\Q}{\mathbb{Q}}
\newcommand{\Sph}{\mathbb{S}}
\newcommand{\D}{\mathbb{D}}

\def\dim{\mathop{\rm dim}\nolimits}
\def\image{\mathop{\rm Im}\nolimits}  
\def\interior{\mathop{\rm Int}\nolimits}
\def\kernel{\mathop{\rm Ker}\nolimits}
\def\cokernel{\mathop{\rm Coker}\nolimits}
\def\bd{\mathop{\rm Bd}\nolimits}
\def\ext{\mathop{\rm Ext}\nolimits}
\def\num{\mathop{\#}\limits}
\def\cl{\mathop{\rm Cl}\nolimits}
\def\lub{\mathop{\rm lub}\nolimits}


\newcommand{\proj}{\operatorname{proj}}
\newcommand{\ds}{\displaystyle}
\newcommand{\pa}{\partial}
\newcommand{\ol}{\overline{}}


\newcommand{\degree}{^{\circ}}
\renewcommand{\epsilon}{\varepsilon}

\pagestyle{fancy}
\lhead{Math 6421: Measure Theory}
\chead{\bf Jose Nino: Homework \#1}
\rhead{\today}
\cfoot{Page \thepage}
% \setlength{\headheight}{10pt}

\theoremstyle{definition}
\newtheorem*{thm}{Theorem}
\newtheorem*{exercise}{Exercise}
\newtheorem*{definition}{Definition}
\newtheorem{problem}{Problem}
\newtheorem{lemma}{Lemma}
\newtheorem*{cor}{Corollary}


\begin{document}

% I will prove Problem 2.8 of the textbook and call it a ``lemma'' because I will need it to prove Problem 2.9.

% \begin{lemma}[Problem 2.8]
% \label{cluster-subseq}
%     Show that \( l \) is a cluster point of \( \{ x_n \} \) if and only there is a subsequence \( \{ x_{n_{j}} \}_{j=1}^{\infty} \) that converges to \( l \).

%         \begin{proof}
%             To complete this proof, we will need to show a forward and then a backward implication. 
%             \begin{itemize}
%                 \item[(\( \Rightarrow \))] Let \( \{ x_n\} \) be a sequence in \( \R \) with \( l \) as a cluster point. Let \( \epsilon > 0 \) be chosen. Because \( l \) is a cluster point, 
%                 \item[(\( \Leftarrow \))] Conversely, suppose the subsequence \( \{ x_{n_{j}} \}_{j=1}^{\infty} \) of \( \{ x_n\} \) converges to the point \( l \), and let \( \epsilon > 0 \) be chosen. This means there exists \( N' \in \N \) such that whenever \( j \geq N' \), we have that \( | x_{n_{j}} - l  | < \epsilon \). Now fix any \( N \in \N \). Now let us choose \( j \geq N' \) large enough so that \( n_{j} > N \), and so because \( \{ x_{n_{j}} \}_{j=1}^{\infty} \) is convergent, \( | x_n - l  | < \epsilon \). Thus \( l \) is a cluster point.
%             \end{itemize}
%         \end{proof}
    
% \end{lemma}

\begin{problem}[2.9]
    Properties of sequences in \( \R \).
    \begin{enumerate}[label = (\alph{*}.)] 
  
        \item Show that \( \lim \sup x_n \) and \( \lim \inf x_n \) are the largest and smallest cluster points of the sequence \( \{ x_n\} \).
            \begin{proof}
                Let \( \{ x_n \} \) be any sequence in \( \R \). First to show that \( l  = \lim \sup x_n \) is indeed a cluster point, let \( \epsilon > 0 \) be chosen. Because \( l \) is the limit superior, there exists \( n_1 \in \N \) such that \( x_{k_{1}} < l + \epsilon \) for all \( k_1 \geq n_1 \). Additionally, there are infinitely many value of this value \( n_1 \) such that \( x_k  > l - \epsilon \) for some \( k_1 \geq n_1 \), which together with the last sentence implies that \( | x_{k_{1}} - l | < \epsilon \). To inductively create a subsequence, let \( n_1, \ldots, n_j \) and \( x_{k_{1}}, \ldots, x_{k_{j}} \) be arbitrary. Let \( n_{j+1} \) be chosen such that \( n_{j+1} > \max \{ k_1, \ldots, k_j \} \). Then, because \( l \) is the limit superior \( x_{k} < l + \epsilon \) for any \( k \geq n_{j+1} \). Further, for sufficiently large \( n_{j+1} \), there exists \( k_{j+1} \geq n_{j+1} \) such that \( x_{k_{j+1}} > l - \epsilon \), which gives us that \( | x_{k_{j+1}} - l| < \epsilon \). Because we can always choose the next point in the subsequence in this manner, this means that the  subsequence \( \{ x_{n_{j} }\} \) converges to \( l \). By Problem 2.8, this means that \( l \) is a cluster point of \( \{ x_n \} \).
                
                By way of contradiction, suppose that \( l \) is not the largest cluster point of the sequence. That is, there exists a cluster point \( y \) of \( \{ x_n \} \) such that \( y > l \). Note that by Problem 2.8, this means that there exists a subsequence \( \{ x_{n_{j} }\} \) which converges to \( y \). Because \( l \) is the limit superior of the sequence, for any \( \epsilon > 0 \), we can find \( n \in \N \) such that \( x_k < l + \epsilon \) whenever \( k \geq n \). Since this is true for any \( \epsilon > 0 \), we can choose \( \epsilon > 0 \) small enough so that we have 
                    \[
                        l < l + \epsilon < y - \epsilon < y.  
                    \]
                This means that there are a finite number of terms of \( \{ x_n \} \) contained within the interval \( (y - \epsilon, y + \epsilon) \). In other words, there does not exist a subsequence \( \{ x_{n_{j} }\} \) which converges to \( y \) as we would necessarily need an infinite number of terms  within \( \epsilon \) of \( y \)---a contradiction. Therefore, \( l \) is the largest cluster point.

                By a reverse argument, we can show that \( \lim \inf x_n \) is a cluster point of \( \{ x_n\} \) as well as the smallest cluster point.
            \end{proof}
        \item Show that every bounded sequence has a convergent subsequence.
        
            \begin{proof}
                Let \( \{ x_n \} \) be a bounded sequence. In other words, \( \sup x_n \) is a  finite real number. By definition of the limit superior, \( \lim \sup x_n \leq \sup x_n \). From part (a), because \( lim \sup x_n \) is a cluster point, we  can always construct a convergent subsequence (as well as for the limit inferior of the sequence) and so this completes this part.
            \end{proof}
    \end{enumerate}

\end{problem}

\begin{problem}[2.43]

    Let \( f \) be defined as 
        \[
            f(x) =  
            \begin{cases}
                x & \text{if} \ x \ \text{is irrational} \\
                p \sin \left( \frac{1}{q} \right) & \text{if} \ x = \frac{p}{q} \ \text{in lowest terms.}    
            \end{cases}  
        \]

            At what points is \( f \) continuous? (Please justify your answer.)

            \begin{proof}

                I claim that \( f \) is not continuous at the rational numbers. To that end, let \( x \in Q \) and choose \( \epsilon = x - f(x) \). Fix \( \delta > 0 \). Note that we can always find an irrational number \( y \in (x, x + \delta) \). Because \( y \) is irrational, by definition of the function, \( f(y) - f(x) = y - f(x) \). But then \( y - f(x) > x - f(x) = \epsilon \).

                For \( x =  0  \), fix \( \epsilon > 0 \). Choose \( \delta = \epsilon  \), and a pick a point \( y \in \R \) such that \( |x - y| = |y - 0| = |y| < \delta \).
                Now because \(\sin(1/q) < 1/q  \) for any \( q \in \N \), we know that 
                    \begin{align*}
                        |f(y)  - f(0)| &\leq |y - 0| \\ 
                                    &< \delta \\
                                    &= \epsilon
                    \end{align*}
                so \( f \) is continuous at \( 0 \).

                \( f \)  is also continuous at the irrationals. This is because we can if we pick any point \( x \) in the irrationals, we can find sufficiently large \( q \) so a rational number \( y = \frac{p}{q} \) is close to \( x \) (i.e, for a fixed \( \epsilon \), choose \( \delta \) to be smaller than \( f(y) - y \) for this to work). Then this would allow us to bound \( |f(y) - f(x)| \) leveraging that we can put the rational numbers in lowest terms    
                
            \end{proof}
    
\end{problem}

\begin{problem}

    Show that \( F \subset \R \) is a closed set if and only if \( F^{\C} \) is open.

    \begin{proof}

        To complete this proof, we will need a forward and backwards implication. 
        \begin{itemize}
                \item[(\( \Rightarrow \))] Suppose \( F \subset \R \) is a closed set. Because we desire to show that \( F^{\C} \) is open, let \( x \in F^{\C} \) be a point. This means that \( x \not\in F \). Since \( F \) is a closed set (i.e., \( F = \overline{F} \) ) and \( x \not\in F \), we know \( x \) is not a point of closure of \( F \). So there exists \( \delta > 0 \) such that for all \( y \in F \), we do not have \( | x - y| < \delta \). But then if \( |x - y| < \delta \), this must mean that \( y \in F^{\C} \), and so \( F \) is an open set.
            
                \item[(\( \Leftarrow \))] Conversely, suppose that the set \( F^{\C} \) is open. Let \( x \in F^{\C} \). Then there exists \( \delta > 0 \) such that if \( | x - y | < \delta \), then \( y \in F^{C} \). This means that there is no \( y \in F \) such that \(| x-y | < \delta \) and so \( x \) cannot be a point of closure of \( F \). Thus, because \( x \) is arbitrary, \( F \) necessarily contains all its points of closure; in other words, \( F = \overline{F} \) and thus F must be closed, completing this direction.
            \end{itemize}
        Having completed both implications, this completes the proof.
    \end{proof}
    
\end{problem}


\end{document}