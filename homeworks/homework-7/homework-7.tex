\documentclass[12pt]{article}
\usepackage{amsmath,fullpage,graphicx,fancyhdr,enumerate,amsthm,amssymb,tikz,tikz-cd,color,pgfplots,tabularx,amsfonts,amscd}
\usepackage{enumitem}
\usetikzlibrary{arrows,chains,matrix,positioning,scopes}
\pgfplotsset{compat=1.13}
\setlength{\headheight}{15pt}
\setlength{\headsep}{7pt}
\newcommand{\bsnl}{\bigskip\newline}
\pagestyle{fancy}
\usepackage[paper=a4paper, margin = 1in]{geometry}
\usepackage{hyperref}% http://ctan.org/pkg/hyperref
\usepackage[capitalise, noabbrev]{cleveref}

\hypersetup{ % colors hyperlinks with colors to make noticeable but is not an ugly green box like default
    colorlinks,
    linkcolor={red!50!black},
    citecolor={blue!50!black},
    urlcolor={blue!80!black}
}
\usepackage[mathscr]{euscript}\usepackage{pifont}
\usepackage{cleveref}
\usepackage{parskip}
\usepackage[makeroom]{cancel}


\newcommand{\cmark}{\ding{51}}%
\newcommand{\xmark}{\ding{55}}%

\renewcommand{\aa}{\mathbf{a}}
\newcommand{\bb}{\mathbf{b}}
\newcommand{\cc}{\mathbf{c}}
\newcommand{\dd}{\mathbf{d}}
\newcommand{\FF}{\mathbf{F}}
\newcommand{\ii}{\mathbf{i}}
\newcommand{\jj}{\mathbf{j}}
\newcommand{\rr}{\mathbf{r}}
\newcommand{\kk}{\mathbf{k}}
\newcommand{\uu}{\mathbf{u}}
\newcommand{\vv}{\mathbf{v}}
\newcommand{\ww}{\mathbf{w}}
\newcommand{\yy}{\mathbf{y}}
\newcommand{\R}{\mathbb{R}}
\newcommand{\T}{\mathcal{T}}
\newcommand{\N}{\mathbb{N}}
\newcommand{\C}{\mathscr{C}}
\newcommand{\Z}{\mathbb{Z}}
\newcommand{\B}{\mathcal{B}}
\newcommand{\Q}{\mathbb{Q}}
\newcommand{\Sph}{\mathbb{S}}
\newcommand{\D}{\mathbb{D}}

\def\dim{\mathop{\rm dim}\nolimits}
\def\image{\mathop{\rm Im}\nolimits}  
\def\interior{\mathop{\rm Int}\nolimits}
\def\kernel{\mathop{\rm Ker}\nolimits}
\def\cokernel{\mathop{\rm Coker}\nolimits}
\def\bd{\mathop{\rm Bd}\nolimits}
\def\ext{\mathop{\rm Ext}\nolimits}
\def\num{\mathop{\#}\limits}
\def\cl{\mathop{\rm Cl}\nolimits}
\def\lub{\mathop{\rm lub}\nolimits}


\newcommand{\proj}{\operatorname{proj}}
\newcommand{\ds}{\displaystyle}
\newcommand{\pa}{\partial}
\newcommand{\ol}{\overline{}}


\newcommand{\degree}{^{\circ}}
\renewcommand{\epsilon}{\varepsilon}
\newcommand{\toM}{\overset{m}{\to}}


\newcommand{\upint}[2]{
  \overline{\int_{#1}^{#2}}
}
\newcommand{\lowint}[2]{
  \underline{\int_{#1}^{#2}}
}
\newcommand{\dif}{\, \mathrm{d}}
\newcommand{\abs}[1]{
    \left\lvert #1 \right\rvert
}
\newcommand{\norm}[1]{
    \left\lVert #1 \right\rVert
}

\pagestyle{fancy}
\lhead{Math 6421: Measure Theory}
\chead{\bf Jose Nino: Homework \#7}
\rhead{October 26, 2023}
\cfoot{Page \thepage}
% \setlength{\headheight}{10pt}

\theoremstyle{definition}
\newtheorem*{thm}{Theorem}
\newtheorem*{exercise}{Exercise}
\newtheorem*{definition}{Definition}
\newtheorem{problem}{Problem}
\newtheorem{lemma}{Lemma}
\newtheorem*{cor}{Corollary}
\newtheorem*{prop}{Proposition}


\begin{document}

\begin{problem}[6.11]

    Prove that \( L^{\infty} \) is complete. 

    \begin{proof}
        To show that \( L^{\infty} \) is complete, we must show every Cauchy sequence converges. 
        To that end, let \( \{f_n\} \) be any Cauchy sequence in \( L^{\infty} \) and let \( \epsilon> 0 \) be chosen. 
        Since \( \{f_n\} \) is Cauchy, there exists \( N_1 \in \N \) such that for all \( n, m \geq N_1 \), we have 
            \[
                \norm{f_n - f_m}_{\infty} = \inf \left\{  M : m \left\{ t : \abs{f_n(t) - f_m(t)} > M \right\} = 0 \right\} < \frac{\epsilon}{2}.  
            \]
        So for any fixed \( n,m \geq N_1 \), there exists \( \displaystyle M < \frac{\epsilon}{2} \) such that \( \displaystyle m\left\{ t: \abs{f_n(t) - f_m(t)} > M \right\} = 0 \). implying that  \( \displaystyle m\left\{ t: \abs{f_n(t) - f_m(t)} > \frac{\epsilon}{2} \right\} = 0 \). 
        So then on the set \( L^{\infty} \setminus \left\{ t: \abs{f_n(t) - f_m(t)} > \frac{\epsilon}{2} \right\}  \), the sequence \( \{f_n\} \) converges to some function \( f \) almost everywhere. 
        We must show that this limit function \( f \) is in \( L^{\infty} \)

        Since \( f_n \to f \) almost everywhere, there exists \( N_2 \in \N \) such that for any \( n > N_2 \), \( \displaystyle \abs{f_n - f} < \frac{\epsilon}{2} \). 
        Let \( N = \max\{ N_1, N_2 \} \). 
        Then for any fixed \( n > N \), we can see that 
            \[
                \norm{f_n - f}_{\infty} = \inf \left\{  M : m \left\{ t : \abs{f_n(t) - f(t)} > M \right\} = 0 \right\} < \epsilon 
            \]
        which, because \( \epsilon \) is arbitrary, means that \( \norm{f_n - f} \to 0 \). Thus \( f \in L^{\infty} \) and so \( L^{\infty} \) is complete. 
    \end{proof}
    
\end{problem}

\begin{problem}[6.13]

    Let \( C = C[0,1] \) be the space of all continuous functions on \( [0,1 ] \) and define \( \norm{f} = \max |f(x)| \).
    Show that \( C \) is a Banach space. 

        \begin{proof}
            To show that the space \( (C, \norm{\cdot})\) is a Banach space, we must show \( \norm{f} = \max |f(x)| \) is indeed a norm and that \( C \) with this norm is a complete space. 
            
            First, we will show that \( \norm{\cdot} \) defined above on \( C \) is a norm. 
            That is, we must show the following the following three properties:
                \begin{enumerate}[label = (\roman{*})]
                    \item For any \( f \in C \), \( \norm{f} = 0 \) if and only if \( f = 0 \).
                    \item For any \( f, g \in C \), \( \norm{f + g} \leq \norm{f} + \norm{g} \).
                    \item For any \( f \in C \) and for all \( \alpha \in \R \), \( \norm{\alpha f } = \abs{\alpha} \norm{f} \).
                \end{enumerate}
            For (i), first suppose that \( \norm{f} = 0 \). Then \( \max \abs{f(x) = 0 } \) which is true only if \( f(x) = 0 \) for any \( x \in [0,1] \). 
            Conversely, suppose \( f = 0 \). Then for any \( x \in [0,1] \), we have that \( \norm{f} = \max \abs{f(x)} = \max 0 = 0 \).

            To prove (ii), fix \( f, g \in C \). By the triangle inequality, we know that \( \abs{f + g} \leq \abs{f} + \abs{g} \).
            The \( \max \) function adheres to the triangle inequality and so 
                \begin{align*}
                    \norm{f + g} &= \max \abs{f + g} \\
                    &\leq \max \abs{f} + \max \abs{g} \\
                    &= \norm{f} + \norm{g}.
                \end{align*}
            
            Finally, let \( \alpha \in \R \) and \( f \in C \) be chosen. 
            Then 
                \begin{align*}
                    \norm{\alpha f} = \max \abs{\alpha f} &= \max \abs{\alpha} \abs{f} \\
                    &= \abs{\alpha} \max \abs{f} \\
                    &= \abs{\alpha} \norm{f}
                \end{align*}
            where the last equality follows since \( \alpha \) is a scalar and not dependent upon taking the \( \max \) over \( [0,1] \).

            To show \( C \) is complete, let \( \{f_n\} \) be a Cauchy sequence on \( C \).
            Let \( \epsilon > 0 \) be chosen.
            Since \( \{f_n\} \) is Cauchy, there exists \( N \in \N \) such that for all \( n,m \geq N \), we have that 
                \[
                    \norm{f_n - f_m} < \epsilon.  
                \]
            From how \( \norm{\cdot} \) is defined, this implies that \( |f_n(x) - f_m(x)| < \epsilon \) for any \( x \in [0,1] \). 
            But this means we can always find \( n > N \) large enough so that \( |f_n - f| < \epsilon \) i.e., \( \{f_n\} \) converges to a function \( f \) pointwise. 
            Because \( x \in [0,1] \), this means that that this convergence is uniform and so \( \norm{f_n  - f} < \epsilon \) i.e., \( \norm{f_n - f} \to 0 \). Thus, \( f \in C \) and so \( C \) is a complete space.

            Therefore, having shown that \( \norm{f} = \max \abs{f} \) is a norm and \( C \) is a complete space, \( C \) is a Banach space.
        \end{proof}
    
\end{problem}


\begin{problem}[5.1]

    Let \( f \) be the function defined 
        \[
            f(x) = \begin{cases}
                x \sin \left( \frac{1}{x}\right) & x \neq 0 \\
                0 & x = 0.
            \end{cases}  
        \]
    Find \( D^+f(0)\), \( D_+f(0)\), \( D^-f(0)\), and \( D_-f(0)\).

            \begin{proof}
                First, we will note that 
                    \begin{align*}
                        D^+(0) =  \frac{f(0 + h) - f(0)}{h} = \varlimsup_{h \to 0^+} \sin \left( \frac{1}{h}\right) = 1.
                    \end{align*}
                Similarly,
                    \[
                        D^-(0) =  \frac{f(0) - f(0 - h)}{h} = \varlimsup_{h \to 0^+} \sin \left( \frac{1}{h}\right) = 1.
                    \]
                However, this flips once we look at the limit inferior i.e.,
                    \begin{align*}
                        D_+(0) = \varliminf_{h \to 0^+} \sin \left( \frac{1}{h}\right) = -1
                    \end{align*}
                and
                    \[
                        D_-(0) =  \frac{f(0) - f(0 - h)}{h} = \varlimsup_{h \to 0^+} \sin \left( \frac{1}{h}\right) = -1..
                    \]
            \end{proof}


\end{problem}



\end{document}