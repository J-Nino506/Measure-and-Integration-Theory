\documentclass[12pt]{article}
\usepackage{amsmath,fullpage,graphicx,fancyhdr,enumerate,amsthm,amssymb,tikz,tikz-cd,color,pgfplots,tabularx,amsfonts,amscd}
\usepackage{enumitem}
\usetikzlibrary{arrows,chains,matrix,positioning,scopes}
\pgfplotsset{compat=1.13}
\setlength{\headheight}{15pt}
\setlength{\headsep}{7pt}
\newcommand{\bsnl}{\bigskip\newline}
\pagestyle{fancy}
\usepackage[paper=a4paper, margin = 1in]{geometry}
\usepackage{hyperref}% http://ctan.org/pkg/hyperref
\usepackage[capitalise, noabbrev]{cleveref}

\hypersetup{ % colors hyperlinks with colors to make noticeable but is not an ugly green box like default
    colorlinks,
    linkcolor={red!50!black},
    citecolor={blue!50!black},
    urlcolor={blue!80!black}
}
\usepackage[mathscr]{euscript}\usepackage{pifont}
\usepackage{cleveref}
\usepackage{parskip}
\usepackage[makeroom]{cancel}


\newcommand{\cmark}{\ding{51}}%
\newcommand{\xmark}{\ding{55}}%

\renewcommand{\aa}{\mathbf{a}}
\newcommand{\bb}{\mathbf{b}}
\newcommand{\cc}{\mathbf{c}}
\newcommand{\dd}{\mathbf{d}}
\newcommand{\FF}{\mathbf{F}}
\newcommand{\ii}{\mathbf{i}}
\newcommand{\jj}{\mathbf{j}}
\newcommand{\rr}{\mathbf{r}}
\newcommand{\kk}{\mathbf{k}}
\newcommand{\uu}{\mathbf{u}}
\newcommand{\vv}{\mathbf{v}}
\newcommand{\ww}{\mathbf{w}}
\newcommand{\yy}{\mathbf{y}}
\newcommand{\R}{\mathbb{R}}
\newcommand{\T}{\mathcal{T}}
\newcommand{\N}{\mathbb{N}}
\newcommand{\C}{\mathscr{C}}
\newcommand{\Z}{\mathbb{Z}}
\newcommand{\B}{\mathcal{B}}
\newcommand{\Q}{\mathbb{Q}}
\newcommand{\Sph}{\mathbb{S}}
\newcommand{\D}{\mathbb{D}}

\def\dim{\mathop{\rm dim}\nolimits}
\def\image{\mathop{\rm Im}\nolimits}  
\def\interior{\mathop{\rm Int}\nolimits}
\def\kernel{\mathop{\rm Ker}\nolimits}
\def\cokernel{\mathop{\rm Coker}\nolimits}
\def\bd{\mathop{\rm Bd}\nolimits}
\def\ext{\mathop{\rm Ext}\nolimits}
\def\num{\mathop{\#}\limits}
\def\cl{\mathop{\rm Cl}\nolimits}
\def\lub{\mathop{\rm lub}\nolimits}
\def\sgn{\mathop{\rm sgn}\nolimits}



\newcommand{\proj}{\operatorname{proj}}
\newcommand{\ds}{\displaystyle}
\newcommand{\pa}{\partial}
\newcommand{\ol}{\overline{}}


\newcommand{\degree}{^{\circ}}
\renewcommand{\epsilon}{\varepsilon}
\newcommand{\toM}{\overset{m}{\to}}


\newcommand{\upint}[2]{
  \overline{\int_{#1}^{#2}}
}
\newcommand{\lowint}[2]{
  \underline{\int_{#1}^{#2}}
}
\newcommand{\dif}{\, \mathrm{d}}
\newcommand{\norm}[1]{\left\lVert #1 \right\rVert}
\newcommand{\abs}[1]{\left\lvert #1 \right\rvert}


\pagestyle{fancy}
\lhead{Math 6421: Measure Theory}
\chead{\bf Jose Nino: Homework \#9}
\rhead{November 9, 2023}
\cfoot{Page \thepage}
% \setlength{\headheight}{10pt}

\theoremstyle{definition}
\newtheorem*{thm}{Theorem}
\newtheorem*{exercise}{Exercise}
\newtheorem*{definition}{Definition}
\newtheorem{problem}{Problem}
\newtheorem{lemma}{Lemma}
\newtheorem*{cor}{Corollary}
\newtheorem*{prop}{Proposition}


\begin{document}

\begin{problem}[6.21]

    \begin{enumerate}[label = (\alph{*})]
        \item Let \( g \) be an integrable function on \( [0,1] \). Show that there is a bounded measurable function \( f \) such that \( \norm{f} \neq 0 \) 
            \[
                \int fg = \norm{g}_{1} \cdot \norm{f}_{\infty}.  
            \]

                \begin{proof}
                    Let \( g \) be an integrable function on \( [0,1] \). This brings two cases: (i) \( \norm{g}_1 = 0 \) or (ii) \( \norm{g}_1 \neq 0 \). 
                    For case (i), if \( \norm{g}\), then \( g = 0 \) almost everywhere. Thus, let \( f = 1 \) which gives us that 
                        \[
                            \int 1 \cdot g = \int g = 0 = \norm{g}_1 \cdot \norm{f}_{\infty} . 
                        \]
                    Now suppose \( \norm{g}_1 \neq 0 \). Define \( f = \sgn{g} \). Then \( f \) is a bounded and measurable function, \( \norm{f}_{\infty} = 1 \), and thus
                        \[
                            \int fg = \int \abs{g} =  \norm{g}_{1} = \norm{g} \norm{f}_{\infty} .
                        \]
                    So having exhausted all cases, this completes the proof. 
                \end{proof}
        \item Let \( g \) be a bounded measurable function. Show that for each \( \epsilon > 0 \), there is an integrable function \( f \) such that 
            \[
                \int fg \geq \left( \norm{g}_{\infty} - \epsilon \right) \norm{f}_{1}. 
            \]
                \begin{proof}
                    Let \( g \) be a bounded measurable function, and let \( \epsilon > 0 \) be chosen. 
                    Define the set \( E = \left\{  x: g(x) > \norm{g}_{\infty} - \epsilon \right\}\) and the function \( f \) by \( f(x) = \chi_{E}(x) \). 
                    Then we have that
                        \begin{align*}
                            \int fg = \int_{E} g \geq \left(\norm{g}_{\infty} - \epsilon \right) m(E) = \left(\norm{g}_{\infty} - \epsilon \right) \cdot \norm{f}_{1}
                        \end{align*}
                    and which completes the proof. 
                \end{proof}
    \end{enumerate}


\end{problem}

\begin{problem}[11.3]

    \begin{enumerate}[label = (\alph{*})]
        \item Show that \( \mu(E_1 \bigtriangleup E_2) = 0 \) implies \( \mu(E_1) = \mu(E_2) \) provided that \( E_1, E_2 \in \B  \). 
            \begin{proof}
                Let \( E_1, E_2 \in \B \) and suppose that \( \mu(E_1 \bigtriangleup E_2) = 0 \). This means that
                    \[
                        \mu(E_1 \setminus E_2) = \mu(E_2 \setminus E_1) = 0.
                    \]
                From this, we can write \( E_1 \) and \( E_2 \) as disjoint unions and show that 
                    \begin{align*}
                        \mu(E_1) = \mu(E_1 \setminus E_2) + \mu(E_1 \cap E_2) =  \mu(E_2 \setminus E_1) + \mu(E_1 \cap E_2) =  \mu(E_2)
                    \end{align*}
                which shows the desired result.
            \end{proof}
        \item Not assigned.
    \end{enumerate}


\end{problem}

\end{document}