\documentclass[12pt]{article}
\usepackage{amsmath,fullpage,graphicx,fancyhdr,enumerate,amsthm,amssymb,tikz,tikz-cd,color,pgfplots,tabularx,amsfonts,amscd}
\usepackage{enumitem}
\usetikzlibrary{arrows,chains,matrix,positioning,scopes}
\pgfplotsset{compat=1.13}
\setlength{\headheight}{15pt}
\setlength{\headsep}{7pt}
\newcommand{\bsnl}{\bigskip\newline}
\pagestyle{fancy}
\usepackage[paper=a4paper, margin = 1in]{geometry}
\usepackage{hyperref}% http://ctan.org/pkg/hyperref
\usepackage[capitalise, noabbrev]{cleveref}

\hypersetup{ % colors hyperlinks with colors to make noticeable but is not an ugly green box like default
    colorlinks,
    linkcolor={red!50!black},
    citecolor={blue!50!black},
    urlcolor={blue!80!black}
}
\usepackage[mathscr]{euscript}\usepackage{pifont}
\usepackage{cleveref}
\usepackage{parskip}
\usepackage[makeroom]{cancel}


\newcommand{\cmark}{\ding{51}}%
\newcommand{\xmark}{\ding{55}}%

\renewcommand{\aa}{\mathbf{a}}
\newcommand{\bb}{\mathbf{b}}
\newcommand{\cc}{\mathbf{c}}
\newcommand{\dd}{\mathbf{d}}
\newcommand{\FF}{\mathbf{F}}
\newcommand{\ii}{\mathbf{i}}
\newcommand{\jj}{\mathbf{j}}
\newcommand{\rr}{\mathbf{r}}
\newcommand{\kk}{\mathbf{k}}
\newcommand{\uu}{\mathbf{u}}
\newcommand{\vv}{\mathbf{v}}
\newcommand{\ww}{\mathbf{w}}
\newcommand{\yy}{\mathbf{y}}
\newcommand{\R}{\mathbb{R}}
\newcommand{\T}{\mathcal{T}}
\newcommand{\N}{\mathbb{N}}
\newcommand{\C}{\mathscr{C}}
\newcommand{\Z}{\mathbb{Z}}
\newcommand{\B}{\mathcal{B}}
\newcommand{\Q}{\mathbb{Q}}
\newcommand{\Sph}{\mathbb{S}}
\newcommand{\D}{\mathbb{D}}

\def\dim{\mathop{\rm dim}\nolimits}
\def\image{\mathop{\rm Im}\nolimits}  
\def\interior{\mathop{\rm Int}\nolimits}
\def\kernel{\mathop{\rm Ker}\nolimits}
\def\cokernel{\mathop{\rm Coker}\nolimits}
\def\bd{\mathop{\rm Bd}\nolimits}
\def\ext{\mathop{\rm Ext}\nolimits}
\def\num{\mathop{\#}\limits}
\def\cl{\mathop{\rm Cl}\nolimits}
\def\lub{\mathop{\rm lub}\nolimits}
\def\ess{\mathop{\rm ess}\nolimits}


\newcommand{\proj}{\operatorname{proj}}
\newcommand{\ds}{\displaystyle}
\newcommand{\pa}{\partial}
\newcommand{\ol}{\overline{}}
\newcommand{\toM}{\overset{m}{\to}}

\newcommand{\degree}{^{\circ}}
\renewcommand{\epsilon}{\varepsilon}

\newcommand{\upint}[2]{
  \overline{\int_{#1}^{#2}}
}
\newcommand{\lowint}[2]{
  \underline{\int_{#1}^{#2}}
}


\pagestyle{fancy}
\lhead{Math 6421: Measure Theory}
\chead{\bf Jose Nino: Lecture \#15}
\rhead{October 19, 2023}
\cfoot{Page \thepage}
% \setlength{\headheight}{10pt}

\theoremstyle{definition}
\newtheorem*{thm}{Theorem}
\newtheorem*{exercise}{Exercise}
\newtheorem*{definition}{Definition}
\newtheorem{problem}{Problem}
\newtheorem*{lemma}{Lemma}
\newtheorem*{cor}{Corollary}
\newtheorem*{prop}{Proposition}


\begin{document}


\section*{Section 5.1 Differentiation of Monotone Functions}

Our goal will be to show if \( f \) is monotone increasing over an interval \( [a,b] \), then \( f \) is 
differentiable a.e.

\begin{definition}[\textbf{Vitali Cover}]

    Let \( E \subset \R \) and \( \Gamma \) is a collection of intervals. We call \( \Gamma \) a \textbf{Vitali cover}
    of \( E \) if for all \( x \in E \) and all \( \epsilon > 0 \), there \( I \in \Gamma \) such that \( x \in I \) and 
    \( 0 < |I| < \epsilon \).
    
\end{definition}

Here is an example. Let \( E = [0,1 ] \). Then
    \[
        \Gamma_1 = \left\{ \left( x - \frac{\epsilon}{2}, x + \frac{\epsilon}{2}\right) \right\}  
    \]
[there was more to the above... finish this after]

\begin{lemma}[\textbf{5.1}]

    Let \( E \subset \R \), \( m^{*}(E) < \infty \), and \( \Gamma \) is a Vitali cover of \( E \). Then for all \( \epsilon > 0 \)
    there a finite disjoint collection \( \{ I_1, \ldots, I_n\} \subset \Gamma \) such that 
        \[  
            m^{*} \left( E \setminus \bigcup_{i=1}^{N} I_i \right) < \epsilon.
        \]
    
\end{lemma}

\begin{definition}[\textbf{Dini Derivatives}]

    We define the \textbf{Dini Derivatives} as follows:
        \begin{align*}
            D^{+} f(x) &= \varliminf_{h \to 0+} \frac{f(x+h) - f(x)}{h} \\
            D^{-} f(x) &= \varliminf_{h \to 0+} \frac{f(x) - f(x - h)}{h} \\
            D_{+} f(x) &= \varlimsup_{h \to 0+} \frac{f(x+h) - f(x)}{h} \\
            D_{-} f(x) &= \varlimsup_{h \to 0+} \frac{f(x) - f(x - h)}{h}
        \end{align*}
We can remark by definition the following 
\begin{enumerate}
    \item \begin{align*}
        D^{+} f(x)  &\geq D_{+} f(x) \\
        D^{-} f(x) &\geq D_{-} f(x).
    \end{align*}
    \item         
    Also, \( f \) is differentiable if and only if \( D^{+}_{+} = D_{-}^{-} \).
\end{enumerate}

    
\end{definition}

\begin{lemma}
    Let \( f: [a,b] \to \R \) be an increasing function, and let \( \gamma < R \). Then 
        \[
            E = E_{\gamma, R} = \left\{ x \in (a,b): D_{-}f(x) < \gamma < R < D^{+}f(x)\right\}.
        \]
    has measure zero.

    \begin{proof}
        Write \( m^* E = s < \infty \). Let \( \epsilon > 0 \) be chosen. By definition of outer measure, there exists
        an open set \( E \subset O \) such that \( m(O) < s + \epsilon \). Let \( x \in E \). 
        Then \( \displaystyle D_{-}f(x) < \gamma \) which implies that for any \( \delta > 0 \), there exists \( h \in (0, \delta) \) with 
            \[  
                \frac{f(x) - f(x - h)}{h} < \gamma 
            \]
        coming from the definition of \( \liminf \). 
        Then the collection \( [x - h, x] \) forms a Vitali cover of \( E \). 
        By the definition of Vitalia covering Lemma (\textbf{Lemma 5.1}), for our fixed \( \epsilon > 0 \),
        there exists a finite collection \( \left\{ I_1 = [x_1 - h, x_1 ],  \ldots, I_N = [x_N - h, x_N] \right\} \) such that
            \[
                m^{*} \left( E \setminus \bigcup_{k=1}^{N} I_k \right) < \epsilon.  
            \]
        Let \( \displaystyle A = E \cap \bigcup_{k=1}^{N} I_k \). Then \( m^{*}(A) > s - \epsilon \) (recalling \( s = m^{*}(E) \).)
        This implies that 
            \begin{align*}
                0 \leq \sum_{k=1}^{n} f(x_k) - f(x_k - h_k) &< \gamma \sum_{k=1}^{N} h_k \\
                &= m^{*}\left( \bigcup_{k=1}^{N} I_k \right) \\
                &< \gamma \cdot (s + \epsilon) && \text{because} \ \bigcup_{k=1}^{N} I_k \subset O
            \end{align*}
        and so we are done with this part.

        Let \( y \in A \), and by \( \limsup \) we know that \( D^+f(x) > R \). Then there exists an arbitrarily small 
        \( k > 0 \) such that \( [y, y + k] \subset I_n \) for some \( n \in \N \) such that
            \[
                f(y + k )  - f(y) > R \cdot k. 
            \]
        Then the collection formed from \( [y, y+k] \) forms a Vitali cover on A. Again, by the Vitalia cover lemma (\textbf{Lemma 5.1}), there exists a collection \[J = \left\{  J_1 = [y_1, y_1 + k_1], \ldots, J_M = [y_M, y_M + k_M] \right\} \] such that 
            \[
                m^{*} \left( A \setminus \bigcup_{j}^{M} J_{j} \right) < \epsilon.
            \]
        This implies that 
            \[
                m^{*}\left( A \cap  \setminus \bigcup_{j}^{M} J_{j}\right) > s - 2 \epsilon.
            \]
        So we can sum across these intervals and get that
            \begin{align*}
                \sum_{j=1}^{M} f(y_j + k_j) - f(y_j) &> R \sum_{j=1}^{M} k_j \\
                &> R(s - 2\epsilon).
            \end{align*}
        Putting this all together and noting that \( f \) is an increasing function
            \begin{align*}
                \sum_{n=1}^{N} f(x_n) - f(x_n - h_n) &\geq \sum_{j=1}^{M}f(y_j + k_j) - f(y_j) \\
                &> R(s - 2\epsilon)
            \end{align*}
        and so we get that
            \[
                \gamma(s + \epsilon) < R(s - 2 \epsilon)  .
            \]
        Thus \( \gamma s \geq Rs \) and implies that \( s = 0 \) (noting that \( R > \gamma \)). Now we are done!
    \end{proof}
\end{lemma}

\begin{thm}[\textbf{5.3}]

    Let \( f: [a,b] \to \R \) be monotonic increasing. Then \( f \) is differentiable almost everywhere, \( f' \)
    is measurable, and 
        \[
            \int_{a}^{b} f'(x) \mathrm{d}\,x \leq f(b) - f(a).  
        \]

        \begin{proof}
            Let \( E = \{ x: D_{-}f(x) < D^{+} f(x)\} \). Then 
                \[
                    E = \bigcup_{r, R \in \Q} E_{r, R}  
                \]
            has a measure zero by our lemma above. This tells us that \( f \) is differentiable almost everywhere because
            we showed that the sets where any two derivates are not equal have measure zero. Thus
                \[
                    f'(x) = \lim_{h \to 0} \frac{f(x+h) - f(x)}{h}  
                \] 
            exists almost everywhere. Define the function
                \[
                    f(x) = \begin{cases}
                        f(b) & x \geq b \\
                        f(x) & x \in [a,b]
                    \end{cases}  
                \]
            which extends the function \( f \) to the right. Define the sequence \( \{ G_n\} \) by 
                    \[
                        G_n(x) = \frac{f\left( x + \frac{1}{n} \right) - f(x)}{\frac{1}{n}} \geq 0.
                    \]
            This sequence converges a.e. to \( f'(x) \) as \( n \to \infty \). So by Fatou's lemma,
                    \begin{align*}
                        \int_{a}^{b} f'(x) \, \mathrm{d}x & \leq \varliminf_{n \to \infty} \int_{a}^{b} G_n(x) \, \mathrm{d} x \\
                        &= \varliminf_{n \to \infty} \int_{a}^{b} n \left( f \left( x + \frac{1}{n}\right) - f(x) \right) \mathrm{d} x \\
                        &= \int_{a + 1/n}^{b + 1/n} f(y) \, \mathrm{d} y \\
                        &= \varliminf_{n \to \infty} n \int_{a + 1/n}^{b + 1/n} f(y) \, \mathrm{d} x - \int_{a}^{a + 1/n} f(x) \, \mathrm{d}x \\
                        &= \varliminf_{n \to \infty} n \left(\int_{b}^{b + 1/n} f(y) \, \mathrm{d} x - \int_{a}^{a + 1/n} f(x) \, \mathrm{d}x \right) \\
                        &= \varliminf_{n \to \infty} n \cdot f(b) \cdot \frac{1}{n} - \varliminf_{n \to \infty} \int_{a}^{a + 1/n} f(x) \, \mathrm{d}x \\ 
                        &\leq f(b) -  \varliminf_{n \to \infty} \int_{a}^{a + 1/n} f(x) \, \mathrm{d}x \\
                        &\leq f(b) - f(a) && \text{noting that} \, f(x) \geq \frac{1}{n}
                    \end{align*}
                
        \end{proof}
The next thing that will be covered is functions of bounded variation in Section 5.2.

\begin{definition}
    Let \( f: [a,b ]\to \R  \). Then \( f \) is a function of \textbf{bounded variation} if \( \lVert f \rVert_{TV[a,b]} < \infty \) where \( TV[a,b] \) is the total variation of \( [a,b] \) and
        \[
            \lVert f \rVert_{TV[a,b]} = \sup \left\{ \sum_{i=0}^{n} |f(x_{i+1}) - f(x)| : P \ \text{ is a partition of}\ [a,b] \right\}.
        \]
\end{definition}

\end{thm}


\end{document}