\documentclass[12pt]{article}
\usepackage{amsmath,fullpage,graphicx,fancyhdr,enumerate,amsthm,amssymb,tikz,tikz-cd,color,pgfplots,tabularx,amsfonts,amscd}
\usepackage{enumitem}
\usetikzlibrary{arrows,chains,matrix,positioning,scopes}
\pgfplotsset{compat=1.13}
\setlength{\headheight}{15pt}
\setlength{\headsep}{7pt}
\newcommand{\bsnl}{\bigskip\newline}
\pagestyle{fancy}
\usepackage[paper=a4paper, margin = 1in]{geometry}
\usepackage{hyperref}% http://ctan.org/pkg/hyperref
\usepackage[capitalise, noabbrev]{cleveref}

\hypersetup{ % colors hyperlinks with colors to make noticeable but is not an ugly green box like default
    colorlinks,
    linkcolor={red!50!black},
    citecolor={blue!50!black},
    urlcolor={blue!80!black}
}
\usepackage[mathscr]{euscript}\usepackage{pifont}
\usepackage{cleveref}
\usepackage{parskip}
\usepackage[makeroom]{cancel}


\newcommand{\cmark}{\ding{51}}%
\newcommand{\xmark}{\ding{55}}%

\renewcommand{\aa}{\mathbf{a}}
\newcommand{\bb}{\mathbf{b}}
\newcommand{\cc}{\mathbf{c}}
\newcommand{\dd}{\mathbf{d}}
\newcommand{\FF}{\mathbf{F}}
\newcommand{\ii}{\mathbf{i}}
\newcommand{\jj}{\mathbf{j}}
\newcommand{\rr}{\mathbf{r}}
\newcommand{\kk}{\mathbf{k}}
\newcommand{\uu}{\mathbf{u}}
\newcommand{\vv}{\mathbf{v}}
\newcommand{\ww}{\mathbf{w}}
\newcommand{\yy}{\mathbf{y}}
\newcommand{\R}{\mathbb{R}}
\newcommand{\T}{\mathcal{T}}
\newcommand{\N}{\mathbb{N}}
\newcommand{\C}{\mathscr{C}}
\newcommand{\Z}{\mathbb{Z}}
\newcommand{\B}{\mathcal{B}}
\newcommand{\Q}{\mathbb{Q}}
\newcommand{\Sph}{\mathbb{S}}
\newcommand{\D}{\mathbb{D}}

\def\dim{\mathop{\rm dim}\nolimits}
\def\image{\mathop{\rm Im}\nolimits}  
\def\interior{\mathop{\rm Int}\nolimits}
\def\kernel{\mathop{\rm Ker}\nolimits}
\def\cokernel{\mathop{\rm Coker}\nolimits}
\def\bd{\mathop{\rm Bd}\nolimits}
\def\ext{\mathop{\rm Ext}\nolimits}
\def\num{\mathop{\#}\limits}
\def\cl{\mathop{\rm Cl}\nolimits}
\def\lub{\mathop{\rm lub}\nolimits}
\def\ess{\mathop{\rm ess}\nolimits}


\newcommand{\proj}{\operatorname{proj}}
\newcommand{\ds}{\displaystyle}
\newcommand{\pa}{\partial}
\newcommand{\ol}{\overline{}}
\newcommand{\toM}{\overset{m}{\to}}

\newcommand{\degree}{^{\circ}}
\renewcommand{\epsilon}{\varepsilon}

\newcommand{\upint}[2]{
  \overline{\int_{#1}^{#2}}
}
\newcommand{\lowint}[2]{
  \underline{\int_{#1}^{#2}}
}


\pagestyle{fancy}
\lhead{Math 6421: Measure Theory}
\chead{\bf Jose Nino: Lecture \#14}
\rhead{October 17, 2023}
\cfoot{Page \thepage}
% \setlength{\headheight}{10pt}

\theoremstyle{definition}
\newtheorem*{thm}{Theorem}
\newtheorem*{exercise}{Exercise}
\newtheorem*{definition}{Definition}
\newtheorem{problem}{Problem}
\newtheorem*{lemma}{Lemma}
\newtheorem*{cor}{Corollary}
\newtheorem*{prop}{Proposition}


\begin{document}


\begin{thm}[\textbf{6.6, Riesz-Fisher}]

    \( L^p \) is complete for \( p \in [1, \infty] \).

    \begin{proof}
        Note \( p = \infty \) is an exercise. We want to show every absolute summable series is summable.
        Let \( \{f_n\}  \subset L^p \) such that 
            \[
                \sum_{n=1}^{\infty} \lVert f_n \rVert < M.  
            \]
        Consider the function \( \displaystyle g_n(x) = \sum_{k=1}^{n} f_k(x) \) and want to show that \(g_n \) converges some function \( g \) (i.e., the limit exists.) By the triangle inequality,
            \begin{align*}
                \lVert g_n(x) \rVert \leq \sum_{k=1}^{n} \lVert f_k(x) \rVert < M
            \end{align*}
        and so we know that \( \displaystyle \int |g_n|^p < M^p \). We know we want to do the following:
        \begin{enumerate}
            \item Find a limit of \( g_n \).
            \item Then show it is in \( L^p \).
        \end{enumerate}
        For each fixed \( x \in L^P \), \( \{g_n(x) \} \) is monotonically increasing. By the Monotone Convergence Theorem, there exists \( g(x) \in \overline{\R} \) (extended real numbers) and \( g_n(x) \to g(x) \). Because \( g_n \geq 0 \) is measurable, \( g(x) \geq 0 \) is measurable as well an dso 
            \[
                \int g^p \leq M^p
            \]
        and so this implies that \( g \in L^p \). Notice that for each \( x \) such that \( g(x) \) is finite, \( \displaystyle \sum_{k=1}^{\infty} f_k(x) \) is absolutely summable over \( \R \). Since \( (\R, |\cdot|) \) is complete, we know that \( \displaystyle S_n(x) = \sum_{k=1}^{n} f_k(x) \) is summable i.e.,
            \[
                S_n(x) \to S(x) = \sum_{k=1}^{f_k(x)}  
            \]
        over \( \R \).  We can look at just the limit \( S(x) \) because we are looking at only where \( g(x)\) has finite measure and defined as
            \[
                \widetilde{S} =  \begin{cases}
                    S_n(x) & g(x) < \infty \\
                    0 & \text{otherwise}.
                \end{cases}  
            \]
        In other words, \( S(x) \) is measurable since \( \displaystyle S(x) \neq \sum_{k=1}^{\infty} f_k(x) \) only on a measure zero set. Moreover, \( |S_n(x)| \leq |g(x)| \) for all \( n \in \N \) and so 
        \( |\widetilde{S}(x)| \leq |g(x)| \) and therefore \( \widetilde{S}(x) \in L^p \). We claim that
                \[
                    \left\lVert \sum_{k=1}^{n} f_n(x) - \widetilde{S}_n(x) \right\rVert_{p} \to 0 \ \text{as}\ n \to \infty. 
                \]
            We know that because \( S \in L^p \), 
                \[
                    \left| S_n(x) - \widetilde{S}_n(x) \right|^p \leq 2^p \cdot |g(x)|^p.
                \]
            So by the Lebesgue Dominated Convergence, we know that \( \lVert s_n  - \widetilde{s}_n \rVert^p \to 0 \) and so \( \lVert s_n - \widetilde{s}_n \rVert \to 0 \). Thus the sum \( s\) from \( \{f_n\} \) is in \( L^p \).
     \end{proof}
\end{thm}
    
\section*{Section 6.4 Approximation in \(L^p\)}

Our goal with approximation properties is to approximate functions in \( f \in  L^p \) spaces by step functions \( \phi \) and continuous functions \( g \). That is, for any \( \epsilon > 0 \), there exists functions \( \phi \) and \( g \) such that
     \[
        \lVert f - g\rVert_{p} < \epsilon \quad \text{and} \quad \lVert f - \phi \rVert_p < \epsilon.
     \]

\begin{lemma}[\textbf{6.7}]

    Given \( f \in L^p \), \( p \in [1, \infty] \), and any \( \epsilon > 0 \), there is a bounded measurable functions \( f_M \) with \( |f_m| \leq M \) and \( \lVert f - f_M \rVert < \epsilon \).

        \begin{proof}
            Consider the function 
                \[
                    f_N(x) = \begin{cases}
                        N & N \leq f(x) \\
                        f(x) & -N \leq f(x) \leq N \\
                        -N & f(x) \leq - N.
                    \end{cases}  
                \]
        \end{proof}
    Note that \( |f_n| \leq N \) and so is measurable for each \( n \in \N \). Then \( f_N(x) \to f(x) \) converges a.e (there may be unbounded points but they are of measure zero) which implies
    that 
    \( | f_N(x) - f(x)|^{p} \to 0 \) a.e. Thus,
        \[
            |f_N(X) - f(x)|^p < 2|f(x)|^p    
        \]
    almost everywhere. So by the Lebesgue Dominated Convergence Theorem,
        \[
            \int |f_N(x) - f(x) |^p \to 0   
        \]
    or, equivalently, \( \displaystyle \lVert f_N - f(x) \rVert_p \to 0 \) and so \( f_n(x) \to f(x) \) is in \( L^p \).
    
\end{lemma}

\begin{prop}[\textbf{6.8}]

    Given \( f \in L^p \), \( p \in [1, \infty) \) and any \( \epsilon > 0 \), there is a step function \( \phi \) and a continuous function \( g\) such that 
        \[
            \lVert f - g\rVert_{p} < \epsilon \quad \text{and} \quad \lVert f - \phi \rVert_p < \epsilon.
        \]  
        \begin{proof}
            Let \( \epsilon > 0 \) be chosen. For step functions, we note that by Lemma 6.7, there exists 
            an \( f_M \) such that \( \displaystyle \lVert f - f_M \rVert < \frac{\epsilon}{2} \). By Theorem 3.22 (Littlewood's 2nd Principle), there exists a step function \( \phi \) such that \( |f_M - \phi| < \frac{\epsilon}{4} \) except on a set \( E \) of measure less than 
                \[
                    \delta = \left( \frac{\epsilon}{8M} \right)^p.  
                \]
            Then
                \begin{align*}
                    \lVert f_M - \phi \rVert_{p} &= \int |f_M - \phi|^p \\
                    &= \int_{[0,1] \setminus E} \int |f_M - \phi|^p + \int_{E} |f_M - \phi|^p \\
                    &\leq \int_{[0,1]} \left( \frac{\epsilon}{4}  \right)^{p} + \int_{E} |f_M - \phi|^p  \\
                    &=\int_{[0,1]} \left( \frac{\epsilon}{4}  \right)^{p} = (2M)^p \cdot m(E) \\
                    &\leq \int_{[0,1]} \left( \frac{\epsilon}{4} \right) + (2M)^p \cdot \left( \frac{\epsilon}{8M}\right)
                \end{align*}
            and taking the \( p \)th root, we know that
                \[
                    \lVert f_M - \phi \rVert_p \leq \frac{\epsilon}{2}.  
                \]
            Doing the same thing with a continuous functions \( f\) than step function, we can show that
                \[
                    \lVert f_M - g \rVert < \frac{\epsilon}{2}.  
                \]
        \end{proof}
\end{prop}

This means that ``step functions'' and ``continuous functions'' are ``dense'' in \( L^p \) (i.e., we can always use step and continuous functions in the limit to approximate functions \( f \in L^p\)).



\end{document}