\documentclass[12pt]{article}
\usepackage{amsmath,fullpage,graphicx,fancyhdr,enumerate,amsthm,amssymb,tikz,tikz-cd,color,pgfplots,tabularx,amsfonts,amscd}
\usepackage{enumitem}
\usetikzlibrary{arrows,chains,matrix,positioning,scopes}
\pgfplotsset{compat=1.13}
\setlength{\headheight}{15pt}
\setlength{\headsep}{7pt}
\newcommand{\bsnl}{\bigskip\newline}
\pagestyle{fancy}
\usepackage[paper=a4paper, margin = 1in]{geometry}
\usepackage{hyperref}% http://ctan.org/pkg/hyperref
\usepackage[capitalise, noabbrev]{cleveref}

\hypersetup{ % colors hyperlinks with colors to make noticeable but is not an ugly green box like default
    colorlinks,
    linkcolor={red!50!black},
    citecolor={blue!50!black},
    urlcolor={blue!80!black}
}
\usepackage[mathscr]{euscript}\usepackage{pifont}
\usepackage{cleveref}
\usepackage{parskip}
\usepackage[makeroom]{cancel}


\newcommand{\cmark}{\ding{51}}%
\newcommand{\xmark}{\ding{55}}%

\renewcommand{\aa}{\mathbf{a}}
\newcommand{\bb}{\mathbf{b}}
\newcommand{\cc}{\mathbf{c}}
\newcommand{\dd}{\mathbf{d}}
\newcommand{\FF}{\mathbf{F}}
\newcommand{\ii}{\mathbf{i}}
\newcommand{\jj}{\mathbf{j}}
\newcommand{\rr}{\mathbf{r}}
\newcommand{\kk}{\mathbf{k}}
\newcommand{\uu}{\mathbf{u}}
\newcommand{\vv}{\mathbf{v}}
\newcommand{\ww}{\mathbf{w}}
\newcommand{\yy}{\mathbf{y}}
\newcommand{\R}{\mathbb{R}}
\newcommand{\T}{\mathcal{T}}
\newcommand{\N}{\mathbb{N}}
\newcommand{\C}{\mathscr{C}}
\newcommand{\Z}{\mathbb{Z}}
\newcommand{\B}{\mathcal{B}}
\newcommand{\Q}{\mathbb{Q}}
\newcommand{\Sph}{\mathbb{S}}
\newcommand{\D}{\mathbb{D}}

\def\dim{\mathop{\rm dim}\nolimits}
\def\image{\mathop{\rm Im}\nolimits}  
\def\interior{\mathop{\rm Int}\nolimits}
\def\kernel{\mathop{\rm Ker}\nolimits}
\def\cokernel{\mathop{\rm Coker}\nolimits}
\def\bd{\mathop{\rm Bd}\nolimits}
\def\ext{\mathop{\rm Ext}\nolimits}
\def\num{\mathop{\#}\limits}
\def\cl{\mathop{\rm Cl}\nolimits}
\def\lub{\mathop{\rm lub}\nolimits}


\newcommand{\proj}{\operatorname{proj}}
\newcommand{\ds}{\displaystyle}
\newcommand{\pa}{\partial}
\newcommand{\ol}{\overline{}}


\newcommand{\degree}{^{\circ}}
\renewcommand{\epsilon}{\varepsilon}
\newcommand{\dif}{\; \mathrm{d}}

\newcommand{\upint}[2]{
  \overline{\int_{#1}^{#2}}
}
\newcommand{\lowint}[2]{
  \underline{\int_{#1}^{#2}}
}

\pagestyle{fancy}
\lhead{Math 6421: Measure Theory}
\chead{\bf Jose Nino: Lecture \#8}
\rhead{September 19, 2023}
\cfoot{Page \thepage}
% \setlength{\headheight}{10pt}

\theoremstyle{definition}
\newtheorem*{thm}{Theorem}
\newtheorem*{exercise}{Exercise}
\newtheorem*{definition}{Definition}
\newtheorem{problem}{Problem}
\newtheorem*{lemma}{Lemma}
\newtheorem*{cor}{Corollary}
\newtheorem*{prop}{Proposition}


\begin{document}

\begin{prop}[\textbf{4.3}]
  Let \( E \in \mathfrak{M} \), and let \( f: E \to \R \) with \( m(E) < \infty \) be a bounded, measurable function. Let \( \phi, \psi \) be simple functions. Then
      \[
          \inf \left\{ \int \psi: \psi \geq f \right\} = \sup \left\{ \int \phi : \phi \leq f \right\}.  
      \] 
  if and only if \( f \) is measurable. 

\begin{proof}\footnote{Proof is on pages 79-80.}
  We will need to show two implications.
  \begin{enumerate}
    \item[(\(\Leftarrow\))] First, suppose that \( f \) is measurable. 
    Fix any \( n \in \N \) and define the set 
      \[ 
         E_k = \left\{x: \frac{k-1}{n} M < f(x) \leq \frac{kM}{n} \right\}
      \]
    with \( k \in [-n, n] \) and \( |f(x)| < M \). Note that because \( f \) is measurable, each \( E_k \) is a measurable set and also we have that \( \displaystyle \bigcup_{k = -n}^{n} E_K = E \). 
    Define the upper and lower sequence of simple functions, \( \{\psi_n\} \) and \( \{\phi_n\} \), respectively, as 
      \[
          \psi_n (x) = \sum_{k=-n}^{n} \frac{M}{n} \cdot k \cdot \chi_{E_{k}}(x) 
          \quad \text{and} \quad
          \phi_n = \sum_{k=-n}^{n} \frac{M}{n} \cdot (k-1) \cdot \chi_{E_{k}}(x).
      \]
    So for any \( x \in E \), \( \phi(x) \leq f(x) \leq \psi(x) \). Thus,
      \[
        \inf_{\psi \geq f} \int_{E} \psi \leq \int_{E} \psi = \frac{M}{n}  \sum_{k=-n}^{n} k \cdot m(E_k) 
      \]
    and 
      \[
          \sup_{\phi \leq f} \int_{E} \phi \geq \int_{E} \phi = \frac{M}{n} \sum_{k=-n}^{n} (k-1) \cdot m(E_k).
      \]
    Putting these two inequalities together, we can show that 
      \begin{align*}
        \inf_{\psi \geq f} \int_{E} \psi  - \sup_{\phi \leq f} \int_{E} \phi &\leq \int_{E} \psi - \phi  \\
        &= \sum_{k=-n}^{n} \left( \psi_n - \phi_n \right) m(E_k) \\
        &= \frac{M}{n} m(E).
      \end{align*}
    Since \( n \in \N \) is fixed, this quantity is zero. Thus
      \[
        \inf_{\psi \geq f} \int_{E} \psi  - \sup_{\phi \leq f} \int_{E} \phi = 0 \Rightarrow \inf_{\psi \geq f} \int_{E} \psi  = \sup_{\phi \leq f} \int_{E} \phi
      \]
    completing this direction. 
    \item[(\( \Rightarrow \))] Conversely, suppose that 
        \[
          \inf_{\psi \geq f} \int_{E} \psi  = \sup_{\phi \leq f} \int_{E} \phi.
        \]
      By definition of the infimum and supremum, there exists sequences of simple functions \( \{ \phi_n\} \) and \( \{ \psi_n\} \) 
      such that \( \phi_n \leq f \leq \psi_n \) for all \( n \in \N \) with 
        \[
            \int_{E} \phi_n - \psi_n < \frac{1}{n}.
        \]
      Define \( \displaystyle \phi^{*} = \sup_{n} \phi_{n} \) and \( \displaystyle \psi^{*} = \inf_{n} \psi(n) \). 
      Since simple functions are measurable functions, by Proposition 3.20, \( \phi^{*} \) and \( \psi^{*} \) are measurable as well and \( \phi_n \leq f \leq \psi_n \).

      We claim that \( f = \phi^{*} \) a.e. Consider the set
        \[
          \Delta = \{ x: \phi^{*}(x) < \psi^{*}(x) \}.
        \]
      Let \( \nu \in \N \) and let 
        \[
            \Delta_{\nu} = \left\{ x: \phi^{*}(x) < \psi^{*}(x) - \frac{1}{\nu} \right\}
        \]
      which means that 
        \[
            \Delta = \bigcup_{\nu \in \N} \Delta_{\nu}.
        \]
      For any \( n \in \N \),
        \[
          \Delta_{\nu} \subset \left\{ x: \phi(x) < \psi(x) - \frac{1}{\nu} \right\}.
        \]
      Thus, we have that, for any \( n \in \N \),
        \begin{align*}
          m \left( \Delta_{\nu} \right) &= \int \chi_{\Delta_{\nu}} = \nu \int \frac{1}{\nu} \cdot \chi_{\Delta_{\nu}} \\
          &\leq \nu \int_{\Delta_{\nu}} \left( \psi_n - \phi_n \right) \\
          &< \mu \int_{E} \frac{1}{n} \\
          &= \frac{\nu}{n} m(E).
        \end{align*}
      Because \( \nu \) is fixed and \( n \) is arbitrary, \( \displaystyle m(\Delta_{\nu}) = 0 \) which implies that \( \displaystyle m(\Delta) = 0 \). So then \( \phi^{*} = \psi^{*} \) except on a set of measure zero, and \( \phi^{*} = f \) except on a set of measure zero i.e., \( f = \phi^{*} \) a.e. implying that \( f \) is measurable by Proposition 3.21.
  \end{enumerate}
  Having completed both directions, this completes the proof and shows the well-definedness of Lebesgue integration. 
\end{proof}
\end{prop}

\begin{prop}[\textbf{4.4}] Let \( f \) be a bounded function defined on \( [a,b] \). If \( f \) is Riemann integrable, then it is measureble and 
    \[
        R \int_{a}^{b} f(x) \dif x = \int_{a}^{b} f(x) \dif x.
    \]

    \begin{proof}
      This proof is on page 82 of Royden (very simple proof, in fact).
    \end{proof}
  
\end{prop}

\begin{prop}[\textbf{4.5}]

  If \( f \) and \( g \) are bounded measurable functions defined on a set \( E \) of finite measure, then:
    \begin{enumerate}[label = \roman{*}.]
      \item For any \( a,b \in \R \), \[ \int_{E} (af + bg) = a \int_{E} f + b \int_{E} g. \]
      \item If \( f = g \) a.e., then 
        \[
            \int_{E} f = \int_{E} g.
        \]
      \item If \( f \leq g \) almost everywhere. then 
        \[
            \int_{E} f \leq \int_{E} g. 
        \]
      Hence 
        \[
            \left| \int_{E} f \right| \leq \int_{E} \left| f \right|.
        \]
      \item If \( A \leq f(x) \leq B \), then 
        \[
            Am(E) \leq \int_{E} f \leq Bm(E).
        \]
      \item If \( A \) and \( B \) are disjoint measurable sets of finite measure
    \end{enumerate}
  
\end{prop}

\begin{prop}[\textbf{4.6, Bounded Convergence Theorem}]

  Let \( \{ f_n \} \) be a sequence of measurable functions defined over a measurable set \( E \) of finite measure. Suppose there is \( M \in \R  \) such that \( |f_n(x) | \leq M \) for all \( x \in E \) and for all \( n \in \N \). If \( f_n(x) \to f(x) \) pointwise (i.e., \( \displaystyle \lim_{n \to \infty} f_n = f(x) \)), then 
    \[
        \int_{E} f_n \to \int_{E} f \Leftrightarrow \lim_{n \to \infty} \int_{E} f_n(x) = \int_{E} f.
    \]
  
\end{prop}

  \begin{proof}
    Let \( \epsilon > 0 \) be chosen. By Proposition 3.23 (Weak Ergoroff's Theorem), there exists \( N \in \N \) and \( A \subset E \) with 
      \[
        m(A) = \tilde{\delta} < \frac{\epsilon}{4M}
      \]
    such that that for all \( x \in E \setminus A \) and for all \( n > N \),
      \[
          \left| f_n(x) - f(x) \right| < \frac{\epsilon}{2 \cdot m(E)} = \tilde{\epsilon}.
      \]
    Thus, we can show the following:
    \begin{align*}
        \left| \int_{E} f_n - \int_{E} f \right| = \left| \int_{E} f_n - f \right| 
        &\leq \int_{E} \left| f_n - f \right| \\
        &= \int_{A} |f_n - f| + \int_{E \setminus A} |f_n - f| \\
        &\leq 2M \cdot m(A) + \int_{E \setminus A} |f_n - f| \\
        &\leq 2M \cdot m(A) + \frac{\epsilon \cdot  m(E \setminus A)}{2 \cdot m(E)} \\
        &< 2M \cdot \frac{\epsilon}{4 M} + \frac{\epsilon \cdot m(E \setminus A)}{2 \cdot m(E)} \\ 
        &\leq \frac{\epsilon}{2} + \frac{\epsilon}{2} \\
        &= \epsilon
    \end{align*}
  and so we are done!!
  \end{proof}

\begin{prop}[\textbf{4.7}]
  Let \( f \) be bounded on \( [a, b] \). Then \( f \) is Riemann integrable if and only the set of discontinuities has measure zero. 
  
\end{prop}

\end{document}