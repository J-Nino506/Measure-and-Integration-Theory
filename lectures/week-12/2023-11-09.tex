\documentclass[12pt]{article}
\usepackage{amsmath,fullpage,graphicx,fancyhdr,enumerate,amsthm,amssymb,tikz,tikz-cd,color,pgfplots,tabularx,amsfonts,amscd}
\usepackage{enumitem}
\usetikzlibrary{arrows,chains,matrix,positioning,scopes}
\pgfplotsset{compat=1.13}
\setlength{\headheight}{15pt}
\setlength{\headsep}{7pt}
\newcommand{\bsnl}{\bigskip\newline}
\pagestyle{fancy}
\usepackage[paper=a4paper, margin = 1in]{geometry}
\usepackage{hyperref}% http://ctan.org/pkg/hyperref
\usepackage[capitalise, noabbrev]{cleveref}

\hypersetup{ % colors hyperlinks with colors to make noticeable but is not an ugly green box like default
    colorlinks,
    linkcolor={red!50!black},
    citecolor={blue!50!black},
    urlcolor={blue!80!black}
}
\usepackage[mathscr]{euscript}\usepackage{pifont}
\usepackage{cleveref}
\usepackage{parskip}
\usepackage[makeroom]{cancel}


\newcommand{\cmark}{\ding{51}}%
\newcommand{\xmark}{\ding{55}}%

\renewcommand{\aa}{\mathbf{a}}
\newcommand{\bb}{\mathbf{b}}
\newcommand{\cc}{\mathbf{c}}
\newcommand{\dd}{\mathbf{d}}
\newcommand{\FF}{\mathbf{F}}
\newcommand{\ii}{\mathbf{i}}
\newcommand{\jj}{\mathbf{j}}
\newcommand{\rr}{\mathbf{r}}
\newcommand{\kk}{\mathbf{k}}
\newcommand{\uu}{\mathbf{u}}
\newcommand{\vv}{\mathbf{v}}
\newcommand{\ww}{\mathbf{w}}
\newcommand{\yy}{\mathbf{y}}
\newcommand{\R}{\mathbb{R}}
\newcommand{\T}{\mathcal{T}}
\newcommand{\N}{\mathbb{N}}
\newcommand{\C}{\mathscr{C}}
\newcommand{\Z}{\mathbb{Z}}
\newcommand{\B}{\mathcal{B}}
\newcommand{\Q}{\mathbb{Q}}
\newcommand{\Sph}{\mathbb{S}}
\newcommand{\D}{\mathbb{D}}

\def\dim{\mathop{\rm dim}\nolimits}
\def\image{\mathop{\rm Im}\nolimits}  
\def\interior{\mathop{\rm Int}\nolimits}
\def\kernel{\mathop{\rm Ker}\nolimits}
\def\cokernel{\mathop{\rm Coker}\nolimits}
\def\bd{\mathop{\rm Bd}\nolimits}
\def\ext{\mathop{\rm Ext}\nolimits}
\def\num{\mathop{\#}\limits}
\def\cl{\mathop{\rm Cl}\nolimits}
\def\lub{\mathop{\rm lub}\nolimits}
\def\ess{\mathop{\rm ess}\nolimits}
\def\sgn{\mathop{\rm sgn}\nolimits}


\newcommand{\dif}{\, \mathrm{d}}

\newcommand{\proj}{\operatorname{proj}}
\newcommand{\ds}{\displaystyle}
\newcommand{\pa}{\partial}
\newcommand{\ol}{\overline{}}
\newcommand{\toM}{\overset{m}{\to}}
\newcommand{\floor}[1]{\left\lfloor #1 \right\rfloor}
\newcommand{\ceil}[1]{\left\lceil #1 \right\rceil}
\newcommand{\norm}[1]{\left\lVert #1 \right\rVert}
\newcommand{\abs}[1]{\left\lvert #1 \right\rvert}



\newcommand{\degree}{^{\circ}}
\renewcommand{\epsilon}{\varepsilon}

\newcommand{\upint}[2]{
  \overline{\int_{#1}^{#2}}
}
\newcommand{\lowint}[2]{
  \underline{\int_{#1}^{#2}}
}


\pagestyle{fancy}
\lhead{Math 6421: Measure Theory}
\chead{\bf Jose Nino: Lecture \#21}
\rhead{November 9, 2023}
\cfoot{Page \thepage}
% \setlength{\headheight}{10pt}

\theoremstyle{definition}
\newtheorem*{thm}{Theorem}
\newtheorem*{exercise}{Exercise}
\newtheorem*{definition}{Definition}
\newtheorem{problem}{Problem}
\newtheorem*{lemma}{Lemma}
\newtheorem*{cor}{Corollary}
\newtheorem*{prop}{Proposition}
\newtheorem*{remark}{Remark}


\begin{document}

\begin{definition}
  
  Let \( (X, \B, \mu) \) be a measure space (and will always be implicitly assumed).  
  Let \( f \geq 0 \) be a nonnonegative measurable function. Then define
    \[
        \int f \dif \mu := \sup_{\phi} \left\{ \int \phi \dif \mu \right\}
    \]
  where \( 0 \leq \phi \leq f \) is a simple function. 
\end{definition}

\begin{prop}

  Note that the following comes from the definition above:
    \begin{enumerate}[label = (\arabic{*})]
      \item If \( c \geq 0 \), then \( \displaystyle \int cf  = c \int f \).
      \item If \( 0 \leq f \leq g \), then \( \displaystyle \int f \leq \int g \).
    \end{enumerate}

\end{prop}

\begin{remark}
    We will prove the linearity of integrals in a general measure space i.e., for any \( \alpha, \beta \in \R \), we have that 
      \[
        \int \alpha f + \int \beta g = \alpha \int f + \beta \int g.
      \]

\end{remark}

\begin{thm}[11.11, Fatuo's Lemma]

    Let \( \{f_n\} \) be a sequence of nonnegative measurable functions. 
    Suppose that \( f_n \to f \) almost everywhere on \( E \in \B \).
    Then 
      \[
          \int_{E} f \dif \mu \leq \varliminf_{n \to \infty} \int_{E} f_n.
      \]
        \begin{proof}
          Without loss of generality, we may assume that \( f_n \to f \) everywhere for each \( x \in E \). 
          We want to show that for any simple function \( 0 \leq \phi \leq f \), we have 
            \[
                \int_{E} \phi \dif \mu \leq \varliminf_{n \to \infty} \int_{E} f_n \dif \mu.
            \] 
          We can write the simple function \( \phi \) in its canonical representation i.e.,
            \[
                \phi(x) = \sum_{i=1}^{n} c_i \chi_{E_{i}}
            \]
          for \( c_i \in \R \).

          We have two cases to show: (i) If \( \displaystyle \int_{E} \phi = \infty \), then \( \displaystyle \int_{E} f_n \to \infty \).

          Then this means that there exists \( i \in \N \) such that \( \mu \left( E_i \cap E\right) = \infty \). 
          For notation, let \( a = c_i \) and \( A = E_i \cap E \).
          Consider the set 
            \[
              A_n = \left\{ x \in E : f_k(x) > a, \ \text{for all} \ k \geq n\right\}.  
            \]
          We can note two things:
            \begin{enumerate}[label = (\arabic{*})]
              \item The sequence \( \{A_n\} \) is an increasing sequence i.e., \( A_{n+1} \supset A_{n} \),
              \item And \( \displaystyle A = \bigcup_{n=1}^{\infty} A_n \).
            \end{enumerate}
          Since \( \mu(A) = \infty \) and \( \{A_n\} \) is an increasing sequence (i.e., the measure of each subsequence \(A_n\) is increasing), we know that \( \mu(A_n) \to \infty \).
          This implies that 
            \[  
                \int_{E} f_n \geq a \cdot \mu(A_n)
            \]
          From this, we know that 
            \[
                \lim_{n \to \infty} = \infty = \int_{E} \phi
            \]
          and completes this case.

          For case (ii), we will suppose \( \displaystyle \int_{E} \phi < \infty \). Define the set 
            \[  
                A = \left\{ x \in E: \phi(x) > 0 \right\} \in \B 
            \]
          which has \( \mu(A) < \infty \) because \( A \subset E \) and \( E \) has finite measure as well. 

          Let \( \epsilon > 0 \) be chosen, and let \( M \) be equal to the max of \( \phi \).
          For this fixed \( \epsilon \), we can construct a sequence of sets such that 
            \[
                A_n = \left\{  x \in E : f_k(x) > (1 - \epsilon) \phi(x), \ \text{for all} \ k \geq n \right\}.
            \]
          Since \( f_n \to f \) on \(E \), we know that \( \{A_n\} \) is an increasing sequence as in case (i), and also \( \displaystyle \lim_{n \to \infty} A_n = A \subset \bigcup_{n=1}^{\infty} A_n \).
          In other words, \( A \setminus A_n = A \cap A_n^{\C} \) and the sequence \( \{ A \setminus A_n\} \) is decreasing which means that 
            \[
                \bigcap_{n=1}^{\infty} \left( A \setminus A_n \right) = \emptyset.
            \]
          This implies that \( \displaystyle \lim_{n \to \infty} \mu(A \setminus A_n) = 0 \). 
          So for our fixed \( \epsilon \), there exists \( n \in \N \) such that \( \mu(A \setminus A_n) < \epsilon \).
          Then for all \( k \geq n \),
            \begin{align*}
              \int_{E} f_k &\geq \int_{A_k} \geq \int_{A_{k}} (1 - \epsilon) \phi(x) \\
              &\geq (1 - \epsilon) \int_{E} \phi - \int_{A \setminus A_{k}} \phi 
            \end{align*}
          and so it follows that 
            \begin{align*}
              (1- \epsilon) \int_{A_{k}} \phi + \int_{A \setminus A_{k}} \phi  
              &\geq (1 - \epsilon) \int_{A} \phi \\
              &\geq \int_{E} \phi - \epsilon \left( \int_{E} \phi + M \right).
            \end{align*}
          Thus, from the definition of \( \liminf \),
            \[
                \varliminf_{n \to \infty} \int_{E} f_n \geq int_{E} \phi - \epsilon \left( \int_{E} \phi + M \right)
            \]
          and since \( \epsilon \) was arbitrary,
            \[ \varliminf_{n \to \infty} \int_{E} f_n \geq \int_{E} \phi. \]
        \end{proof}

\end{thm}

\begin{thm}[11.12, Monotone Convergence Theorem]

  Let \( \{f_n\} \) be a sequence of nonnegative measurable functions such that \( f_n \to f \) almost everywhere.
  Suppose that for all \( n \in \N \), \( f_n \leq f \).
  Then
    \[
        \int f = \lim_{n \to \infty} \int f_n.
    \]  
      \begin{proof}
        By monotonicity, since \( f_n \geq f \),
          \[
              \int f_n \leq \int f.
          \]
        Then by Fatuo's lemma,
          \[
              \varlimsup_{n \to \infty} \int f_n \leq \int f \leq \varliminf_{n \to \infty} \int f_n.
          \]
      \end{proof}
\end{thm}


\begin{prop}[11.13, Linearity]

  If \( f, g \geq 0 \) and \( a,b \geq 0 \), then
    \[  
        \int af + \int bg = a \int f + b \int g.
    \]
  We have \( \displaystyle \int f \geq 0 \) with equality if and only if \( f = 0 \) almost everywhere. 

\end{prop}

\begin{definition}
      Let \( f \geq 0 \). 
      
        \begin{enumerate}[label = (\arabic{*})]
          \item Then \( f \) is called integrable over \( E \in \B \) if
            \[
                \int f \dif \mu < \infty
            \]
          or \( f \in L^1(E) \), \( f \in L^1(\mu) \), or \( f \in L^1(X, \mu) \).
          \item A measurable function \( f \) is called integrable if \( f^+ \) and \( f^- \)are integrable. In this case.
            \[
                \int_{E} f := \int_{E} f^+ \int_{E} f^-.
            \]
        \end{enumerate}

\end{definition}

\begin{prop}

  Let \( f,g \in L^1(X, \mu) \).
  Then we have 
    \begin{enumerate}[label = (\arabic{*})]
      \item \( \displaystyle \int_{E} af + bg = a \int_{E} f + b \int_{E} g \).
      \item If \( \abs{h} \leq \abs{f} \) and \( h \) is measurable, then
       \( h \in L^1(X, \mu )\).
       \item If \( f \geq g \) almost everywhere, then \( \displaystyle \int f \geq \int g \).
    \end{enumerate}

\end{prop}

\begin{thm}[11.16, Lesbesgue Convergence Theorem]

    Let \( f \in L^1(X, \mu) \) and suppose \( \{f_n\} \) is a sequence of measurable functions such that \( \abs{f_n(x) \leq g } \) and such that almost everywhere, \( f_n(x) \to f \) for \( x \in E \). Then
      \[  
          \int_{E} f = \lim_{n \to \infty} f_n.
      \]
  
\end{thm}

\subsection*{Section 11.3 General Convergence Theorems}

\begin{definition}
  We say \( \{\mu_{n}\}_{n=1}^{\infty} \) converges to \( \mu \) setwisely if \( \mu_n(A) \to \mu(A) \) for all \( A \in \B \).
\end{definition}

\begin{prop}
  
  Let \( \{\mu_n\} \) be a sequence of measures which converges setwisely to \( \mu \), and \( \{f_n\} \) be nonnegative and converge to \( f \) pointwise. Then
    \[
        \int f \dif \mu = \varliminf \int f_n \dif \mu.
    \]
\end{prop}
\end{document}