\documentclass[12pt]{article}
\usepackage{amsmath,fullpage,graphicx,fancyhdr,enumerate,amsthm,amssymb,tikz,tikz-cd,color,pgfplots,tabularx,amsfonts,amscd}
\usepackage{enumitem}
\usetikzlibrary{arrows,chains,matrix,positioning,scopes}
\pgfplotsset{compat=1.13}
\setlength{\headheight}{15pt}
\setlength{\headsep}{7pt}
\newcommand{\bsnl}{\bigskip\newline}
\pagestyle{fancy}
\usepackage[paper=a4paper, margin = 1in]{geometry}
\usepackage{hyperref}% http://ctan.org/pkg/hyperref
\usepackage[capitalise, noabbrev]{cleveref}

\hypersetup{ % colors hyperlinks with colors to make noticeable but is not an ugly green box like default
    colorlinks,
    linkcolor={red!50!black},
    citecolor={blue!50!black},
    urlcolor={blue!80!black}
}
\usepackage[mathscr]{euscript}\usepackage{pifont}
\usepackage{cleveref}
\usepackage{parskip}
\usepackage[makeroom]{cancel}


\newcommand{\cmark}{\ding{51}}%
\newcommand{\xmark}{\ding{55}}%

\renewcommand{\aa}{\mathbf{a}}
\newcommand{\bb}{\mathbf{b}}
\newcommand{\cc}{\mathbf{c}}
\newcommand{\dd}{\mathbf{d}}
\newcommand{\FF}{\mathbf{F}}
\newcommand{\ii}{\mathbf{i}}
\newcommand{\jj}{\mathbf{j}}
\newcommand{\rr}{\mathbf{r}}
\newcommand{\kk}{\mathbf{k}}
\newcommand{\uu}{\mathbf{u}}
\newcommand{\vv}{\mathbf{v}}
\newcommand{\ww}{\mathbf{w}}
\newcommand{\yy}{\mathbf{y}}
\newcommand{\R}{\mathbb{R}}
\newcommand{\T}{\mathcal{T}}
\newcommand{\N}{\mathbb{N}}
\newcommand{\C}{\mathscr{C}}
\newcommand{\Z}{\mathbb{Z}}
\newcommand{\B}{\mathcal{B}}
\newcommand{\Q}{\mathbb{Q}}
\newcommand{\Sph}{\mathbb{S}}
\newcommand{\D}{\mathbb{D}}

\def\dim{\mathop{\rm dim}\nolimits}
\def\image{\mathop{\rm Im}\nolimits}  
\def\interior{\mathop{\rm Int}\nolimits}
\def\kernel{\mathop{\rm Ker}\nolimits}
\def\cokernel{\mathop{\rm Coker}\nolimits}
\def\bd{\mathop{\rm Bd}\nolimits}
\def\ext{\mathop{\rm Ext}\nolimits}
\def\num{\mathop{\#}\limits}
\def\cl{\mathop{\rm Cl}\nolimits}
\def\lub{\mathop{\rm lub}\nolimits}
\def\ess{\mathop{\rm ess}\nolimits}
\def\sgn{\mathop{\rm sgn}\nolimits}


\newcommand{\dif}{\, \mathrm{d}}

\newcommand{\proj}{\operatorname{proj}}
\newcommand{\ds}{\displaystyle}
\newcommand{\pa}{\partial}
\newcommand{\ol}{\overline{}}
\newcommand{\toM}{\overset{m}{\to}}
\newcommand{\floor}[1]{\left\lfloor #1 \right\rfloor}
\newcommand{\ceil}[1]{\left\lceil #1 \right\rceil}
\newcommand{\norm}[1]{\left\lVert #1 \right\rVert}
\newcommand{\abs}[1]{\left\lvert #1 \right\rvert}



\newcommand{\degree}{^{\circ}}
\renewcommand{\epsilon}{\varepsilon}

\newcommand{\upint}[2]{
  \overline{\int_{#1}^{#2}}
}
\newcommand{\lowint}[2]{
  \underline{\int_{#1}^{#2}}
}


\pagestyle{fancy}
\lhead{Math 6421: Measure Theory}
\chead{\bf Jose Nino: Lecture \#20}
\rhead{November 7, 2023}
\cfoot{Page \thepage}
% \setlength{\headheight}{10pt}

\theoremstyle{definition}
\newtheorem*{thm}{Theorem}
\newtheorem*{exercise}{Exercise}
\newtheorem*{definition}{Definition}
\newtheorem{problem}{Problem}
\newtheorem*{lemma}{Lemma}
\newtheorem*{cor}{Corollary}
\newtheorem*{prop}{Proposition}
\newtheorem*{remark}{Remark}


\begin{document}

\begin{prop}[11.1]

  If \( A \in \B \), \( B \in \B \), and \( A \subset B \), then 
    \[
        \mu(A) \leq \mu(B).
    \]

  

\end{prop}

\begin{prop}[11.2]

  If \( E_i \in \B \), \( \mu(E_1) < \infty \) and \( E_i \supset E_{i+1} \), then 
    \[
        \mu \left( \bigcap_{i=1}^{\infty} E_i \right) = \lim_{n \to \infty} \mu \left( E_n \right).
    \]


\end{prop}


\begin{prop}[11.3]

  If \( E_i \in \B \), then
    \[
        \mu \left( \bigcup_{i=1}^{\infty} E_i \right) \leq \sum_{i=1}^{n} \mu \left( E_i \right).
    \]


\end{prop}

\begin{definition}

  Let \( (X, \B, \mu ) \) be a measure space. Then  
    \begin{enumerate}[label = (\arabic{*})]
      \item If \( \mu(x) < \infty \) and hence \( \mu(E) < \infty \) for all \( E \in \B \) then \( \mu \) is called \textbf{finite}.
      
      \item Let \( \displaystyle X = \bigcup_{i=1}^{\infty} E_i \) where \( E_i \in \B \) and \( \mu(E_i) < \infty \) for all \( i \in \N \).
      Then \( \mu \) is called \( \sigma \)-\textbf{finite}.
       
      \item  If for all \( E \in \B \) with \( \mu(E) = \infty \) and  there exists nonempty \( F \subset E \) such that \( F \in \B \) and \( \mu(F) < \infty \), then \( \mu \) is called \textbf{semi-finite}.
    \end{enumerate}

\end{definition}

\begin{remark}

  Note if \( \mu \) is \( \sigma \)-finite, then \( \mu \) is semi-finite.  Additionally, note the following:
    \begin{enumerate}[label = (\arabic{*})]
      \item Note that
        \[
            some stuff.
        \]
      \item The triple \( (\R, \mathcal{M}, \B) \) is \( \sigma \)-finite because 
        \[
            \R = \bigcup_{n=1}^{\infty} [n, n+1] \cup \bigcup_{n=-1}^{\infty}[n-1, n] \cup [-1,1].
        \]
        \item We will mostly only discuss \( \sigma \)-finite measure.
    \end{enumerate}
  
\end{remark}


\begin{definition}
  A measure space \( (X, \B, \mu) \) is said to be \textbf{complete} if \( \B \) contains all subsets of measure zero i.e., if \( B \in \B \), \( \mu(B) = 0 \), and \( A \subset B \), then \( A \in \B \).
\end{definition}

\begin{prop}[11.4]

  If \( (X, \B, \mu) \) is a measure space, then there exists a complete measure space \( (X, \B_0, \mu_0) \) such that 
    \begin{enumerate}[label = (\roman{*})]
      \item \( \B \subset \B_0 \).  
      \item If \( E \in \B \), then \( \mu(E) = \mu_0(E) \). 
      \item \( E \in \B_0 \) if and only if \( E = A \cup B \) where \( B \in \B \) and \( A \subset C \), \( C \in \B \), and \( \mu(C) = 0 \). 
    \end{enumerate}

\end{prop}

\subsection*{Section 11.2 Measurable Functions}

Let \( (X, \B) \) be a measurable space for any of the following propositions and definitions. 

\begin{prop}[11.5]
  
  Let \( f: X \to \overline{\R} \) be a function, and let \( \alpha \in \R \) be fixed.
  Then the following statements are equivalent:
    \begin{enumerate}[label = (\roman{*})]
        \item \(\{x: f(x) < \alpha\} \in \B \). 
        \item \(\{x: f(x) \leq \alpha\} \in \B \).  
        \item \(\{x: f(x) > \alpha\} \in \B \). 
        \item \(\{x: f(x) \geq \alpha\} \in \B \). 
    \end{enumerate}
\end{prop}

\begin{definition}
  The function \( f: X \to \overline{\R} \) is a \textbf{measurable function} if any of the above statements in Proposition 11.5 hold. 
\end{definition}


\begin{thm}[11.6]

  If \( c \in \R \) and the functions \( f \) and \( g \) are measurable, then so are the functions \( f + c \), \( f + g \), \( f \cdot g \), and \( f \lor g \).
  Moreover, if \( \{f_n\} \) is a sequence of functions, then \( \sup f_n \), \( \inf f_n \), \( \varlimsup f_n \), and \( \varliminf f_n \) are all measurable.

\end{thm}

\begin{definition}
  Define a \textbf{simple function} by 
    \[
        \phi(x) = \sum_{i=1}^{n} a_i X_{E_{i}}
    \]
  for \( a_i \in \R \) and with \( E_i \in \B \) where \( E_i \cap E_j  = \emptyset \) for all \( i \neq j \).
\end{definition}


\begin{prop}[11.7]

  Let \( f \) be an nonnegative measurable function. 
  Then there is a sequence \( \{\phi_n\} \) of simple functions with \( \phi_{n+1} \geq \phi_n \) such that \( \displaystyle f = \lim_{n \to \infty} \phi_n \) at each point of \( X \), If \( f \) is defined on a  \( \sigma \)-finite measure space, then we may choose the functions \( \phi_n \) so that each vanishes outside a set of finite measure. 

\end{prop}

\begin{prop}[11.8]

  If \( \mu \) is a complete measure and \( f \) is a measurable function, then \( f = g \) almost everywhere implies \( g \) is measurable. 
  
\end{prop}

\subsection*{Section 11.3 Integration}

Let \( (X, \B, \mu) \) be a measure space.


\begin{definition}

  Let
    \[
        \phi(x) = \sum_{i=1}^{n} a_i \chi_{E_{i}}
    \]
  be a simple function. The integration of \( \phi \) with respect to \( \mu \) on \( E \) is defined as 
    \[
        \int_{E} \phi = \int_{E}  \phi \dif \mu := \sum_{i=1}^{n} a_i \cdot \mu \left( E_i \cup E \right).
    \]
\end{definition}

\begin{prop}

  Let \( \phi, \psi \) be nonnegative simple functions. 
    \begin{enumerate}[label = (\alph{*})]
      \item If \( \alpha, \beta  \geq 0 \), then 
        \[
             \int_{E} \alpha \phi + \beta \int_{E} \psi = \alpha \int_{E} \phi + \beta \int_{E} \psi.
        \]
      \item If \( 0 \geq \phi \leq \psi \), then 
          \[
              \int_{E} \phi \leq \int_{E} \psi.
          \]
      \item The map \( \eta: \B \to \R^+ \cup \{ 0 \} \) defined by \( \displaystyle A \mapsto \int_{A} \phi \dif \mu \) is a measure on \( \B \).
    \end{enumerate}
  
\end{prop}


\end{document}