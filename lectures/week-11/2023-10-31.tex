\documentclass[12pt]{article}
\usepackage{amsmath,fullpage,graphicx,fancyhdr,enumerate,amsthm,amssymb,tikz,tikz-cd,color,pgfplots,tabularx,amsfonts,amscd}
\usepackage{enumitem}
\usetikzlibrary{arrows,chains,matrix,positioning,scopes}
\pgfplotsset{compat=1.13}
\setlength{\headheight}{15pt}
\setlength{\headsep}{7pt}
\newcommand{\bsnl}{\bigskip\newline}
\pagestyle{fancy}
\usepackage[paper=a4paper, margin = 1in]{geometry}
\usepackage{hyperref}% http://ctan.org/pkg/hyperref
\usepackage[capitalise, noabbrev]{cleveref}

\hypersetup{ % colors hyperlinks with colors to make noticeable but is not an ugly green box like default
    colorlinks,
    linkcolor={red!50!black},
    citecolor={blue!50!black},
    urlcolor={blue!80!black}
}
\usepackage[mathscr]{euscript}\usepackage{pifont}
\usepackage{cleveref}
\usepackage{parskip}
\usepackage[makeroom]{cancel}


\newcommand{\cmark}{\ding{51}}%
\newcommand{\xmark}{\ding{55}}%

\renewcommand{\aa}{\mathbf{a}}
\newcommand{\bb}{\mathbf{b}}
\newcommand{\cc}{\mathbf{c}}
\newcommand{\dd}{\mathbf{d}}
\newcommand{\FF}{\mathbf{F}}
\newcommand{\ii}{\mathbf{i}}
\newcommand{\jj}{\mathbf{j}}
\newcommand{\rr}{\mathbf{r}}
\newcommand{\kk}{\mathbf{k}}
\newcommand{\uu}{\mathbf{u}}
\newcommand{\vv}{\mathbf{v}}
\newcommand{\ww}{\mathbf{w}}
\newcommand{\yy}{\mathbf{y}}
\newcommand{\R}{\mathbb{R}}
\newcommand{\T}{\mathcal{T}}
\newcommand{\N}{\mathbb{N}}
\newcommand{\C}{\mathscr{C}}
\newcommand{\Z}{\mathbb{Z}}
\newcommand{\B}{\mathcal{B}}
\newcommand{\Q}{\mathbb{Q}}
\newcommand{\Sph}{\mathbb{S}}
\newcommand{\D}{\mathbb{D}}

\def\dim{\mathop{\rm dim}\nolimits}
\def\image{\mathop{\rm Im}\nolimits}  
\def\interior{\mathop{\rm Int}\nolimits}
\def\kernel{\mathop{\rm Ker}\nolimits}
\def\cokernel{\mathop{\rm Coker}\nolimits}
\def\bd{\mathop{\rm Bd}\nolimits}
\def\ext{\mathop{\rm Ext}\nolimits}
\def\num{\mathop{\#}\limits}
\def\cl{\mathop{\rm Cl}\nolimits}
\def\lub{\mathop{\rm lub}\nolimits}
\def\ess{\mathop{\rm ess}\nolimits}

\newcommand{\dif}{\, \mathrm{d}}

\newcommand{\proj}{\operatorname{proj}}
\newcommand{\ds}{\displaystyle}
\newcommand{\pa}{\partial}
\newcommand{\ol}{\overline{}}
\newcommand{\toM}{\overset{m}{\to}}
\newcommand{\floor}[1]{\left\lfloor #1 \right\rfloor}
\newcommand{\ceil}[1]{\left\lceil #1 \right\rceil}
\newcommand{\norm}[1]{\left\lVert #1 \right\rVert}



\newcommand{\degree}{^{\circ}}
\renewcommand{\epsilon}{\varepsilon}

\newcommand{\upint}[2]{
  \overline{\int_{#1}^{#2}}
}
\newcommand{\lowint}[2]{
  \underline{\int_{#1}^{#2}}
}


\pagestyle{fancy}
\lhead{Math 6421: Measure Theory}
\chead{\bf Jose Nino: Lecture \#18}
\rhead{October 31, 2023}
\cfoot{Page \thepage}
% \setlength{\headheight}{10pt}

\theoremstyle{definition}
\newtheorem*{thm}{Theorem}
\newtheorem*{exercise}{Exercise}
\newtheorem*{definition}{Definition}
\newtheorem{problem}{Problem}
\newtheorem*{lemma}{Lemma}
\newtheorem*{cor}{Corollary}
\newtheorem*{prop}{Proposition}
\newtheorem*{remark}{Remark}


\begin{document}

\begin{lemma}[\textbf{5.13}]

    If \( f \) is absolutely continuous on \( [a,b] \) and \( f'(x) = 0 \) almost everywhere, then \( f \) is constant. 
    
    \begin{proof}
        Let \( E \subset [a, c] \) be the set such that for all \( x \in E \), \( f'(x) = 0 \).
        We claim that for all \( c \in [a,b] \), \( f(a) = f(c) \). 
        Let \( \epsilon, \eta > 0 \) be chosen. 
        For every \( x \in E \), this means there exists \( h > 0 \) such that 
        \( (x, x + h) \subset [a,c] \) and \( |f(x + h) - f(x)| < \eta \cdot h \) since \( \eta \) is arbitrary.
        By Vitali covering lemma, for any \( \delta > 0 \), there exists a finite cover \( \{ x_k, y_k \} \) 
        of nonoverlapping intervals contained in \( [a,c]\)
        such that 
            \[
                \sum_{k=0}^{n} | x_{k+1} - y_k | < \delta.     
            \]
        So then 
            \begin{align*}
                \sum_{k=1}^{n} |f(y_k) - f(x_k)| \leq \eta n \sum_{k=1}^{n} (y_k - x_k) \leq \eta (c-a).
            \end{align*}
        Moreover, by the absolute continuity, there exists a \( \delta > 0 \) for our fixed \( \epsilon \) such that
            \[
                \sum_{k=0}^{n} |f(x)_{k+1} - f(y_k)|  < \epsilon.
            \]
        Then 
            \begin{align*}
                |f(x) - f(a)| &= \left| \sum_{k=0}^{n} f(x_{k+1} - f(y_k)) + \sum_{k=0}^{n} f(y_k) - f(x_k) \right| \\
                &\leq \left| \sum_{k=0}^{n} f(x_{k+1} - f(y_k)) \right| +  \left| \sum_{k=0}^{n} f(y_k) - f(x_k) \right| \\
                &< \epsilon + \eta(c-a).
            \end{align*}
        Because \( \epsilon \) and \( \eta \) are arbitrary, \( |f(c) - f(a)| = 0 \)
        and so \( f(c) = f(a) \) (i.e., \( f \) is constant).
    \end{proof}
\end{lemma}

\begin{thm}[\textbf{5.14}]

    A function \( F \) is an indefinite integral if and only \( F \) is absolutely continuous. 
    \begin{proof}
        We will show two directions. 
        \begin{enumerate}
            \item[\((\Rightarrow)\)] Suppose \( F \) is an indefinite integral i.e.,
                \[
                    F(x) = \int_{a}^{c} f(t) \dif t.  
                \]
            Fix \( \epsilon > 0 \).
            By Proposition 4.14, if \( f \geq 0 \) and \( f \in L^1 \), then there exists \( \delta > 0 \) such that if \( m(A) < \delta \), then 
                \[
                    \int_{A} f < \epsilon.  
                \]
            Thus absolute continuity follows from this proposition. 
            \item[\((\Leftarrow)\)] Now suppose \( F \) is absolutely continuous. Because absolute continuity implies a function is of bounded variation,
            we know that \( F'(x) \) exists almost everywhere. 
            Additionally, this means it is the subtraction of two monotone increasing functions i.e., \( F(x) = F_1(x) - F_2(x) \).
            So then 
                \begin{align*}
                    |F'(x)| &\leq F'_1(x) + F'_2(x).
                \end{align*}
            This implies that
                \begin{align*}
                    \int_{a}^{b} |F'(x)| &\leq \int_{a}^{b} F'_{1}(x) + \int_{a}^{b} F'_{2}(x) \\
                    &\leq \left( F_{1}(b) - F_{2}(a) \right) + \left( F_{2}(b) - F_{2}(a) \right)
                \end{align*}
            This means that \( F'(x) \) is integrable on \( (a,b) \). 
            Consider the function 
                \[ 
                    G(x) = \int_{a}^{x} F'(t) \dif t.
                \]
            So \( G(x) \) is absolutely continuous on \( [a,b] \) and 
                \begin{align*}
                    \left(G(x) - F(x) \right)' &= G'(x) - F'(x) \\
                    &= 0 
                \end{align*}
            almost everywhere. By the previous lemma, we know that \( G(x) - F(x) = x = F(a) \) and so take \( x = a \). Therefore, 
                \[  
                    F(x) = \int_{a}^{x} F'(t) \dif t + F(a).    
                \]
        \end{enumerate}
    \end{proof}
\end{thm}

\section*{Section 6.5 Bounded Linear Functionals on the \( L^p \) Space}

\begin{definition}
    Let \( \left(X, \norm{\cdot}\right)\) be a normed linear space. 
    A \textbf{linear functional} is a map \( F: X \to \R \) such that 
        \[
            F(\alpha f + \beta g) = \alpha F(f) + \beta F(g) 
        \]
    for all \( f,g \in X\) and for \( \alpha, \beta \in \R \). 

    A linear functional \( F \) is \textbf{bounded} if there exists \( M > 0 \) such that
        \[
            \left| F(f) \right| \leq M \cdot \norm{f} \ \text{for all} \ f \in X.
        \]
    Finally, we can define the norm of \( F \) by
        \[
            \norm{F} = \sup_{f \in X} \frac{|F(f)|}{\norm{f}}.  
        \]
        
\end{definition}

For sake of clarity, \( X = L^p \) and \( \norm{\cdot} = \norm{\cdot}_{p} \) 
and \( p, q \in \R \) always satisfy the equation \( \displaystyle \frac{1}{p}  + \frac{1}{q} = 1 \).
\begin{remark}
    If \( g \in L^q \), define
        \[
            F_{g}(f) = \int f \cdot g \ \text{for all} \ f \in L^p.  
        \]
    By H\"{o}lder's inequality
        \begin{align*}
            \int f \cdot g \leq \norm{f}_{p} \cdot \norm{g}_{g}
        \end{align*}
    and so this implies that 
        \[
            \norm{Fg} \leq \norm{g}_{q}.
        \]
    The linear functional \( Fg: L^p \to \R \) is a bounded linear functional. 
\end{remark}

\begin{prop}[\textbf{6.11}]

    \[
        \norm{Fg} = \norm{g}_{q}.  
    \]
        \begin{proof}
            We claim there exists \( f \in L^p \) such \( F(f) = \norm{f}_{p}\cdot \norm{g}_{q} \).
            Set \( f = |g|^{q/p}\). 
        \end{proof}

\end{prop}

We know that \( \displaystyle Fg = \int fg \) defines a bounded linear functional
and also vise versa---that is, all bounded, linear functionals can take the form 
    \[
        F(f) = \int f \cdot g  
    \]
for some \( g \in L^q \).


\begin{lemma}[\textbf{6.12}]

    Let \( g \) be an integrable function over \( [0,1] \). Suppose there exists \( M >0 \) such that 
        \begin{align*}
            \left| \int fg \right| \leq M \cdot \norm{f}_{p}
        \end{align*}
    for all \( f \) which are bounded, measurable functions. 
    Then \( g \in L^q \) and \( \norm{g}_{q} \leq M \).
    
\end{lemma}

\end{document}