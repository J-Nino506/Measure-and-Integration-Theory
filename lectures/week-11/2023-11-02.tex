\documentclass[12pt]{article}
\usepackage{amsmath,fullpage,graphicx,fancyhdr,enumerate,amsthm,amssymb,tikz,tikz-cd,color,pgfplots,tabularx,amsfonts,amscd}
\usepackage{enumitem}
\usetikzlibrary{arrows,chains,matrix,positioning,scopes}
\pgfplotsset{compat=1.13}
\setlength{\headheight}{15pt}
\setlength{\headsep}{7pt}
\newcommand{\bsnl}{\bigskip\newline}
\pagestyle{fancy}
\usepackage[paper=a4paper, margin = 1in]{geometry}
\usepackage{hyperref}% http://ctan.org/pkg/hyperref
\usepackage[capitalise, noabbrev]{cleveref}

\hypersetup{ % colors hyperlinks with colors to make noticeable but is not an ugly green box like default
    colorlinks,
    linkcolor={red!50!black},
    citecolor={blue!50!black},
    urlcolor={blue!80!black}
}
\usepackage[mathscr]{euscript}\usepackage{pifont}
\usepackage{cleveref}
\usepackage{parskip}
\usepackage[makeroom]{cancel}


\newcommand{\cmark}{\ding{51}}%
\newcommand{\xmark}{\ding{55}}%

\renewcommand{\aa}{\mathbf{a}}
\newcommand{\bb}{\mathbf{b}}
\newcommand{\cc}{\mathbf{c}}
\newcommand{\dd}{\mathbf{d}}
\newcommand{\FF}{\mathbf{F}}
\newcommand{\ii}{\mathbf{i}}
\newcommand{\jj}{\mathbf{j}}
\newcommand{\rr}{\mathbf{r}}
\newcommand{\kk}{\mathbf{k}}
\newcommand{\uu}{\mathbf{u}}
\newcommand{\vv}{\mathbf{v}}
\newcommand{\ww}{\mathbf{w}}
\newcommand{\yy}{\mathbf{y}}
\newcommand{\R}{\mathbb{R}}
\newcommand{\T}{\mathcal{T}}
\newcommand{\N}{\mathbb{N}}
\newcommand{\C}{\mathscr{C}}
\newcommand{\Z}{\mathbb{Z}}
\newcommand{\B}{\mathcal{B}}
\newcommand{\Q}{\mathbb{Q}}
\newcommand{\Sph}{\mathbb{S}}
\newcommand{\D}{\mathbb{D}}

\def\dim{\mathop{\rm dim}\nolimits}
\def\image{\mathop{\rm Im}\nolimits}  
\def\interior{\mathop{\rm Int}\nolimits}
\def\kernel{\mathop{\rm Ker}\nolimits}
\def\cokernel{\mathop{\rm Coker}\nolimits}
\def\bd{\mathop{\rm Bd}\nolimits}
\def\ext{\mathop{\rm Ext}\nolimits}
\def\num{\mathop{\#}\limits}
\def\cl{\mathop{\rm Cl}\nolimits}
\def\lub{\mathop{\rm lub}\nolimits}
\def\ess{\mathop{\rm ess}\nolimits}
\def\sgn{\mathop{\rm sgn}\nolimits}


\newcommand{\dif}{\, \mathrm{d}}

\newcommand{\proj}{\operatorname{proj}}
\newcommand{\ds}{\displaystyle}
\newcommand{\pa}{\partial}
\newcommand{\ol}{\overline{}}
\newcommand{\toM}{\overset{m}{\to}}
\newcommand{\floor}[1]{\left\lfloor #1 \right\rfloor}
\newcommand{\ceil}[1]{\left\lceil #1 \right\rceil}
\newcommand{\norm}[1]{\left\lVert #1 \right\rVert}
\newcommand{\abs}[1]{\left\lvert #1 \right\rvert}



\newcommand{\degree}{^{\circ}}
\renewcommand{\epsilon}{\varepsilon}

\newcommand{\upint}[2]{
  \overline{\int_{#1}^{#2}}
}
\newcommand{\lowint}[2]{
  \underline{\int_{#1}^{#2}}
}


\pagestyle{fancy}
\lhead{Math 6421: Measure Theory}
\chead{\bf Jose Nino: Lecture \#19}
\rhead{November 2, 2023}
\cfoot{Page \thepage}
% \setlength{\headheight}{10pt}

\theoremstyle{definition}
\newtheorem*{thm}{Theorem}
\newtheorem*{exercise}{Exercise}
\newtheorem*{definition}{Definition}
\newtheorem{problem}{Problem}
\newtheorem*{lemma}{Lemma}
\newtheorem*{cor}{Corollary}
\newtheorem*{prop}{Proposition}
\newtheorem*{remark}{Remark}


\begin{document}

\begin{lemma}[\textbf{6.12}]

    Let \( g \) be an integrable function over \( [0,1] \). Suppose there exists \( M >0 \) such that 
        \begin{align*}
            \left| \int fg \right| \leq M \cdot \norm{f}_{p}
        \end{align*}
    for all \( f \) which are bounded, measurable functions. 
    Then \( g \in L^q \) and \( \norm{g}_{q} \leq M \).

    \begin{proof}
        We claim that \( \norm{g}_{q} \leq M \). 
        We are done if we prove this because then \( g \in L^q \).
        First, assume that \( p \in (1, \infty) \) and so \( q \) is over the same region as well.
        Define the function \( g_n \) by
            \[
                g_n(x) = \begin{cases}
                    g(x) & |g(x)| \leq n \\
                    0 & |g(x)| > n.
                \end{cases}  
            \]
        Now define \( \displaystyle f_n(x) = \abs{g_n}^{q/p} \cdot (\mathrm{sgn} \ g_n) \).
        Now,
            \[
                \norm{f_n}_{p} = \left( \norm{g_n}_{q} \right)^{q/p}  
            \]
        and also 
            \[  
                \abs{g_n}^q = f_n \cdot g_n = f_n \cdot g.
            \]
    Hence,
            \begin{align*}
                \left( \norm{g_n}_q \right)^{q} = \int f_n \cdot g \leq M \norm{f_n}_{P}  = M \left( \norm{g_n}_q \right)^{q/p}.
            \end{align*}
    Because \( \displaystyle q - \frac{p}{q} = 1 \), then for all \( n \in N \),
            \[
                \norm{g_n}_q \leq M. 
            \]
    Note that as \( n \to \infty \), \( |g_n|^q \to |g|^q \) almost everywhere and so the above expression implies by Fatuo's lemma 
            \[
                \int \abs{g_n}^q \leq \lim_{n \to \infty}  \int \abs{g_n}^q \leq M^q.
            \]
    \end{proof}
    
\end{lemma}

\begin{thm}[\textbf{6.13, Riesz Representation Theorem}]

    Let \( F \) be a bounded linear functional on \( L^p \). 
    Then there exists \( g \in L^q \) such that 
        \[  
            F(f) = \int fg. 
        \]
    We also have \( \norm{F} = \norm{g}_q \).

        \begin{proof}
            Our outline of the proof will be as follows:
                \begin{enumerate}[label = Step \Roman{*}:]
                    \item Find \( g \) which is integrable.
                    \item Upgrade \( g \) to \( L^q \). 
                    \item Verify for all \( f \in L^ p\).
                \end{enumerate}
            Let \( \chi_s = 1_{[0,s]} \) and \( \Phi(s) = F(\chi_x) \).
            We claim that \( \Phi(s) \) is absolutely continuous. 
            This means that there exists \( g \) integrable such that  
                \[ 
                    \Phi(s) - \underbrace{\Phi(0)}_{=0} = \int_{0}^{s} g(t) \dif t = \int_{0}^{1} g \cdot \chi_{s}.
                \]
            Let \( \epsilon > 0 \) be chosen. Let \( \{ (s_i, s'_{i}) \} \) be a disjoint collection of intervals over \( [0, 1] \) with 
                \[
                    \sum_{i=1}^{k} s'_{i} - s_{i} < \delta.    
                \]
            We want to show that this implies that
                \[  
                    \sum_{i=1}^{k} \abs{ \Phi(s'_{i}) - \Phi(s)} < \epsilon.
                \]
            Note that 
                \[
                    \sum_{i=1}^{k}  \abs{ \Phi(s'_{i}) - \Phi(s)} = F(f)
                \]
            where
                \begin{align*}
                    f &= \sum_{i=1}^{k} \left( \chi_{s'_{i}} - \chi_{s_{i}}\right) \cdot \sgn\left( \Phi(s'_{i}) - \Phi(s)  \right) \\
                    &\leq \sum_{i=1}^{k} \left( s'_{i} - s_{i} \right)  \\
                    &< \delta
                \end{align*}
            and so this implies that \( \norm{f}_{p} < \delta \).
            By definition of \( \Phi \) and \( F \) being a bounded linear functional, we know that 
                \begin{align*}
                    \sum_{i=1}^{k} \abs{ \Phi(s'_{i}) - \Phi(s)} &= F(f) \\
                    &\leq M \cdot \norm{f}_{p} \\
                    &< M \cdot \delta \\
                    &< \epsilon.
                \end{align*}
            Thus \( \Phi \) is absolutely continuous if we choose \( \displaystyle \delta = \frac{\epsilon}{M} \). So this tell us
                \[
                    \Phi(s) = \int_{0}^{1} g \cdot \chi_{s}  
                \]
            for some integrable function \( g \). Since every step function \( \phi \) on \( [0,1] \) is a linear combination of \( \phi'_{s} \)'s, we know that 
                \[
                    F(\phi) = \int_{0}^{1} g \phi    
                \]
            for every step function \( \phi \).
            
            Now we claim that this function \( g \) is in \( L^q \). 
            To that end, let \( f \) be a bounded measurable function on \( [0,1] \).
            Then there exists bounded step functions \( \phi_n \) such that \( \phi_n \to f \) almost everywhere. 
            Thus the sequence \( \displaystyle \left\{ \abs{f - \phi_n}^p \right\} \) is uniformly bounded and converges to \( 0 \) almost everywhere.
            This tells us that \( \norm{f - \phi_n}_{p} \to 0 \).
            Since \( F \) is a bounded linear functional, we know that
                \begin{align*}
                    \abs{F(f) - F(\phi_n)} &= \abs{F(f - \phi_n)} \\
                    &\leq M \cdot \norm{f - \phi_n}_{p},
                \end{align*}
            and so we know that \( \displaystyle F(f) = \lim_{n \to \infty} F(\phi_n) \). Letting \( \tilde{M} \) be the uniform bound of the \( \phi_n \)'s, we have \( g \phi_n < \abs{g} \cdot \tilde{M} \).
            By the Lebesgue dominated convergence theorem,
                \begin{align*}
                    \int f \cdot g &= \lim_{n \to \infty} \int g \cdot \phi_n \\
                    &= \lim_{n \to \infty} F(\phi_n) \\
                    &= F(f)
                \end{align*}
            for all \( f \) which are bounded measurable functions. Thus, by the H\"{o}lder inequality,
                \[
                    \abs{\int fg} =  \abs{F(f)} \leq \norm{F} \cdot \norm{f}_p.
                \]
            By Lemma 6.12, \( g \in L^q \), finishing our second claim. 

            Finally, we claim that 
                \[
                    F(f) = \int g \cdot f  
                \]
            for all \( f \in L^p \).

            Let \( f \in L^p \) be any function and fix \( \epsilon > 0 \). 
            By the density of step functions in \( L^p \), there exists a step function \( \phi \) such that 
                \[
                    \norm{f - \phi}_{p} < \epsilon  
                \]
            and so
                \[
                    F(\phi) = \int \phi g.  
                \]
            Hence, we have by the linearity of \( F \),
                \begin{align*}
                    \abs{F(f) - \int f\cdot g} &= \abs{F(f) - F(\phi) + F(\phi) - \int f \cdot g} \\
                    &= \abs{F(f - \phi)+ \int g(\phi - f)} \\
                    &\leq \abs{F(f - \phi)} + \abs{\int g(\phi - f)} \\
                    &\leq M \cdot \norm{f - \phi}_{p} + \abs{\int g(\phi - f)}  \\
                    &\leq  M \cdot \underbrace{\norm{f - \phi}_{p}}_{<\epsilon} + \norm{g}_{q} \cdot \underbrace{\norm{f - \phi}_{p}}_{<\epsilon} && \text{H\"{o}lder Inequality} \\
                \end{align*}
            This tells us that 
                \[
                    F(f) = \int fg.
                \]
        \end{proof}

\end{thm}

\section*{Chapter 11 Measure Spaces}

The set up will be that we have any set \( X \) and \( \mathcal{B} \) which is a \(\sigma\)-algebra i.e., a collection of subsets of \( X \) such that 
    \begin{enumerate}
        \item \(\emptyset \in \mathcal{B} \)
        \item If \( A \in \mathcal{B} \), then \( A^{\C} \in \mathcal{B}\).
        \item Closed under countable union. 
    \end{enumerate}
Define \( \mu: \B \to \overline{\R} = \R \cup \{ \pm \infty\} \) to be a set function such that \( E \mapsto \mu(E) \in \overline{\R}\).

\begin{definition}
    A couple \( (X, \B) \) is called a measurable space consisting of a set \(X\) and a \(\sigma\)-algebra \( \B \) subsets of \( X \). A set \( E \subset X \) is called \textbf{measurable} if \( E \in \B \).
\end{definition}

\begin{definition}
    A set function \( \mu: \B \to \overline{\R} \) is called a \textbf{measure} if we have the following:
        \begin{enumerate}
            \item \( \mu(\emptyset) = 0 \) 
            \item Countable additivity i.e., \( \displaystyle \mu \left( \bigcup_{i=1}^{\infty} E_i \right) = \sum_{i-1}^{\infty} \mu \left( E_i \right)\) 
            for any \( E_i \cap E_j = 0 \) with \( i \neq j \).
        \end{enumerate}
    We then call the triple \( (X, \B, \mu) \) a  \textbf{measure space}.
\end{definition}

\begin{remark}
    We know that countable additivity implies finite additivity. An example of this is the triple \( (\R, \mathcal{M}, m) \) where \(\mathcal{M} \) is all Lebesgue measurable sets and \( m \) is the Lebesgue measure. 
\end{remark}

\end{document}