\documentclass[12pt]{article}
\usepackage{amsmath,fullpage,graphicx,fancyhdr,enumerate,amsthm,amssymb,tikz,tikz-cd,color,pgfplots,tabularx,amsfonts,amscd}
\usepackage{enumitem}
\usetikzlibrary{arrows,chains,matrix,positioning,scopes}
\pgfplotsset{compat=1.13}
\setlength{\headheight}{15pt}
\setlength{\headsep}{7pt}
\newcommand{\bsnl}{\bigskip\newline}
\pagestyle{fancy}
\usepackage[paper=a4paper, margin = 1in]{geometry}
\usepackage{hyperref}% http://ctan.org/pkg/hyperref
\usepackage[capitalise, noabbrev]{cleveref}

\hypersetup{ % colors hyperlinks with colors to make noticeable but is not an ugly green box like default
    colorlinks,
    linkcolor={red!50!black},
    citecolor={blue!50!black},
    urlcolor={blue!80!black}
}
\usepackage[mathscr]{euscript}\usepackage{pifont}
\usepackage{cleveref}
\usepackage{parskip}
\usepackage[makeroom]{cancel}


\newcommand{\cmark}{\ding{51}}%
\newcommand{\xmark}{\ding{55}}%

\renewcommand{\aa}{\mathbf{a}}
\newcommand{\bb}{\mathbf{b}}
\newcommand{\cc}{\mathbf{c}}
\newcommand{\dd}{\mathbf{d}}
\newcommand{\FF}{\mathbf{F}}
\newcommand{\ii}{\mathbf{i}}
\newcommand{\jj}{\mathbf{j}}
\newcommand{\rr}{\mathbf{r}}
\newcommand{\kk}{\mathbf{k}}
\newcommand{\uu}{\mathbf{u}}
\newcommand{\vv}{\mathbf{v}}
\newcommand{\ww}{\mathbf{w}}
\newcommand{\yy}{\mathbf{y}}
\newcommand{\R}{\mathbb{R}}
\newcommand{\T}{\mathcal{T}}
\newcommand{\N}{\mathbb{N}}
\newcommand{\C}{\mathscr{C}}
\newcommand{\Z}{\mathbb{Z}}
\newcommand{\B}{\mathcal{B}}
\newcommand{\Q}{\mathbb{Q}}
\newcommand{\Sph}{\mathbb{S}}
\newcommand{\D}{\mathbb{D}}

\def\dim{\mathop{\rm dim}\nolimits}
\def\image{\mathop{\rm Im}\nolimits}  
\def\interior{\mathop{\rm Int}\nolimits}
\def\kernel{\mathop{\rm Ker}\nolimits}
\def\cokernel{\mathop{\rm Coker}\nolimits}
\def\bd{\mathop{\rm Bd}\nolimits}
\def\ext{\mathop{\rm Ext}\nolimits}
\def\num{\mathop{\#}\limits}
\def\cl{\mathop{\rm Cl}\nolimits}
\def\lub{\mathop{\rm lub}\nolimits}


\newcommand{\proj}{\operatorname{proj}}
\newcommand{\ds}{\displaystyle}
\newcommand{\pa}{\partial}
\newcommand{\ol}{\overline{}}


\newcommand{\degree}{^{\circ}}
\renewcommand{\epsilon}{\varepsilon}

\pagestyle{fancy}
\lhead{Math 6421: Measure Theory}
\chead{\bf Jose Nino: Lecture \#5}
\rhead{September 7, 2023}
\cfoot{Page \thepage}
% \setlength{\headheight}{10pt}

\theoremstyle{definition}
\newtheorem*{thm}{Theorem}
\newtheorem*{exercise}{Exercise}
\newtheorem*{definition}{Definition}
\newtheorem{problem}{Problem}
\newtheorem*{lemma}{Lemma}
\newtheorem*{cor}{Corollary}
\newtheorem*{prop}{Proposition}


\begin{document}

\begin{lemma}[\textbf{3.9}]
    Let \( A \) be any set, and \( E_1, \ldots, E_n \) be a finite sequence of sets such that \( E_i \cap E_j \) for all \( i \neq j \). Then
        \[
            m^{*} \left( A \cap \left[ \bigcup_{i=1}^{n} E_i \right]\right)  = \sum_{i=1}^{n} m^{*}(A \cap E_i).
        \]

        \begin{proof}
            We proceed by induction. For \( n = 1 \), we have the set \( E_1 \) and the equality holds. Suppose that we have \( n = k \) sets \( E_1, \ldots, E_k \) with \( E_i \cap E_j \neq \emptyset \) for all \( i \neq j \) so that
                \[
                    m^{*} \left( A \cap \left[ \bigcup_{i=1}^{k} E_i \right]\right)  = \sum_{i=1}^{k} m^{*}(A \cap E_i).
                \]
            Consider \( n =  k + 1 \). Because each \( E_i \) is disjoint, 
                \begin{align*}
                    A \cap \left( \bigcup_{i=1}^{k+1} E_i \right) \cap E_{k+1}  &= A \cap E_{k+1}; \\
                    A \cap \left( \bigcup_{i=1}^{k+1} E_i \right) \cap E^{\C}_{k+1} &= A \cap \bigcup_{i=1}^{k} E_i.  
                \end{align*}
            Because the \( E_i \)'s are measurable,
                \begin{align*}
                    m^{*} \left( A \cap \bigcup_{i=1}^{k+1} E_i \right) &= m^{*} \left( A \cap E_{k+1} \right) + m^{*} \left(A \cap \bigcup_{i=1}^{k} E_i \right) \\
                    &= m^{*} \left( A \cap E_{k+1} \right) + \sum_{i=1}^{k} m^{*}(A \cap E_i) && \text{Induction Hypothesis} \\
                    &= \sum_{i=1}^{k+1} m^{*}(A \cap E_i)
                \end{align*}
            which, by induction, completes the proof. 
        \end{proof}
\end{lemma}

\begin{thm}[\textbf{3.10}]
    
    \( \mathcal{M} \) is a \( \sigma \)-algebra. In other words, in addition to being an algebra of sets, if \( \{ E_{i} \}_{i=1}^{\infty} \subset \mathcal{M}\), then 
    \( \displaystyle \bigcup_{i=1}^{\infty} E_i \in \mathcal{M} \).

        \begin{proof}\footnote{Proof on bottom of page 59 and top of page 60.}


        \end{proof}
    
\end{thm}

\begin{lemma}[\textbf{3.11}]

    The interval \( (a, \infty) \) is measurable for all \( a \in \R \).

        \begin{proof}\footnote{Proof on the bottom of page 60 through the middle of page 61.}
            
        \end{proof}
    
\end{lemma}

\begin{thm}[\textbf{3.12}]

    Every Borel set is measurable. In particular, each open set and each closed set is measurable.

        \begin{proof}\footnote{Proof on the bottom of page 61.} 
            
        \end{proof}

    
\end{thm}

\begin{definition}
    Let \( E \in \mathcal{M} \). We define \( m(E) := m^{*}(E) \) to be the \textbf{Lebesgue measure} of \( E \)/
\end{definition}

\begin{prop}[\textbf{3.13, Countable Additivity}]

    Let \( \{ E_i\}_{i=1}^{n} \) be a sequence of measurable sets. Then 
        \[
            m \left( \bigcup_{i=1}^{\infty} E_i \right) \leq \sum_{i=1}^{n} m(E_i).    
        \]
    If, in addition, \( E_i \cap E_j \) for all \( i \neq j \). then 
        \[
            m \left( \bigcup_{i=1}^{\infty} E_i \right) = \sum_{i=1}^{n} m(E_i).
        \]

    
\end{prop}

\begin{prop}[\textbf{3.14}]

    Let \( \{E_i\} \subset \mathcal{M} \) be a decreasing sequence (i.e., \( E_{i+1} \subset E_i \)). Let \( m(E_1) < \infty \). Then
        \[
            m\left( \bigcap_{n=1}^{\infty} E_n \right) = \lim_{n \to \infty} m(E_n).
        \]
    
\end{prop}

\begin{prop}[\textbf{3.15}]

    
    Let \( E \) be any given set. Then the following are equivalent:
        \begin{enumerate}[label = (\roman{*})]
            \item \( E \) is measurable.
            \item For all \( \epsilon > 0 \), there is an open set \( O \supset E \) with \( m^{*}(O \setminus E) < \epsilon \).
            \item For all \( \epsilon > 0 \), there is a closed set \( F \subset E \) with \( m^{*}(E \setminus F) < \epsilon \).
            \item There is a \( G \in G_{\delta} \) with \( E \subset G \) such that \( m^{*}(G \setminus E) = 0 \). 
            \item There is a \( F \in F_{\sigma} \) with \( F \subset E \) such that \( m^{*}(E \setminus F) = 0 \). 
            
            \noindent If \( m^{*}(E) < \infty \), the above statements are equivalent:

            \item For all \( \epsilon > 0 \), there is a finite union \( U \) of open intervals such that \( m^{*}(U \Delta E) < \epsilon \).
        \end{enumerate}
    
\end{prop}

\end{document}

\section*{Section 3.5, Measurable Functions}