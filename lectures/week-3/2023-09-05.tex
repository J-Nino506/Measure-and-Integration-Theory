\documentclass[12pt]{article}
\usepackage{amsmath,fullpage,graphicx,fancyhdr,enumerate,amsthm,amssymb,tikz,tikz-cd,color,pgfplots,tabularx,amsfonts,amscd}
\usepackage{enumitem}
\usetikzlibrary{arrows,chains,matrix,positioning,scopes}
\pgfplotsset{compat=1.13}
\setlength{\headheight}{15pt}
\setlength{\headsep}{7pt}
\newcommand{\bsnl}{\bigskip\newline}
\pagestyle{fancy}
\usepackage[paper=a4paper, margin = 1in]{geometry}
\usepackage{hyperref}% http://ctan.org/pkg/hyperref
\usepackage[capitalise, noabbrev]{cleveref}

\hypersetup{ % colors hyperlinks with colors to make noticeable but is not an ugly green box like default
    colorlinks,
    linkcolor={red!50!black},
    citecolor={blue!50!black},
    urlcolor={blue!80!black}
}
\usepackage[mathscr]{euscript}\usepackage{pifont}
\usepackage{cleveref}
\usepackage{parskip}
\usepackage[makeroom]{cancel}


\newcommand{\cmark}{\ding{51}}%
\newcommand{\xmark}{\ding{55}}%

\renewcommand{\aa}{\mathbf{a}}
\newcommand{\bb}{\mathbf{b}}
\newcommand{\cc}{\mathbf{c}}
\newcommand{\dd}{\mathbf{d}}
\newcommand{\FF}{\mathbf{F}}
\newcommand{\ii}{\mathbf{i}}
\newcommand{\jj}{\mathbf{j}}
\newcommand{\rr}{\mathbf{r}}
\newcommand{\kk}{\mathbf{k}}
\newcommand{\uu}{\mathbf{u}}
\newcommand{\vv}{\mathbf{v}}
\newcommand{\ww}{\mathbf{w}}
\newcommand{\yy}{\mathbf{y}}
\newcommand{\R}{\mathbb{R}}
\newcommand{\T}{\mathcal{T}}
\newcommand{\N}{\mathbb{N}}
\newcommand{\C}{\mathscr{C}}
\newcommand{\Z}{\mathbb{Z}}
\newcommand{\B}{\mathcal{B}}
\newcommand{\Q}{\mathbb{Q}}
\newcommand{\Sph}{\mathbb{S}}
\newcommand{\D}{\mathbb{D}}

\def\dim{\mathop{\rm dim}\nolimits}
\def\image{\mathop{\rm Im}\nolimits}  
\def\interior{\mathop{\rm Int}\nolimits}
\def\kernel{\mathop{\rm Ker}\nolimits}
\def\cokernel{\mathop{\rm Coker}\nolimits}
\def\bd{\mathop{\rm Bd}\nolimits}
\def\ext{\mathop{\rm Ext}\nolimits}
\def\num{\mathop{\#}\limits}
\def\cl{\mathop{\rm Cl}\nolimits}
\def\lub{\mathop{\rm lub}\nolimits}


\newcommand{\proj}{\operatorname{proj}}
\newcommand{\ds}{\displaystyle}
\newcommand{\pa}{\partial}
\newcommand{\ol}{\overline{}}


\newcommand{\degree}{^{\circ}}
\renewcommand{\epsilon}{\varepsilon}

\pagestyle{fancy}
\lhead{Math 6421: Measure Theory}
\chead{\bf Jose Nino: Lecture \#4}
\rhead{September 5, 2023}
\cfoot{Page \thepage}
% \setlength{\headheight}{10pt}

\theoremstyle{definition}
\newtheorem*{thm}{Theorem}
\newtheorem*{exercise}{Exercise}
\newtheorem*{definition}{Definition}
\newtheorem{problem}{Problem}
\newtheorem*{lemma}{Lemma}
\newtheorem*{cor}{Corollary}
\newtheorem*{prop}{Proposition}


\begin{document}

\section*{Section 3.2, Outer Measure}

\begin{definition}

    The \textbf{outer measure} \( m^{*}(A) \) of a set \( A \subset \R \) is given
        \[
            m^{*}(A) = \inf_{A \subset \bigcup I_{n}} \sum_{n} l(I_n)  
        \]
    where \( \{I_n\} \) is a countable collection of open intervals that cover \( A \).
    
\end{definition}

Note that from this definition, we get that 
    \begin{enumerate}
        \item \( m^{*}(\emptyset) =  0 \)
        \item If \( A \subset B \), \( m^{*}(A) < m^{*}(B) \).
        \item \( m^{*} \) does not satisfy disjoint additivity. 
    \end{enumerate}
    
\begin{prop}[\textbf{3.1}]

    The outer measure of an interval is its length; that is, \( m^{*}(I) = l(I) \) where \( I = [a, b], (a, b), [a, b),\) or \( (a,b] \).

    \begin{proof}

        It is sufficient to show that \( m^{*}([a,b]) = l([a,b]) \) since every other interval is a subset of \( [a, b] \). Let \( \epsilon > 0 \). Then \( [a, b] \subset \left[a - \frac{\epsilon}{2}, b + \frac{\epsilon}{2}\right]   \) which implies, by the definition of the outer measure,
            \[
                m^{*}([a, b]) \leq l \left( \left[ a - \frac{\epsilon}{2}, b + \frac{\epsilon}{2} \right]\right) = b - a + \epsilon.  
            \]
        Because \( \epsilon \) was fixed, this means that \( m^{*} \leq b - a \). 
        
        Now we must show that \( m^{*} \geq b - a \). Because \( [a, b] \) is compact, for any collection \( \{I_n\} \) of open intervals covering \( [a,b] \), there exists a finite collection of intervals \( \{ I_1, \ldots, I_k \} \) so that 
            \[
                [a, b] \subset \bigcup_{n=1}^{k} I_n.    
            \]
        This gives us that
            \[
                \sum_{n} I_n \geq \sum_{n=1}^{k} I_n \geq b - a
            \]
        and so \( b - a \) is a lower bound. But since \( m^{*} \) as the greatest lower bound of all such sums, we have that \( m^{*} \geq b - a \).

        Therefore, \( m^{*}([a, b]) = l([a,b]) = b - a \).
        
    \end{proof}

\end{prop}

\begin{prop}[\textbf{3.2, Subadditivity}]

    Let \( \{ A_n \} \) be a countable collection of sets on \( \R \). Then 
        \[
            m^{*} \left( \bigcup_{n} A_n \right) \leq \sum_{n} m^{*}(A_n).  
        \]

    \begin{proof}
        Proof on page 57. 

    \end{proof}

    
\end{prop}

\begin{cor}[\textbf{3.3}]

    If \( A \) is a countable set, then \( m^{*}(A) = 0 \).

    \begin{proof}

        Proof is on the end of page 57. 

    \end{proof}
\end{cor}


\section*{Section 3.2, Measurable Sets and Lebesgue Measure}

\begin{definition}
    A set \( E \subset \R \) is (Lebesgue) \textbf{measurable} if for all sets \( A \), we have that
        \[
            m^{*}(A) = m^{*}(A \cup E) + m^{*}(A \cup E^{\C}).  
        \]
\end{definition}

\begin{lemma}[\textbf{3.6}]
    
    If \( m^{*}(E) = 0 \), then \( E \) is measurable. 

        \begin{proof}
            Let \( A \) be any chosen set. Because \( A \cap E \subset E \) and \( m^{*}(E) = 0 \),
                \[
                    m^{*}(A \cap E) \leq m(E) = 0.  
                \]
            Note that \( A \cap E^{\C} \subset A \) and so \( m^{*}(A) \leq m^{*}(A \cap E^{\C}) \) and so it suffices to show that
            \( m^{*}(A) \geq m^{*}(A \cap E^{\C}) \). Using this, we can show that
                \begin{align*}
                    m^{*}(A) \geq m^{*}(A \cap E^{\C}) + 0 = m^{*}(A \cap E^{\C}) = m^{*}(A \cap E)
                \end{align*}
            giving us the desired result.
        \end{proof}

\end{lemma}

\begin{definition}
    Let \( \mathcal{M} \) be the set of measurable sets in \( \R \) 
\end{definition}

\begin{lemma}[\textbf{3.7}]

    If \( E_1 \) and \( E_2 \) are measurable sets, then so is \( E_1 \cup E_2 \).

        \begin{proof}
            Proof on top of page 57.
        \end{proof}
    
\end{lemma}

\begin{cor}[\textbf{3.8}]

    The family \( \mathcal{M} \) of measurable sets if an algebra of sets. In other words, if \( E \in \mathcal{M} \), then \( E^{\C} \in \mathcal{M} \). Further, if \( E_1, E_2 \in \mathcal{M} \), then \( E_1 \cup E_2 \in \mathcal{M} \).
    
\end{cor}


\end{document}