\documentclass[12pt]{article}
\usepackage{amsmath,fullpage,graphicx,fancyhdr,enumerate,amsthm,amssymb,tikz,tikz-cd,color,pgfplots,tabularx,amsfonts,amscd}
\usepackage{enumitem}
\usetikzlibrary{arrows,chains,matrix,positioning,scopes}
\pgfplotsset{compat=1.13}
\setlength{\headheight}{15pt}
\setlength{\headsep}{7pt}
\newcommand{\bsnl}{\bigskip\newline}
\pagestyle{fancy}
\usepackage[paper=a4paper, margin = 1in]{geometry}
\usepackage{hyperref}% http://ctan.org/pkg/hyperref
\usepackage[capitalise, noabbrev]{cleveref}

\hypersetup{ % colors hyperlinks with colors to make noticeable but is not an ugly green box like default
    colorlinks,
    linkcolor={red!50!black},
    citecolor={blue!50!black},
    urlcolor={blue!80!black}
}
\usepackage[mathscr]{euscript}\usepackage{pifont}
\usepackage{cleveref}
\usepackage{parskip}
\usepackage[makeroom]{cancel}


\newcommand{\cmark}{\ding{51}}%
\newcommand{\xmark}{\ding{55}}%

\renewcommand{\aa}{\mathbf{a}}
\newcommand{\bb}{\mathbf{b}}
\newcommand{\cc}{\mathbf{c}}
\newcommand{\dd}{\mathbf{d}}
\newcommand{\FF}{\mathbf{F}}
\newcommand{\ii}{\mathbf{i}}
\newcommand{\jj}{\mathbf{j}}
\newcommand{\rr}{\mathbf{r}}
\newcommand{\kk}{\mathbf{k}}
\newcommand{\uu}{\mathbf{u}}
\newcommand{\vv}{\mathbf{v}}
\newcommand{\ww}{\mathbf{w}}
\newcommand{\yy}{\mathbf{y}}
\newcommand{\R}{\mathbb{R}}
\newcommand{\T}{\mathcal{T}}
\newcommand{\N}{\mathbb{N}}
\newcommand{\C}{\mathscr{C}}
\newcommand{\Z}{\mathbb{Z}}
\newcommand{\B}{\mathcal{B}}
\newcommand{\Q}{\mathbb{Q}}
\newcommand{\Sph}{\mathbb{S}}
\newcommand{\D}{\mathbb{D}}

\def\dim{\mathop{\rm dim}\nolimits}
\def\image{\mathop{\rm Im}\nolimits}  
\def\interior{\mathop{\rm Int}\nolimits}
\def\kernel{\mathop{\rm Ker}\nolimits}
\def\cokernel{\mathop{\rm Coker}\nolimits}
\def\bd{\mathop{\rm Bd}\nolimits}
\def\ext{\mathop{\rm Ext}\nolimits}
\def\num{\mathop{\#}\limits}
\def\cl{\mathop{\rm Cl}\nolimits}
\def\lub{\mathop{\rm lub}\nolimits}


\newcommand{\proj}{\operatorname{proj}}
\newcommand{\ds}{\displaystyle}
\newcommand{\pa}{\partial}
\newcommand{\ol}{\overline{}}


\newcommand{\degree}{^{\circ}}
\renewcommand{\epsilon}{\varepsilon}

\pagestyle{fancy}
\lhead{Math 6421: Measure Theory}
\chead{\bf Jose Nino: Lecture \#3}
\rhead{August 31, 2023}
\cfoot{Page \thepage}
% \setlength{\headheight}{10pt}

\theoremstyle{definition}
\newtheorem*{thm}{Theorem}
\newtheorem*{exercise}{Exercise}
\newtheorem*{definition}{Definition}
\newtheorem{problem}{Problem}
\newtheorem*{lemma}{Lemma}
\newtheorem*{cor}{Corollary}
\newtheorem*{prop}{Proposition}

\begin{document}

\section*{Compactness}

\begin{thm}
    Let \(E \subset \R \). Then \( E \) is compact if and only \( E\) is sequentially compact. That is, for every \( \{ x_n \} \) in \( E \), there exists a convergent subsequence \( x_{n_{m} } \to x_0 \) in E.
\end{thm}


\begin{thm}

        Let \( \{ I_n \} \) be a sequence of closed intervals such that \( I_{n+1} \subset I_n \). Then 
            \[
                \bigcap_{n=1}^{\infty} I_n \neq \emptyset.  
            \]
        If \( [a_n, b_n] \) is an interval and \( \displaystyle\lim_{n \to \infty} a_n = \lim_{n \to \infty} = a \), then \( \displaystyle \bigcap_{n=1}^{\infty} \).
\end{thm}

\section*{Section 2.6, Continuous Functions}

\begin{definition}
    Let \( E \subset \R \), and let \( f: E \to \R \) be a real-valued function. Then \( f \) is \textbf{continuous} at the point \( x = a \in E\) if for all \( \epsilon > 0 \), there exists \( \delta > 0 \) such that for all \( y \in E \) with \( |x - y| < \delta \) implies that \( |f(x) - f(y)| < \epsilon \).
\end{definition}

Note that we can have continuity in terms of sequences. I will state it as a theorem here even though it was not in lecture because it is important to be able to use on its own.

\begin{thm}
    Let \( f: E \to \R \) be a function with \( E \subset \R \). Let \( x \in E \) be any point. Then \( f \) is continuous at \( a \) if and only for every sequence \( \{x_n\} \) in E converging to \( a\), the sequence \( \{ f(x_n) \} \) in \( f(E) \) (the image of \( E\)) converges to \( f(a) \).
\end{thm}

\begin{prop}
    Let \( E \subset \R \) be compact. Let \( f: E \to \R \) be continuous real-valued function. Then \( f(E) \) is a compact set. 

    \begin{proof}
        Let \( \subset \R \) be a compact and suppose the function \( f: E \to \R \) is continuous. To show that \( f(E) \) is compact, we will use the Heine-Borel theorem and show that it is closed and bounded. To show that \( f(E) \) is closed, suppose we have any sequence \( \{ f(x_n) \} \) converging to the point \( f(a) \in \R \). Additionally, let \( \{ x_n \} \) be any sequence in \( E \). Because \( E \) is compact, there exists a subsequence \( \{ x_{n_{m}} \}\) which converges to a point \( x_0 \in E \). Since \( f \) is continuous, by the preceding theorem this means that the sequence \( \{ f(x_{n_{m}} )\} \) converges to \( f(x_{0}) \in f(E) \).
    \end{proof}
\end{prop}

\begin{prop}[\textbf{2.17, Extreme Value Theorem}]

    Let \( E \subset \R \) be a compact set, and let \( f: E \to \R \) be a continuous function. Then there exists \( x_1, x_2 \in E \) such that
        \[
            f(x_1) \leq f(x) \leq (x_2), \ \text{for all x} \in E.  
        \] 
    
\end{prop}

\begin{prop}[\textbf{2.18}]

    Let \( f: \R \to \R \) be a function. Then \( f \) is \textbf{continuous} if and only if \( f^{-1}(O) \) is open for all open sets \( O \subset \R \).
    
\end{prop}

\begin{prop}[\textbf{2.19}]

    Let \( E \subset \R \), and let \( f: E \to \R \) be continuous. Without loss of generality, suppose that \( f(a) \leq f(b) \). Then for all \( \gamma \in [f(a), f(b)] \), there exists \( c \in [a, b] \) such that \( f(c) = \gamma \).
    
\end{prop}

\begin{definition}[\textbf{Uniform Continuity}]

    Let \( E \subset \R \). A function \( f: E \to \R \) is \textbf{uniformly continuous} if for all \( \epsilon > 0 \), there exists \( \delta > 0 \) such that for all \( x, y \in E \) with \( |x - y| < \delta \) implies that \( |f(x) - f(y)| < \epsilon \).
    
\end{definition}

\begin{prop}[\textbf{2.20}]

    Let \( E \subset \R \) be a compact set. If \( f: E \to \R \) is a continuous function on \( E \), then \( f \) is uniformly continuous on \( E \).
    
\end{prop}

\begin{definition}
    Let \( f_n: E \to \R \) be a sequence of functions, and let \( f: E \to \R \).

        \begin{enumerate}
            \item The sequence \( \{ f_n \} \) \textbf{converges pointwise} on \( E \) to \( f \) if for all \( \epsilon > 0 \) and for all \( x \in E \), there exists \( N \in \N \) such that for all \( n \geq  N\), \( |f(x) - f_n(x)| < \epsilon \).
            \item The sequence \( \{ f_n \} \) \textbf{converges uniformly} if for all \( \epsilon > 0 \), there exists \( N \in \N \) such that for all \( x \in E \) and for all \( n \geq N\), \( |f(x) - f_n(z)| < \epsilon \).
        \end{enumerate}
\end{definition}

\section*{Section 3.1, Lebesgue Measure}


[Perhaps finish these notes another time...]



\end{document}