\documentclass[12pt]{article}
\usepackage{amsmath,fullpage,graphicx,fancyhdr,enumerate,amsthm,amssymb,tikz,tikz-cd,color,pgfplots,tabularx,amsfonts,amscd}
\usepackage{enumitem}
\usetikzlibrary{arrows,chains,matrix,positioning,scopes}
\pgfplotsset{compat=1.13}
\setlength{\headheight}{15pt}
\setlength{\headsep}{7pt}
\newcommand{\bsnl}{\bigskip\newline}
\pagestyle{fancy}
\usepackage[paper=a4paper, margin = 1in]{geometry}
\usepackage{hyperref}% http://ctan.org/pkg/hyperref
\usepackage[capitalise, noabbrev]{cleveref}

\hypersetup{ % colors hyperlinks with colors to make noticeable but is not an ugly green box like default
    colorlinks,
    linkcolor={red!50!black},
    citecolor={blue!50!black},
    urlcolor={blue!80!black}
}
\usepackage[mathscr]{euscript}\usepackage{pifont}
\usepackage{cleveref}
\usepackage{parskip}
\usepackage[makeroom]{cancel}


\newcommand{\cmark}{\ding{51}}%
\newcommand{\xmark}{\ding{55}}%

\renewcommand{\aa}{\mathbf{a}}
\newcommand{\bb}{\mathbf{b}}
\newcommand{\cc}{\mathbf{c}}
\newcommand{\dd}{\mathbf{d}}
\newcommand{\FF}{\mathbf{F}}
\newcommand{\ii}{\mathbf{i}}
\newcommand{\jj}{\mathbf{j}}
\newcommand{\rr}{\mathbf{r}}
\newcommand{\kk}{\mathbf{k}}
\newcommand{\uu}{\mathbf{u}}
\newcommand{\vv}{\mathbf{v}}
\newcommand{\ww}{\mathbf{w}}
\newcommand{\yy}{\mathbf{y}}
\newcommand{\R}{\mathbb{R}}
\newcommand{\T}{\mathcal{T}}
\newcommand{\N}{\mathbb{N}}
\newcommand{\C}{\mathscr{C}}
\newcommand{\Z}{\mathbb{Z}}
\newcommand{\B}{\mathcal{B}}
\newcommand{\Q}{\mathbb{Q}}
\newcommand{\Sph}{\mathbb{S}}
\newcommand{\D}{\mathbb{D}}

\def\dim{\mathop{\rm dim}\nolimits}
\def\image{\mathop{\rm Im}\nolimits}  
\def\interior{\mathop{\rm Int}\nolimits}
\def\kernel{\mathop{\rm Ker}\nolimits}
\def\cokernel{\mathop{\rm Coker}\nolimits}
\def\bd{\mathop{\rm Bd}\nolimits}
\def\ext{\mathop{\rm Ext}\nolimits}
\def\num{\mathop{\#}\limits}
\def\cl{\mathop{\rm Cl}\nolimits}
\def\lub{\mathop{\rm lub}\nolimits}


\newcommand{\proj}{\operatorname{proj}}
\newcommand{\ds}{\displaystyle}
\newcommand{\pa}{\partial}
\newcommand{\ol}{\overline{}}


\newcommand{\degree}{^{\circ}}
\renewcommand{\epsilon}{\varepsilon}

\pagestyle{fancy}
\lhead{Math 6421: Measure Theory}
\chead{\bf Jose Nino: Lecture \#2}
\rhead{August 29, 2023}
\cfoot{Page \thepage}
% \setlength{\headheight}{10pt}

\theoremstyle{definition}
\newtheorem*{thm}{Theorem}
\newtheorem*{exercise}{Exercise}
\newtheorem*{definition}{Definition}
\newtheorem{problem}{Problem}
\newtheorem*{lemma}{Lemma}
\newtheorem*{cor}{Corollary}
\newtheorem{prop}{Proposition}


\begin{document}

\begin{thm}[2.3, \textbf{Axiom of Archimedes}]

    If \( x \in \R \) is any real number, then there exists \( n \in \N \) such that \( x < n \).

    \begin{proof}
        We can break this into two cases
            \begin{enumerate}
                \item Let \( x < 1 \). If so, then simply choose \( x = 1 \).
                \item Let \( x \geq 1 \). Define the set \( S = \{ n \in N : n \leq x \} \). Then since this set is bounded above, by the Completeness Axiom, \( \sup S = y \) exists. Because \( x \) is an upper bound \( S \), by definition of the supremum, we have that \( y \leq x \).
                Let \( r  = \frac{1}{2} \). Then we can find \( k \in S \) such that \( y - \frac{1}{2}  < k \leq y \). But then we have that \( y <  y + \frac{1}{2} < k + 1 \leq y +1 \). Then this means \( k + 1 \not\in S \) and so \( x < k + 1\), completing this case.
            \end{enumerate}
        Having exhausted all cases, this completes the proof.
    \end{proof}
    
\end{thm}

\begin{prop}[\textbf{Well-Ordering Principle}]

    Every nonempty subset \( S \subset \N \) has a minimum. 

\end{prop}

\begin{prop}[\textbf{Density of the Rational Numbers}]

    Let \( x, y \in \R \). Then if \( x < y \), there exists \( q \in \Q \) such that \( x < \gamma < y \)
    
\end{prop}

\section*{Section 2.4, Sequences in \(\R\)}

\begin{definition}
    We define a \textbf{sequence} of real numbers to be a function that maps each each natural number \( n \) into the real number \( x \). That is, a sequence is a function \( s: \N \to A \) for \( A \subset \R \). This is written as \( \{ x_n \} \) or \( \{ x_n \}_{n=1}^{\infty} \).
\end{definition}

\begin{definition}[\textbf{Convergence of a Sequence}]

    A sequence converges to the real number \( l \in \R \) if for all \( \epsilon > 0 \), there exists \( N \in \N \) such that for all \( n \geq N \),
        \[
            |a_n - l| < \epsilon.  
        \]
    
\end{definition}

\begin{definition}[\textbf{Cauchy Sequence}]

    A sequence \( \{ x_n \} \) in \( \R \) is \textbf{Cauchy} sequence if for all \( \epsilon > 0 \), there exists \( N \in \N \) such that for all \( n, m \geq N \),
        \[
            | a_n - a_m| < \epsilon.  
        \]
    
\end{definition}

\begin{thm}
    Let \( \{x_n\} \) be a sequence in \( \R \). Then \( \{ x_n\} \) is Cauchy if and only \( \{ x_n \} \) is Cauchy.

\end{thm}

\begin{definition}
    The number \( l \in \R \) is called a \textbf{cluster point} of \(  \{x_n\} \) if there exists a subsequence \( \{ x_{n_{m}} \} \) of \( \{ x_n\} \) such that \( x_{n_{m}} \to l \).

    We can define this in another way. The number \( l \in \R \) is called \textbf{cluster point} of \( \{ x_n \} \) if for all \( \epsilon > 0 \) and for all \( N \in \N \), there exists \( n \geq N \) such that \( |x_n - l| < \epsilon \).
\end{definition}

\begin{definition}
    We define the \textbf{limit superior} of a sequence \( \{ x_n \} \) in \( \R \) to be 
        \[
            \overline{\lim}_{n \to \infty} \ x_n = \inf_{n} \sup_{k \geq n} x_k. 
        \]
    This is also denoted as \( \lim \sup \).
\end{definition}

\begin{thm}
    A number \( l \in \R \) is the \textbf{limit superior} of the sequence \( \{ x_n \} \) if and only if 
        \begin{enumerate}[label = (\roman{*})]
            \item For all \( \epsilon > 0 \), there exists \( n \in \N \) such that for all \( k \geq n\), \( x_k < l + \epsilon \)
            \item For all \( \epsilon > 0 \) and for all \( n \in \N \), there exists \( k \geq n \) such that \( x_{k} > l - \epsilon \). 
        \end{enumerate}
\end{thm}

\begin{definition}
    We define the \textbf{limit inferior} of a sequence \( \{ x_n \} \) in \( \R \) to be 
        \[
            \underline{\lim}_{n \to \infty} \ x_n = \sup_{n} \inf_{k \geq n} x_k. 
        \]
    This is also denoted as \( \lim \inf \).
\end{definition}

\begin{thm}
    A number \( l \in \R \) is the \textbf{limit inferior} of the sequence \( \{ x_n \} \) if and only if 
        \begin{enumerate}[label = (\roman{*})]
            \item For all \( \epsilon > 0 \), there exists \( n \in \N \) such that for all \( k \geq n\), \( x_k > l - \epsilon \)
            \item For all \( \epsilon > 0 \) and for all \( n \in \N \), there exists \( k \geq n \) such that \( x_{k} < l + \epsilon \). 
        \end{enumerate}
\end{thm}

\begin{prop}
    From the last two definitions, we have the following property.
    \begin{itemize}

        \item \( \overline{\lim}_{n \to \infty} \) is the largest cluster point.
        \item \( \underline{\lim}_{n \to \infty} \) is the smallest cluster point.
    \end{itemize}
    
\end{prop}

\section*{Section 2.5, Open and Closed Sets in \( \R \)}

\begin{definition}
    The set \( O \subset \R \) is called an \textbf{open} set if for all \( x \in O \), there exists \( \delta > 0 \) such that \( x - \delta, x + \delta \).

    Equivalently, \( O \) is an \textbf{open} set if for all \( x \in O \), there is a \( \delta > 0 \) such that each \( y \) with \( |x - y| < \delta \) belongs to \( O \).
\end{definition}

\begin{prop}
    From this above, we have the following properties:

    \begin{enumerate}
        \item The set \( \displaystyle \bigcup_{\alpha} O_{\alpha} \) is open. 
        \item The set \( \displaystyle \bigcup_{n=1}^{n} O_m \) is open.
    \end{enumerate}
\end{prop}

\begin{thm}[\textbf{Lindelof Theorem}]

    Every open set in \( \R \) is a disjoint union of countable union of open intervals.

        \begin{proof}
            This proof is contained on page 42 of Royden.
        \end{proof}
    
\end{thm}

\begin{definition}
    A real number \( x \in \R \) is called \textbf{point of closure} of a set \( E \subset \R \) if for every \( \delta > 0 \) there exists a \( y \in E \) such that \( |x - y| < \delta \).

    The set of points of closure of \( E \) is denoted \( \overline{E} \).
\end{definition}

\begin{prop}
    If \( A \subset B \subset \R \), then \( \overline{A} \subset \overline{B} \). Additionally, \( \overline{A \cup B } = \overline{A} \cup \overline{B} \).

        \begin{proof}
            The proof of this is on page 43 of Royden.
        \end{proof}
\end{prop}

\begin{definition}
    A set \( F \subset \R \) is called a \textbf{closed} set if \( \overline{F} = F \).
\end{definition}

Note that because \( F  \subset \overline{F} \) always, a set \( F \) is closed if \( \overline{F} \subset F \)---that is, \( F \) contains all of its points of closure.

\begin{prop}
    For any set \( E \), the set \( \overline{E} \) is closed; that is \( \overline{\overline{E}} = \overline{E} \).
\end{prop}

\begin{prop}
    Let \( E \subset \R \). Then \( E \) is open if and only if \( E^{\C} \) is closed.
\end{prop}

\begin{definition}
    We say that a collection of sets \( \C \) is a \textbf{cover} of a set \( F \) if 
        \[
            F \subset \bigcup_{O \in \C} O.  
        \]
    The collection \( \C \) is a covering of the set \( F \).
\end{definition}

\begin{thm}[\textbf{Heine-Borel}]

    Let \( E \subset \R \) be set. Then \( E \) is compact if and only if \( E \) is closed and bounded.
    
\end{thm}
\end{document}