\documentclass[12pt]{article}
\usepackage{amsmath,fullpage,graphicx,fancyhdr,enumerate,amsthm,amssymb,tikz,tikz-cd,color,pgfplots,tabularx,amsfonts,amscd}
\usepackage{enumitem}
\usetikzlibrary{arrows,chains,matrix,positioning,scopes}
\pgfplotsset{compat=1.13}
\setlength{\headheight}{15pt}
\setlength{\headsep}{7pt}
\newcommand{\bsnl}{\bigskip\newline}
\pagestyle{fancy}
\usepackage[paper=a4paper, margin = 1in]{geometry}
\usepackage{hyperref}% http://ctan.org/pkg/hyperref
\usepackage[capitalise, noabbrev]{cleveref}

\hypersetup{ % colors hyperlinks with colors to make noticeable but is not an ugly green box like default
    colorlinks,
    linkcolor={red!50!black},
    citecolor={blue!50!black},
    urlcolor={blue!80!black}
}
\usepackage[mathscr]{euscript}\usepackage{pifont}
\usepackage{cleveref}
\usepackage{parskip}
\usepackage[makeroom]{cancel}


\newcommand{\cmark}{\ding{51}}%
\newcommand{\xmark}{\ding{55}}%

\renewcommand{\aa}{\mathbf{a}}
\newcommand{\bb}{\mathbf{b}}
\newcommand{\cc}{\mathbf{c}}
\newcommand{\dd}{\mathbf{d}}
\newcommand{\FF}{\mathbf{F}}
\newcommand{\ii}{\mathbf{i}}
\newcommand{\jj}{\mathbf{j}}
\newcommand{\rr}{\mathbf{r}}
\newcommand{\kk}{\mathbf{k}}
\newcommand{\uu}{\mathbf{u}}
\newcommand{\vv}{\mathbf{v}}
\newcommand{\ww}{\mathbf{w}}
\newcommand{\yy}{\mathbf{y}}
\newcommand{\R}{\mathbb{R}}
\newcommand{\T}{\mathcal{T}}
\newcommand{\N}{\mathbb{N}}
\newcommand{\C}{\mathscr{C}}
\newcommand{\Z}{\mathbb{Z}}
\newcommand{\B}{\mathcal{B}}
\newcommand{\Q}{\mathbb{Q}}
\newcommand{\Sph}{\mathbb{S}}
\newcommand{\D}{\mathbb{D}}

\def\dim{\mathop{\rm dim}\nolimits}
\def\image{\mathop{\rm Im}\nolimits}  
\def\interior{\mathop{\rm Int}\nolimits}
\def\kernel{\mathop{\rm Ker}\nolimits}
\def\cokernel{\mathop{\rm Coker}\nolimits}
\def\bd{\mathop{\rm Bd}\nolimits}
\def\ext{\mathop{\rm Ext}\nolimits}
\def\num{\mathop{\#}\limits}
\def\cl{\mathop{\rm Cl}\nolimits}
\def\lub{\mathop{\rm lub}\nolimits}


\newcommand{\proj}{\operatorname{proj}}
\newcommand{\ds}{\displaystyle}
\newcommand{\pa}{\partial}
\newcommand{\ol}{\overline{}}
\newcommand{\toM}{\overset{m}{\to}}

\newcommand{\degree}{^{\circ}}
\renewcommand{\epsilon}{\varepsilon}

\newcommand{\upint}[2]{
  \overline{\int_{#1}^{#2}}
}
\newcommand{\lowint}[2]{
  \underline{\int_{#1}^{#2}}
}

\pagestyle{fancy}
\lhead{Math 6421: Measure Theory}
\chead{\bf Jose Nino: Lecture \#11}
\rhead{September 28, 2023}
\cfoot{Page \thepage}
% \setlength{\headheight}{10pt}

\theoremstyle{definition}
\newtheorem*{thm}{Theorem}
\newtheorem*{exercise}{Exercise}
\newtheorem*{definition}{Definition}
\newtheorem{problem}{Problem}
\newtheorem*{lemma}{Lemma}
\newtheorem*{cor}{Corollary}
\newtheorem*{prop}{Proposition}


\begin{document}

\section*{Section 4.5 Convergence in Measure}

\begin{definition}
    Let \( \{f_n\} \) be a sequence of measurable functions. 
    We say \( \{f_n\} \) \textbf{converges to} \( \displaystyle f \) \textbf{in measure}, \( f_n \overset{m}{\to} f\), if for all \( \epsilon > 0 \), there exists \( N \in \N \) such for all \( n > N \), 
        \[
            m \left\{ x: \left| f_n(x) - f(x) \right| \geq \epsilon \right\} < \epsilon.
        \]
\end{definition}

Note two things from this definition:
    \begin{enumerate}[label = (\arabic{*})]
        \item If \( f_n \to f \) pointwisely over \( E \) with \( m(E) < \infty \), then \( f \toM f \).
        \item So there exists examples with \( f_n \toM f \) but \( f_n \not\to f \).
    \end{enumerate}

\begin{prop}[\textbf{4.18}]

   Let \( \{f_n\} \) be a sequence of measurable functions. Suppose \( f_n \toM f \). Then there exists a subsequence \( \{f_{n_{k}} \} \) such that \( f_{n_{k}}  \to f \) almost everywhere.

    \begin{proof}
        Suppose \( f_n \toM f \). Then given \( \nu \in \N \), there exists \( n_{\nu} \in \N \) such that for all \( n > n_{\nu} \),
            \[
                m \left\{ x: \left| f_n(x) - f(x) \right| \geq \frac{1}{2^{\nu}}  \right\} < \frac{1}{2^{\nu}} \ \text{for all} \ \nu > k.
            \]
        Define the set 
            \[
                E_{\nu} =  \left\{ x: \left| f_n(x) - f(x) \right| \geq \frac{1}{2^{\nu}}  \right\}.
            \]
        Then if \( \displaystyle x \not\in \bigcup_{\nu = k }^{\infty} E_{\nu} \) which implies that
            \[
                |f_{\nu_{v}}(x) - f(x)| < \frac{1}{2^{\nu}} \ \text{ for all } \ \nu > k.
            \]
        Then \( f_n(x) \to f(x) \) pointwise for all \( \displaystyle x \not\in A = \bigcap_{k=1}^{\infty} \bigcup_{\nu = k }^{\infty} E_{\nu} \). Because we are taking the intersection over all \( k \),
            \[
                m(A) \leq m \left(\bigcup_{\nu = k }^{\infty} E_{\nu}  \right) \leq \sum_{\nu = k}^{\infty} m\left(E_{\nu}\right) \leq 2^{-\nu - 1}.
            \]
        Because  \( \nu \in \N \) is given, \( m(A) = 0 \) and so \( f_n(x) \to f(x) \) almost everywhere. 
    \end{proof}
\end{prop}

\begin{cor}[\textbf{4.19}]

    Let \( \{f_n\} \) be a sequence of measurable functions defined on a measurable set \( E \) of finite measure. Then \( {f_n} \toM  f \) if and only if every subsequence of \( \{f_n\} \) has a subsequence that converges almost everywhere to \( f \).
    
\end{cor}

The result aboves follows almost right away from Problem 2.12. Also note that by Problem 4.20 the following is true:

    Let \( \{f_n\} \) be a sequence of measurable functions. 
    If \( f_n \toM f \), then every subsequence \( \{x_{n_{k}}\} \toM f \).

\begin{prop}[\textbf{4.20}]

    Fatou's lemma, the Monotone Convergence Theorem, and the Lebesgue Dominated Convergence Theorem remain valid 
    if \( f_n \to f \) almost everywhere is replaced by \( f_n \toM f \).

    \begin{enumerate}[label = (\arabic{*})]
        \item \textbf{Fatou's Lemma}
            \begin{proof}
            Let \( \{f_n\} \) be a sequence of measurable functions such that \( f_n \toM f \). 
            Let us pick a subsequence \( \{x_{n_{k}} \} \) such 
                    \[
                        \int f_{n_{k}} \to \varliminf_{n \to \infty} \int f_{n}
                    \]
                which follows by the definition of the limit inferior.  
                Since \(  f_{n_{k}} \toM f \), by Problem 4.20, there exists a subsequence \( \{ f_{n_{k_{p}}}\}\) of \( \{f_{n_{k}}\}\) such that \( f_{n_{k_{p}}} \overset{p \to \infty}{\to} f \) almost everywhere by Proposition 4.18. Then by applying Fatuo's lemma,
                    \begin{align*}
                        \int f = \lim_{p \to \infty} f_{n_{k_{p}}} & \leq \varliminf_{n \to \infty} \int f_{n_{k_{p}}} \\
                        &= \varliminf_{k \to \infty} \int f_{n_{k}} \\
                        &= \varliminf_{n \to \infty} \int f_{n}
                    \end{align*}
                and so the result holds!
            \end{proof}
        \item \textbf{Lebesgue Dominated Convergence Theorem} Suppose \( |f_n| \leq g \) and \( f_n \toM f \). Then 
            \[
              \lim_{n \to \infty} \int f_n = \int f.   
            \]
        
            \begin{proof}
                We claim that to show this result, we must show that we can construct any subsequence \( \displaystyle \int f_{n_{k}} \) such that \( \displaystyle \int f_n \) which then implies that
                    \[
                        \lim_{k \to \infty} \int f_{n_{k}} = \int f.  
                    \]
                
                Because \( f_{n_{k}} \toM f\), there exists a subsequence \( \{f_{n_{k_{p}}}\} \) of \( \{f_{n_{k}}\} \) such that 
                \( f_{n_{k_{p}}} \to f \) almost everywhere. By the Lebesgue Dominated Convergence Theorem,
                    \[
                        \lim_{p \to \infty} \int f_{n_{k_{p}}} = \lim_{k \to \infty} \int  f_{n_{k}} =  \int f.
                    \]
                Then by Problem 2.12,
                    \[
                        \lim_{n \to \infty} f_n = \int f  
                    \]
                which is what we desired to show.
            \end{proof}
    \end{enumerate}
\end{prop}

\end{document}