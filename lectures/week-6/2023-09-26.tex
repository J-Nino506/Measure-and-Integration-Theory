\documentclass[12pt]{article}
\usepackage{amsmath,fullpage,graphicx,fancyhdr,enumerate,amsthm,amssymb,tikz,tikz-cd,color,pgfplots,tabularx,amsfonts,amscd}
\usepackage{enumitem}
\usetikzlibrary{arrows,chains,matrix,positioning,scopes}
\pgfplotsset{compat=1.13}
\setlength{\headheight}{15pt}
\setlength{\headsep}{7pt}
\newcommand{\bsnl}{\bigskip\newline}
\pagestyle{fancy}
\usepackage[paper=a4paper, margin = 1in]{geometry}
\usepackage{hyperref}% http://ctan.org/pkg/hyperref
\usepackage[capitalise, noabbrev]{cleveref}

\hypersetup{ % colors hyperlinks with colors to make noticeable but is not an ugly green box like default
    colorlinks,
    linkcolor={red!50!black},
    citecolor={blue!50!black},
    urlcolor={blue!80!black}
}
\usepackage[mathscr]{euscript}\usepackage{pifont}
\usepackage{cleveref}
\usepackage{parskip}
\usepackage[makeroom]{cancel}


\newcommand{\cmark}{\ding{51}}%
\newcommand{\xmark}{\ding{55}}%

\renewcommand{\aa}{\mathbf{a}}
\newcommand{\bb}{\mathbf{b}}
\newcommand{\cc}{\mathbf{c}}
\newcommand{\dd}{\mathbf{d}}
\newcommand{\FF}{\mathbf{F}}
\newcommand{\ii}{\mathbf{i}}
\newcommand{\jj}{\mathbf{j}}
\newcommand{\rr}{\mathbf{r}}
\newcommand{\kk}{\mathbf{k}}
\newcommand{\uu}{\mathbf{u}}
\newcommand{\vv}{\mathbf{v}}
\newcommand{\ww}{\mathbf{w}}
\newcommand{\yy}{\mathbf{y}}
\newcommand{\R}{\mathbb{R}}
\newcommand{\T}{\mathcal{T}}
\newcommand{\N}{\mathbb{N}}
\newcommand{\C}{\mathscr{C}}
\newcommand{\Z}{\mathbb{Z}}
\newcommand{\B}{\mathcal{B}}
\newcommand{\Q}{\mathbb{Q}}
\newcommand{\Sph}{\mathbb{S}}
\newcommand{\D}{\mathbb{D}}

\def\dim{\mathop{\rm dim}\nolimits}
\def\image{\mathop{\rm Im}\nolimits}  
\def\interior{\mathop{\rm Int}\nolimits}
\def\kernel{\mathop{\rm Ker}\nolimits}
\def\cokernel{\mathop{\rm Coker}\nolimits}
\def\bd{\mathop{\rm Bd}\nolimits}
\def\ext{\mathop{\rm Ext}\nolimits}
\def\num{\mathop{\#}\limits}
\def\cl{\mathop{\rm Cl}\nolimits}
\def\lub{\mathop{\rm lub}\nolimits}


\newcommand{\proj}{\operatorname{proj}}
\newcommand{\ds}{\displaystyle}
\newcommand{\pa}{\partial}
\newcommand{\ol}{\overline{}}


\newcommand{\degree}{^{\circ}}
\renewcommand{\epsilon}{\varepsilon}

\newcommand{\upint}[2]{
  \overline{\int_{#1}^{#2}}
}
\newcommand{\lowint}[2]{
  \underline{\int_{#1}^{#2}}
}

\pagestyle{fancy}
\lhead{Math 6421: Measure Theory}
\chead{\bf Jose Nino: Lecture \#9}
\rhead{September 26, 2023}
\cfoot{Page \thepage}
% \setlength{\headheight}{10pt}

\theoremstyle{definition}
\newtheorem*{thm}{Theorem}
\newtheorem*{exercise}{Exercise}
\newtheorem*{definition}{Definition}
\newtheorem{problem}{Problem}
\newtheorem*{lemma}{Lemma}
\newtheorem*{cor}{Corollary}
\newtheorem*{prop}{Proposition}


\begin{document}

\section*{Section 4.4 General Lebesgue Integral}

\begin{definition}

    Define 
        \begin{align*}
            f^{+}(x) &= \ \text{non-negative part of} \ f  \\
                     &= \max \{ f(x), 0 \}.
        \end{align*}
    and 
        \begin{align*}
            f^{-}(x) &= \ \text{non-positive part of } \ f \\
                     &= \max \{ -f(x), 0 \}. 
        \end{align*}
    
\end{definition}

Note that then \[ f(x) = f^{+}(x) - f^{-}(x) \] and 
    \[
        |f(x)| = f^{+}(x) + f^{-}(x).
    \]

\begin{definition}
    A measurable function \( f \) is said to be \textbf{ Lebesgue integrable} over \( E \) if \( f^+ \) and \( f^- \) are integrable. In this case, then 
        \[
            \int_{E} = \int_{E} f^+ - \int_{E} f^{-}.
        \]
\end{definition}

\begin{prop}[\textbf{4.15}]

    Let \( f, g \) be integrable functions over \( E \). Then 
        \begin{enumerate}[label = \roman{*}.]
            \item  \( cf \) are integrable for all \( c \in \R \) over \( E \).
            \item \( f + g \) is integrable and 
                \[
                    \int_{E} f + g = \int_{E} f + \int_{E} g  
                \]
            \item If \( f \leq g \) a.e, then 
                \[
                    \int_{E} f \leq \int_{E} g.  
                \]
            \item If \( A, B \subset E \) and \( A \cap B = \emptyset \), then 
                \[
                    \int_{A \cup B} f = \int_{A} f + \int_{B.}  
                \]
        \end{enumerate}

\end{prop}

\begin{prop}[\textbf{4.16, Lebesgue Convergence Theorem}] Let \( g \) be integrable over \( E \) and let \( \{ f_n\} \) be a sequence of measurable functions. Suppose \( f_n \to f \) pointwise almost everywhere and \( |f_n| \leq g \) on \( E \). Then 
        \[
            \lim_{n \to \infty} \int_{E} f_n = \int_{E} f.  
        \]

    \begin{proof}
        Since \( g - f_n \geq 0 \) for all \(n \in \N \), Fatou's lemma, \( |f_n| \leq g \) on \( E \)
            \begin{align*}
                \int g - \int f = \int g - f &\leq \varliminf_{n \to \infty} \int g - f_n \\
                &= \varliminf_{n \to \infty} \int f - \varliminf_{n \to \infty} \int f_n \\
                &= \int g - \varliminf_{n \to \infty} \int f_n
            \end{align*}
        and so 
            \[
                \int g - \int f \leq  \int g - \varliminf_{n \to \infty} \int f_n
            \]
        which implies that
            \[
                \int f \geq \varliminf_{n \to \infty} \int f_n.
            \]
        Note that \( g + f_n \geq 0 \) as well. Then 
            \begin{align*}
                \int g + \int f_n = \int g + f_n &\leq \varliminf_{n \to \infty} \int g + f_n \\
                &= \varliminf_{n \to \infty} \int g + \varliminf_{n \to \infty} \int f_n \\
                &= \int g + \varliminf_{n \to \infty} \int f_n
            \end{align*}
        implying that
            \[
                \int f \leq \varliminf_{n \to \infty} \int f_n.  
            \]
        Because \( \{f_n\} \) converges, we know that \( \displaystyle \varliminf_{n \to \infty} f_n = \lim_{n \to \infty} f_n \) and so the result follows. 
    \end{proof}

\end{prop}

\begin{prop}[\textbf{4.17, Lebesgue Generalized Dominant Convergent Theorem}] 
    Let \( \{g_n\} \) be a sequence of integrable functions and \( g_n \to g \) pointwise a.e with \( g \) integrable. Let \( \{f_n\} \) be a sequence of measurable functions such that \( |f_n| \leq g_n \) and \( \{ f_n \} \to f \) pointwise a.e. If 
        \[
            \int g = \lim_{n \to \infty} \int g_n,  
        \]
    then 
        \[
            \int f = \lim_{n \to \infty} \int f_n.
        \]
    \begin{proof}\footnote{Proof is on page 92 of Royden.} This proof is also similar to Proposition 4.16 but use \( g_n \) instead of \( g \).
        
    \end{proof}
\end{prop}

\begin{problem}[\textbf{4.15}]
    Properties of function \( f \) being integrable.
    \begin{enumerate}[label = (\alph{*})]
        \item Let \( f \) be integrable over \( E \). Then for all \( \epsilon > 0 \), there exists a simple function \( \phi \) such that 
            \[
            \int_{E} \left| f - \phi \right| < \epsilon.
            \]
        \item Let \( f \) be integrable over \( E \). Then for all \( \epsilon > 0 \), there exists a step function \( \psi \) such that 
            \[ 
            \int_{E} \left| f - \psi \right| < \epsilon.
            \]
        \item Let \( f \) be integrable over \( E \). Then for all \( \epsilon > 0 \), there exists a continuous function \( g \) such that 
            \[ 
            \int_{E} \left| f - g \right| < \epsilon.
            \]
    \end{enumerate}
\end{problem}
\end{document}