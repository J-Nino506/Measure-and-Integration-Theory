\documentclass[12pt]{article}
\usepackage{amsmath,fullpage,graphicx,fancyhdr,enumerate,amsthm,amssymb,tikz,tikz-cd,color,pgfplots,tabularx,amsfonts,amscd}
\usepackage{enumitem}
\usetikzlibrary{arrows,chains,matrix,positioning,scopes}
\pgfplotsset{compat=1.13}
\setlength{\headheight}{15pt}
\setlength{\headsep}{7pt}
\newcommand{\bsnl}{\bigskip\newline}
\pagestyle{fancy}
\usepackage[paper=a4paper, margin = 1in]{geometry}
\usepackage{hyperref}% http://ctan.org/pkg/hyperref
\usepackage[capitalise, noabbrev]{cleveref}

\hypersetup{ % colors hyperlinks with colors to make noticeable but is not an ugly green box like default
    colorlinks,
    linkcolor={red!50!black},
    citecolor={blue!50!black},
    urlcolor={blue!80!black}
}
\usepackage[mathscr]{euscript}\usepackage{pifont}
\usepackage{cleveref}
\usepackage{parskip}
\usepackage[makeroom]{cancel}


\newcommand{\cmark}{\ding{51}}%
\newcommand{\xmark}{\ding{55}}%

\renewcommand{\aa}{\mathbf{a}}
\newcommand{\bb}{\mathbf{b}}
\newcommand{\cc}{\mathbf{c}}
\newcommand{\dd}{\mathbf{d}}
\newcommand{\FF}{\mathbf{F}}
\newcommand{\ii}{\mathbf{i}}
\newcommand{\jj}{\mathbf{j}}
\newcommand{\rr}{\mathbf{r}}
\newcommand{\kk}{\mathbf{k}}
\newcommand{\uu}{\mathbf{u}}
\newcommand{\vv}{\mathbf{v}}
\newcommand{\ww}{\mathbf{w}}
\newcommand{\yy}{\mathbf{y}}
\newcommand{\R}{\mathbb{R}}
\newcommand{\T}{\mathcal{T}}
\newcommand{\N}{\mathbb{N}}
\newcommand{\C}{\mathscr{C}}
\newcommand{\Z}{\mathbb{Z}}
\newcommand{\B}{\mathcal{B}}
\newcommand{\Q}{\mathbb{Q}}
\newcommand{\Sph}{\mathbb{S}}
\newcommand{\D}{\mathbb{D}}

\def\dim{\mathop{\rm dim}\nolimits}
\def\image{\mathop{\rm Im}\nolimits}  
\def\interior{\mathop{\rm Int}\nolimits}
\def\kernel{\mathop{\rm Ker}\nolimits}
\def\cokernel{\mathop{\rm Coker}\nolimits}
\def\bd{\mathop{\rm Bd}\nolimits}
\def\ext{\mathop{\rm Ext}\nolimits}
\def\num{\mathop{\#}\limits}
\def\cl{\mathop{\rm Cl}\nolimits}
\def\lub{\mathop{\rm lub}\nolimits}
\def\ess{\mathop{\rm ess}\nolimits}


\newcommand{\proj}{\operatorname{proj}}
\newcommand{\ds}{\displaystyle}
\newcommand{\pa}{\partial}
\newcommand{\ol}{\overline{}}
\newcommand{\toM}{\overset{m}{\to}}

\newcommand{\degree}{^{\circ}}
\renewcommand{\epsilon}{\varepsilon}

\newcommand{\upint}[2]{
  \overline{\int_{#1}^{#2}}
}
\newcommand{\lowint}[2]{
  \underline{\int_{#1}^{#2}}
}


\pagestyle{fancy}
\lhead{Math 6421: Measure Theory}
\chead{\bf Jose Nino: Lecture \#12}
\rhead{October 3, 2023}
\cfoot{Page \thepage}
% \setlength{\headheight}{10pt}

\theoremstyle{definition}
\newtheorem*{thm}{Theorem}
\newtheorem*{exercise}{Exercise}
\newtheorem*{definition}{Definition}
\newtheorem{problem}{Problem}
\newtheorem*{lemma}{Lemma}
\newtheorem*{cor}{Corollary}
\newtheorem*{prop}{Proposition}


\begin{document}

\section*{Chapter 6 Banach Spaces}

\section*{Section 6.1 \(L^p\) Spaces}

\begin{definition}
    A measurable function \( f: [0,1] \to \R \) is said to be in the space \( L^p = L^{p}([0,1]) \) if
        \[
            \int_{a}^{b} |f|^{p} < \infty . 
        \]
\end{definition}

Note the following 
    \begin{enumerate}[label = (\arabic{*})]
        \item \( L^1 \) is the space of integrable functions.
        \item \( L^p \) is closed under \( +  \) and under scalar multiplication i.e., if \( f,g \in L^p \) then \( f + cg \in L^p \) for all \( c \in \R \).This implies that \( L^p \) is a linear (vector) space. 
    \end{enumerate}

\begin{definition}
    The \( L^p \)-norm on a \( L^p \) space for all \( f \in L^p \) is given by 
        \[
            \lVert f \rVert = \lVert f \rVert_{p} = \left( \int_{0}^{1} |f|^{p}\right)^{1/p}.
        \]  
\end{definition}

In order for \( \lVert \cdot \rVert \) to be a norm over a vector space \( V \), the following properties must be satisfied for all \( v \in V \):
    \begin{enumerate}[label = (\arabic{*})]
        \item \( \lVert v \rVert = 0 \) if and only \( v = 0 \).
        \item For all \( \alpha \in \R \), \( \lVert \alpha v \rVert =  |\alpha| \lVert v \rVert \).
        \item \( \lVert v + w \rVert \leq \lVert v \rVert + \lVert w \rVert \).
    \end{enumerate}

In terms of \( L^p \) spaces, this is what we currently have for all \( f \in L^p \):
    \begin{enumerate}[label = (\arabic{*})]
        \item \( \lVert f \rVert = 0 \) if and only if \( f = 0 \) a.e.
        \item \( \lVert \alpha f \rVert = |\alpha| \lVert f \rVert \) for all \( \alpha \in \R \).
    \end{enumerate}
but we do not have the triangle inequality (third property from above) for norms of \( L^p \) spaces since \( \lVert f \rVert = 0 \) implies \( f = 0 \) almost everywhere rather than strict equality.

However, if we consider equivalence classes of \( L^p \) where functions are equal almost everywhere, we can define norms on these spaces. That is, define the relation \( \sim \) and 
    \[
        \tilde{L^{p}} = L^{P}\bigr/\sim
    \]
where \( f \sim g \Leftrightarrow f = g \) a.e. In other words, if mod out by functions that are equal almost everywhere, we can get a ``nice'' normed linear space!!

\begin{definition}
    The \( L^p \)-norm on a \( L^p \) space is defined as 
        \[
            || f ||_{p} := \left( \int_{0}^{1}|f|^{p} \right) \ \text{for all} \ p \in (0, \infty).
        \]
\end{definition}

If \( p \in (0,1 ) \), then \( ||f+ g||_{p} \leq ||f||_{p} + ||g||_{p}\). We want to show that \( ||f + g||_{p} \leq ||f||_{p} + ||g||_{p} \) for \( p \in [1, \infty] \).

\begin{definition}
    For \( p = \infty \), the space \( L^{\infty }\) is the set of bounded measurable functions for \( f \in L^{\infty} \). Then
        \begin{align*}
            ||f||_{\infty} &= \ess \sup |f(x)|   \\
                           &= \inf \left\{ M : m\{t: f(t) > M \right\} = 0\}.
        \end{align*}
\end{definition}

Note that \( ||\cdot||_{\infty} \) is the limit of \( ||\cdot||_{p} \) i.e., 
    \[
        f \in L^{\infty}, ||f||_{p} \to ||f||_{\infty}.  
    \]

\section*{Section 5.5 Convex Functions}

\begin{definition}
    A function \( \phi: [a,b] \to \R  \) is \textbf{convex} if for all \( x,y \in [a,b] \) and for all \( \lambda \in (0,1) \), we have that 
        \[
            \phi(\lambda x + (1-\lambda)y) \leq \lambda \phi(x) + (1-\lambda)\phi(y)  
        \]  
\end{definition}

\begin{prop}[\textbf{5.17}]\

    If \( \phi \) is convex on \( [a,b] \) then 
        \begin{enumerate}[label = (\arabic{*})]
            \item - (don't care about this)
            \item Right-hand and left-hand derivatives are equal except on a countable set. 
            \item The left- and right-hand derivatives are monotone increasing functions, and at each point the left-hand derivative is less than or equal to the right-hand derivative.
        \end{enumerate}
    
\end{prop}

\begin{cor}[\textbf{5.19}]

    If \( \phi \) is twice-differentiable, then \( \phi \) is convex if and only \( \phi''(x) > 0 \).

\end{cor}

\begin{cor}[\textbf{5.20}, Jensen's Inequality]

    Let \( \phi \) be a convex function on \( (-\infty, \infty) \) and \( f \) be an integrable function \( [0,1] \). Then
        \[
            \int_{0}^{1} \phi(f(t)) \,\mathrm{d} t \geq \phi \left[ \int_{0}^{1} f(t) \,\mathrm{d}t \right].
        \]


\end{cor}

An example of this is \( \phi(x) = x^{p} \). For any \( p \in (1, \infty) \), this function is twice-differentiable. Applying Jensen's inequality, we get
    \[
        \int_{0}^{1} |f(x)|^{p} \, \mathrm{d}x \geq \left( \int_{0}^{1} |f(x)| \, \mathrm{d} x\right).
    \]
If \( f \in L^p \), then \( f \in L^1 \) i.e., \( L^p \subset L^1 \).

\begin{thm}[\textbf{6.1, Minkowski Inequality}]

        If \( f, g \in L^{p} \) with \( p \in [1, \infty] \), then \( f + g \in L^{p} \) and 
            \[
                ||f + g||_{p} \leq ||f||_{p} + \ ||g||_{p}.  
            \]
        If \( p \in (1, \infty) \), then the equality can hold only if and only if there exists \( \alpha, \beta \geq 0 \) such that \( \beta f = \alpha g \).

        \begin{proof}
            We leave \( p = \infty \) as exercise so suppose \( p \) is finite. Let \( p \in [1, \infty] \). We normalize \( f \) and \( g \) i.e., there exists two functions \( f_0, g_0 \in L^{p} \) such that \( |f| = \alpha \cdot f_0 \) and \( |g| = \beta \cdot g_0 \) with \( ||f_0|| = ||g_0|| = 1 \) . Let \( \displaystyle \lambda = \frac{\alpha}{\alpha + \beta } \) and \( \displaystyle 1 - \lambda = \frac{\beta}{\alpha + \beta} \). By the convexity of \( \phi(t) = t^p \) for \( p \in [1, \infty ] \), we have that 
                \begin{align*}
                    |f(x) + g(x)|^p &\leq \left( |f(x)| + |g(x)\right)^{p} \\
                    &= \left(\alpha f_0 + \beta g_0 \right)^{p} \\
                    &= (\alpha + \beta)^{p} \left( \frac{\alpha}{\alpha + \beta}  f_0 + \frac{\beta}{\alpha + \beta} g_0 \right)^{p} \\
                    &\leq (\alpha + \beta)^{p} \left( \lambda f_0 + (1 - \lambda) g_0 \right)^{p}
                \end{align*}
            Now take the integrals of these guys (Jensen's inequality or something like that). In other words,
                \begin{align*}
                    ||f+g||^{p}_{p} &\leq (\alpha + \beta)^{p} \cdot \left( \lambda ||f_0||^{p}_{p} + (1-\lambda) ||g_0||^{p}_{p} \right) \\
                    &=\left( ||f||^{p}_{p} + ||g||^{p}_{p} \right) \cdot 1 && \text{because} \ f_0 = 1 = g_0.
                \end{align*}
            Taking the \( p \)th root, 
                \[
                    ||f+g||_{p} \leq |f||_{p} + ||g||_{p}.
                \]
            
        \end{proof}
This gives us the last norm-space requirement (triangle inequality of normed spaces).
\end{thm}

\begin{lemma}[\textbf{6.3}]

    Let \( p \in [1, \infty] \). Then for \(a, b, t \geq 0 \), we have 
        \[
            (a + tb)^{p} \geq a^{p} + ptba^{p-1}.  
        \]
        \begin{proof}
            Define the function 
                \[
                    \phi(t) = (a + tb)^{p} - a^{p} - ptba^{p-1}.
                \]
            We know \( \phi(0) = 0 \). Take the derivative of this thing and this is greater than zero because
                \begin{align*}
                    \phi'(x) &= p(a + tb)^{p-1} + b - pba^{p-1} \\
                    &= pb \left( (a + bt)^{p-1} - a^{p-1} \right)
                \end{align*}
            and so \( \phi \) is increasing. 
        \end{proof}
    
\end{lemma}

\begin{thm}[\textbf{6.4, Holder Inequality}]\footnote{If \( p,q  = 2 \), then this just reduces to the Cauchy-Schwarz inequality.} If \( p \) and \( q \) are nonnegative extended real numbers such that 
        \[
            \frac{1}{p} + \frac{1}{q} = 1,  
        \]
    and if \( f \in L^p \) and \( g \in L^q \), then \( f \cdot g \in L^1 \) and 
        \[
            \int | f g | \leq ||f||_{p} \cdot ||g||_{q}. 
        \]

\end{thm}



\end{document}