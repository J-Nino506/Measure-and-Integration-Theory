\documentclass[12pt]{article}
\usepackage{amsmath,fullpage,graphicx,fancyhdr,enumerate,amsthm,amssymb,tikz,tikz-cd,color,pgfplots,tabularx,amsfonts,amscd}
\usepackage{enumitem}
\usetikzlibrary{arrows,chains,matrix,positioning,scopes}
\pgfplotsset{compat=1.13}
\setlength{\headheight}{15pt}
\setlength{\headsep}{7pt}
\newcommand{\bsnl}{\bigskip\newline}
\pagestyle{fancy}
\usepackage[paper=a4paper, margin = 1in]{geometry}
\usepackage{hyperref}% http://ctan.org/pkg/hyperref
\usepackage[capitalise, noabbrev]{cleveref}

\hypersetup{ % colors hyperlinks with colors to make noticeable but is not an ugly green box like default
    colorlinks,
    linkcolor={red!50!black},
    citecolor={blue!50!black},
    urlcolor={blue!80!black}
}
\usepackage[mathscr]{euscript}\usepackage{pifont}
\usepackage{cleveref}
\usepackage{parskip}
\usepackage[makeroom]{cancel}


\newcommand{\cmark}{\ding{51}}%
\newcommand{\xmark}{\ding{55}}%

\renewcommand{\aa}{\mathbf{a}}
\newcommand{\bb}{\mathbf{b}}
\newcommand{\cc}{\mathbf{c}}
\newcommand{\dd}{\mathbf{d}}
\newcommand{\FF}{\mathbf{F}}
\newcommand{\ii}{\mathbf{i}}
\newcommand{\jj}{\mathbf{j}}
\newcommand{\rr}{\mathbf{r}}
\newcommand{\kk}{\mathbf{k}}
\newcommand{\uu}{\mathbf{u}}
\newcommand{\vv}{\mathbf{v}}
\newcommand{\ww}{\mathbf{w}}
\newcommand{\yy}{\mathbf{y}}
\newcommand{\R}{\mathbb{R}}
\newcommand{\T}{\mathcal{T}}
\newcommand{\N}{\mathbb{N}}
\newcommand{\C}{\mathscr{C}}
\newcommand{\Z}{\mathbb{Z}}
\newcommand{\B}{\mathcal{B}}
\newcommand{\Q}{\mathbb{Q}}
\newcommand{\Sph}{\mathbb{S}}
\newcommand{\D}{\mathbb{D}}

\def\dim{\mathop{\rm dim}\nolimits}
\def\image{\mathop{\rm Im}\nolimits}  
\def\interior{\mathop{\rm Int}\nolimits}
\def\kernel{\mathop{\rm Ker}\nolimits}
\def\cokernel{\mathop{\rm Coker}\nolimits}
\def\bd{\mathop{\rm Bd}\nolimits}
\def\ext{\mathop{\rm Ext}\nolimits}
\def\num{\mathop{\#}\limits}
\def\cl{\mathop{\rm Cl}\nolimits}
\def\lub{\mathop{\rm lub}\nolimits}
\def\ess{\mathop{\rm ess}\nolimits}

\newcommand{\dif}{\, \mathrm{d}}

\newcommand{\proj}{\operatorname{proj}}
\newcommand{\ds}{\displaystyle}
\newcommand{\pa}{\partial}
\newcommand{\ol}{\overline{}}
\newcommand{\toM}{\overset{m}{\to}}

\newcommand{\degree}{^{\circ}}
\renewcommand{\epsilon}{\varepsilon}

\newcommand{\upint}[2]{
  \overline{\int_{#1}^{#2}}
}
\newcommand{\lowint}[2]{
  \underline{\int_{#1}^{#2}}
}


\pagestyle{fancy}
\lhead{Math 6421: Measure Theory}
\chead{\bf Jose Nino: Lecture \#16}
\rhead{October 24, 2023}
\cfoot{Page \thepage}
% \setlength{\headheight}{10pt}

\theoremstyle{definition}
\newtheorem*{thm}{Theorem}
\newtheorem*{exercise}{Exercise}
\newtheorem*{definition}{Definition}
\newtheorem{problem}{Problem}
\newtheorem*{lemma}{Lemma}
\newtheorem*{cor}{Corollary}
\newtheorem*{prop}{Proposition}


\begin{document}

The goal of this lecture is to derive a different version of the fundamental theorem of calculus. 

\begin{definition}
    Let \( f:[a,b] \to R \), and let \( P = \{x_0 = a, x_1, \ldots, x_k = b\} \) be a partition of \( [a,b] \). Then define
        \begin{align*}
            p &= \sum_{i=1}^{k} \left( f(x_i) - f(x_{i-1}) \right)^{+}  \\
            n &= \sum_{i=1}^{k} \left( f(x_i) - f(x_{i-1}) \right)^{-}.
        \end{align*}

Recall that 
    \begin{align*}
        f^{+}(x) &= \max \{ f(x), 0\} \\
        f^{-}(x) &= \max \{ -f(x), 0 \} \\
        f(x) &= f^+(x) - f^-(x) \\
        |f(x)| &= f^+(x) + f^-(x).
    \end{align*}
Then
    \[
        t = n + p =  \sum_{i=1}^{k} |f(x_i) - f(x_{i-1})|.  
    \]
Further, define
    \begin{align*}
        P &= \sup_{P} p \\
        N &= \sup_{P} n \\
        T &=  \sup_{P} t
    \end{align*}
over all partitions \( P \) of \( [a,b] \). Then \( P \) is the \textbf{positive variation} of \( f \), 
\( N \) is the \textbf{negative variation} of \( f \), 
and \( T \) is the \textbf{total variation} of \( f \).
\end{definition}

Note that \( f(b) - f(a)  = p - n \). Also, for each partition of \( [a,b] \), 
\( p \leq T \leq p + n \).

\begin{definition}
    Using the same structure of the definition above, \( f \) is a function of \textbf{bounded variation} if
        \[
            T = T_{f} < \infty.   
        \]
\end{definition}

This tells us that the function is not ``wiggling'' that much (an example of a function that is not of bounded variation is \( f(x) = \sin(1/x). \))

\begin{lemma}[\textbf{5.4}]

    If \( f:[a,b] \to \R \) is a function of bounded variation on \( [a,b]\), then
        \[
            T_{a}^{b} = P_{a}^{b} + N_{a}^{b}  
        \]
    and 
        \[
            f(b) - f(a) =   P_{a}^{b} - N_{a}^{b}.
        \]
        
        \begin{proof}
            For any partition of \( [a,b] \),
                \[
                    p = n + f(b) - f(a);
                \]
            in other words, for any partition \( P \) of \( [a,b] \). Taking the supremum over any fixed partition,
                \[  
                     p = N + f(b) - f(a).
                \]
            Further,
                \begin{align*}
                    t = p + n &= p + (p - f(b) + f(a))  \\
                    &= 2p - f(b) + f(a).
                \end{align*}
            Taking the supremum over all partitions again,
                \[
                    T = 2P - f(b) - f(a) = P + N  
                \]
            and so we are done!
        \end{proof}

\end{lemma}

\begin{thm}[\textbf{5.5}]
    A function \( f \) is of bounded variation on \( [a,b] \) if and only if \( f \) is the difference of two monotone (increasing) real-valued functions \( [a,b] \).

        \begin{proof}
            
            We will show two directions to complete this proof.
            \begin{enumerate}
                \item[(\(\Rightarrow\))]First, we will note that the functions \( P_a^x \), \( N_a^x \), and \( T_a^x \) are increasing functions in \( x \). We also know that \( 0 \leq P_a^x \leq T_a^x \leq T_a^b < \infty \) and \( 0 \leq N_a^x \leq T_a^x \leq T_a^b < \infty \). Set \( g(x) =  P_a^x \) and \( h(x) =  N_a^x \). By our remark, \( g \) and \( h \) are increasing and so 
                \[
                    f(x) - f(a) = P_a^x - N_a^x = g(x) - h(x)   
                \]
            which follows by Lemma 5.4.
            \item[(\(\Leftarrow\))] Let \( f = g- h \) and suppose \( g, h \) are increasing on \( [a,b] \). Then for any partition of \( [a,b]\),
                \begin{align*}
                    \sum_{i=1}^{k} \left|f(x_{i} - f(x_{i-1})) \right| &\leq \sum_{i=1}^{k} g(x_i) - g(x_{i-1}) - \sum_{i=1}^{k} h(x_{i}) - h(x_{i-1}) \\
                    &= \left( g(b) - g(a) \right) + \left( h(b) - h(a)\right)
                \end{align*}
            which does not depend on the total variation of \( f \). Taking the suprema over partitions,  
                \[
                    T_a^b \leq (g(b) - g(a)) + (h(b) - h(a)).  
                \]
            \end{enumerate}
            Having shown a forward and backwards implication, this completes the proof. 
        \end{proof}

\end{thm}

\begin{cor}[\textbf{5.6}]

    If \( f \) is of bounded variation on \( [a,b] \), then \( f'(x) \) exist almost everywhere on \( [a,b] \).

\end{cor}

Note that if \( f \) is of bounded variation, this implies that \( f \) is bounded. This is because
    \[
        f(x) \leq |f(x) - f(a)| - f(x) \leq t - f(x)
    \]
and so \( f \) is bounded.

\section*{Section 5.3: Differentiation of an Integral}

\begin{definition}
    Let \( f \) be an integrable function \( [a, b] \). Define
        \[
            F(x) = \int_{a}^{x} f(t) \, \mathrm{d}t  
        \]
    for all \( x \in [a,b] \) is called the \textbf{indefinite integral} of \( f \) over \( [a,b] \).
\end{definition}

Our goal is to show that \( F'(x) = f(x) \) almost everywhere provided that \( f \) is integrable. 

\begin{lemma}[\textbf{5.7}]

    If \( f \) is integrable on \( [a, b] \), then then function \( F \) defined by
        \[
            F(x) = \int_{a}^{x} f(t) \, \mathrm{d}t  
        \]
    is a continuous function of bounded variation. 

        \begin{proof}
            Let \( f \geq 0 \) and let \( f \in L^{1}[a,b] \) (an integrable function). Fix \( \epsilon > 0 \). Then by Proposition 4.14, there exists \( \delta > 0 \) such that \( A \subset [a,b] \) with \( m(A) < \delta \) implies that
                \[
                    \int_{A} f < \epsilon.
                \]
            THen we have that 
                \begin{align*}
                    F(x + h) - F(x) &= \int_{a}^{x+h}f - \int_{a}^{x} f \\ 
                    &= \int_{x}^{x + h} f \\
                    &\leq \int_{A} f \\
                    &< \epsilon
                \end{align*}
            and so \( F \) is continuous. To show bounded variation, fix any partition \( P = \{x_0 = a, x_1, \ldots, x_k = b \} \) of \( [a,b] \).
            Then 
                \begin{align*}
                    \sum_{i=1}^{k}\left| F(x_i) - F(x_{i-1}) \right| &= \sum_{i=1}^{k} \left| \int_{x_{i-1}}^{x_{i}} f(t) \, \mathrm{d} t\right| \\
                    &\leq \sum_{i=1}^{k} \int_{x_{i-1}}^{x_{i}} \left| f(t) \right| \mathrm{d}t  \\
                    &= \int_{a}^{b} |f(t)| \, \mathrm{d}t \\
                    &< \infty
                \end{align*}
            and so we are done!
        \end{proof}
\end{lemma}

\begin{lemma}[\textbf{5.8}]

    If \( f \) is integrable on \( [a,b] \) and
        \[
            \int_{a}^{x} f(t) \dif t = 0 
        \]
    for all \(  \in [a,b] \), then \( f(t) = 0 \) almost everywhere on \( [a,b] \).

        \begin{proof}
            By way of contradiction, suppose \( f(t) \neq 0 \) almost everywhere in \( [a,b] \). Let \( E = \{ x: f(x) > 0 \} \) and suppose \( m(E) > 0 \). By Littlewood's first principle, there exists a closed set \( K \subset E\) such that \( m(K) > 0 \). Let \( O = [a,b] \setminus K \) and so is an open set. Then we know that    
                \begin{align*}
                    0  = \int_{a}^{b} f &= \underbrace{\int_{K} f}_{> 0} + \int_{O} f 
                \end{align*}
            which is true because if \( g \geq 0 \) and \( m(A) > 0 \), then \( \displaystyle g = 0 \) if and only if \( g = 0 \) almost everywhere. 
            Thus \( \displaystyle \int_{O} f \neq 0 \) as long as \( O \) is an open set. By Lindelof's lemma, 
                \[
                    O = \bigcup_{n=1}^{\infty} (a_n, b_n)  
                \]
            where \( (a_n, b_n) \cap (a_m, b_m) = \emptyset \) for all \( n \neq m \). So
                \begin{align*}
                    0 \neq \int_{O} f = \sum_{i=1}^{\infty} \int_{a_n}^{b_n} f(t) \dif t.
                \end{align*}
            So there exists \( n \in \N \) such that 
                \begin{align*}
                    0 \neq \int_{a_n}^{b_n} f(x) \dif t &= \int_{b}^{b_n}f(t) \dif t - \int_{a}^{a_n} f(t) \dif t \\
                    &= 0 + 0 \\
                    &= 0
                \end{align*}
            which comes from assumption. But this implies that \( m(E) = 0 \), which is a contradiction. By a similar argument
                \[
                    m\left( \{ x: f(x) < 0 \} \right).
                \]
        \end{proof}
    
\end{lemma}

\begin{lemma}[\textbf{5.9}]

    If \( f \) is bounded and measurable on \( [a,b] \), and 
        \[
            F(x) = \int_{a}^{x} f(t) \dif t,
        \]  
    then \( F'(x) = f(x) \) for almost all \( x \in [a,b ] \).

    \begin{proof}
        By Lemma 5.7, since \( F \) is integrable, \( F \) is a function of bounded variation and so \( F'(x) \) exists almost everywhere on \( [a,b] \). Let \( |f| < K \). Then we write
            \begin{align*}
                f_n(x) &= \frac{F\left( x + \frac{1}{n}\right) - F(x)}{\frac{1}{n}} 
            \end{align*}
        which as \( n \to \infty \), \( f_n(x) \to F'(x) \). So we have that 
            \begin{align*}
                f_n(x) &= \frac{F\left( x + \frac{1}{n}\right) - F(x)}{\frac{1}{n}} \\
                &= n \cdot \int_{x}^{x+ \frac{1}{n}} F(t) \dif t.
            \end{align*}
        Also \( |f_n(x)| \leq K \). Because \( f_n(x) \to F'(x) \) almost everywhere \( f_n(x) \) is bounded, by the Bounded Convergence Theorem, for all \( c \in [a,b] \), 
            \begin{align*}
                \int_{a}^{c} F'(t) \dif t &= \lim_{n \to \infty} \int_{a}^{c} f_n(t) \dif t \\
                &= \lim_{n \to \infty} n \int_{a}^{c} \left( F \left( x + \frac{1}{n}\right) - F(x) \right) \\
                &= \lim_{n \to \infty} n \left( \int_{a}^{c} F \left( x + \frac{1}{n} \right) \dif x - \int_{a}^{c} F(x) \dif x \right) && \int_{a + 1/n}^{b + 1/n} = \int_{a}^{c} F \left( x + \frac{1}{n} \right) - \int_a^c F(x) \dif x \\
                &= \lim_{n \to \infty } n  \left(\int_{c}^{c + \frac{1}{n}} F(x) \dif x  - \int_{a}^{a + \frac{1}{n}} F(x) \dif x \right) \\
                &= F(c) - F(a) \\
                &= \int_{a}^{c} f(x) \dif x. 
            \end{align*}
    \end{proof}
    
\end{lemma}



\end{document}