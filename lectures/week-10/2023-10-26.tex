\documentclass[12pt]{article}
\usepackage{amsmath,fullpage,graphicx,fancyhdr,enumerate,amsthm,amssymb,tikz,tikz-cd,color,pgfplots,tabularx,amsfonts,amscd}
\usepackage{enumitem}
\usetikzlibrary{arrows,chains,matrix,positioning,scopes}
\pgfplotsset{compat=1.13}
\setlength{\headheight}{15pt}
\setlength{\headsep}{7pt}
\newcommand{\bsnl}{\bigskip\newline}
\pagestyle{fancy}
\usepackage[paper=a4paper, margin = 1in]{geometry}
\usepackage{hyperref}% http://ctan.org/pkg/hyperref
\usepackage[capitalise, noabbrev]{cleveref}

\hypersetup{ % colors hyperlinks with colors to make noticeable but is not an ugly green box like default
    colorlinks,
    linkcolor={red!50!black},
    citecolor={blue!50!black},
    urlcolor={blue!80!black}
}
\usepackage[mathscr]{euscript}\usepackage{pifont}
\usepackage{cleveref}
\usepackage{parskip}
\usepackage[makeroom]{cancel}


\newcommand{\cmark}{\ding{51}}%
\newcommand{\xmark}{\ding{55}}%

\renewcommand{\aa}{\mathbf{a}}
\newcommand{\bb}{\mathbf{b}}
\newcommand{\cc}{\mathbf{c}}
\newcommand{\dd}{\mathbf{d}}
\newcommand{\FF}{\mathbf{F}}
\newcommand{\ii}{\mathbf{i}}
\newcommand{\jj}{\mathbf{j}}
\newcommand{\rr}{\mathbf{r}}
\newcommand{\kk}{\mathbf{k}}
\newcommand{\uu}{\mathbf{u}}
\newcommand{\vv}{\mathbf{v}}
\newcommand{\ww}{\mathbf{w}}
\newcommand{\yy}{\mathbf{y}}
\newcommand{\R}{\mathbb{R}}
\newcommand{\T}{\mathcal{T}}
\newcommand{\N}{\mathbb{N}}
\newcommand{\C}{\mathscr{C}}
\newcommand{\Z}{\mathbb{Z}}
\newcommand{\B}{\mathcal{B}}
\newcommand{\Q}{\mathbb{Q}}
\newcommand{\Sph}{\mathbb{S}}
\newcommand{\D}{\mathbb{D}}

\def\dim{\mathop{\rm dim}\nolimits}
\def\image{\mathop{\rm Im}\nolimits}  
\def\interior{\mathop{\rm Int}\nolimits}
\def\kernel{\mathop{\rm Ker}\nolimits}
\def\cokernel{\mathop{\rm Coker}\nolimits}
\def\bd{\mathop{\rm Bd}\nolimits}
\def\ext{\mathop{\rm Ext}\nolimits}
\def\num{\mathop{\#}\limits}
\def\cl{\mathop{\rm Cl}\nolimits}
\def\lub{\mathop{\rm lub}\nolimits}
\def\ess{\mathop{\rm ess}\nolimits}

\newcommand{\dif}{\, \mathrm{d}}

\newcommand{\proj}{\operatorname{proj}}
\newcommand{\ds}{\displaystyle}
\newcommand{\pa}{\partial}
\newcommand{\ol}{\overline{}}
\newcommand{\toM}{\overset{m}{\to}}
\newcommand{\floor}[1]{\left\lfloor #1 \right\rfloor}
\newcommand{\ceil}[1]{\left\lceil #1 \right\rceil}


\newcommand{\degree}{^{\circ}}
\renewcommand{\epsilon}{\varepsilon}

\newcommand{\upint}[2]{
  \overline{\int_{#1}^{#2}}
}
\newcommand{\lowint}[2]{
  \underline{\int_{#1}^{#2}}
}


\pagestyle{fancy}
\lhead{Math 6421: Measure Theory}
\chead{\bf Jose Nino: Lecture \#17}
\rhead{October 26, 2023}
\cfoot{Page \thepage}
% \setlength{\headheight}{10pt}

\theoremstyle{definition}
\newtheorem*{thm}{Theorem}
\newtheorem*{exercise}{Exercise}
\newtheorem*{definition}{Definition}
\newtheorem{problem}{Problem}
\newtheorem*{lemma}{Lemma}
\newtheorem*{cor}{Corollary}
\newtheorem*{prop}{Proposition}
\newtheorem*{remark}{Remark}


\begin{document}

\begin{remark}
    Note that if \( f \geq 0 \), 
        \[  
            F(x) = \int_{a}^{x} f(t) \dif t
        \]
    is increasing. This implies \( F \) is a monotone function and so 
        \[
            \int_{a}^{b} F'(x) \dif x \leq F(b) - F(a).    
        \]
\end{remark}

\begin{thm}[\textbf{5.10}]

    Let \( f \) be an integrable function on \( [a,b] \) and suppose that 
        \[  
            F(x) = F(a) + \int_{a}^{x} f(t) \dif t.
        \]
    Then \( F'(x) = f(x) \) for almost all \( x \in [a,b] \).

        \begin{proof}
            Using the remark above, without loss of generality, suppose \( f \geq 0 \). To use the previous lemma (which supposes \( f \) is bounded), define
                \[
                    f_n(x) = \begin{cases}
                        f(x) & f(x) \leq n \\
                        n & \text{otherwise}.
                    \end{cases}  
                \]
            Then \( f - f_n \geq 0 \) for all \( n \in \N \). Now define 
                \[  
                    G_n(x) = \int_{a}^{x} f  - f_n    
                \]
            which is an increasing function since \( f - f_n \) is nonnegative. 
            So \( G'_n(x) \) exists almost everywhere and \( G'_n(x) \geq 0 \). 
            By Lemma 5.9, since \( f_n(x) \) is a bounded function,
                \begin{align*}
                    \frac{\dif}{\dif x} \left( \int_{a}^{x} f(t) \dif t \right) = f_n(x)
                \end{align*}
            almost everywhere. Then 
                \begin{align*}
                    F'(x) = \frac{\dif }{\dif x} G_n + \frac{\dif }{\dif x} \int_{a}^{x} f_n
                \end{align*}
            implies that \( F'(x) \geq f_n(x) \) almost everywhere.
            From the beginning remark (so that \( F \) is monotonic), this gives that
                \begin{align*}
                    \int_{a}^{b} F'(x) \dif x &\leq F(b) - F(a) \\
                    &= \int_{a}^{b} f(x) \dif x
                \end{align*}
            and so we get that 
                \[
                    \int_{a}^{b} \left( \underbrace{F'(x) - f(x)}_{\geq 0} \right) \dif x  = 0  
                \]
            implying that \( F'(x) = f(x) \) almost everywhere. 
        \end{proof}
\end{thm}

\section*{Section 5.4 Absolute Continuity}

\begin{definition}
    A real-valued function \( f \) on \( [a,b] \) is said to be \textbf{absolutely continuous} on \( [a,b ] \)
    if for all \( \epsilon > 0 \), there exists \( \delta > 0 \) such that 
        \begin{align*}
            \sum_{i=}^{n} \left| f(x'_i) - f(x'_{i}) \right|
        \end{align*}
    for every finite collections of \( \{ (x_i, x'_{i}) \} \) of nonoverlapping intervals with 
        \[
            \sum_{i=1}^{n} \left| x'_{i} - x_{i} \right| < \delta.  
        \]
\end{definition}


\begin{lemma}[\textbf{5.11}]

    If \( f \) is absolutely continuous on \( [a,b] \), then it is of bounded variation on \( [a,b] \).

        \begin{proof}
            Let \( \epsilon  = 1 \). Then there exists \( \delta > 0 \) for the absolutely continuity property. 
            Let \( \displaystyle K = \ceil{\frac{b-a}{\delta} + 1} \)
            For this partition of \( [a,b] \), we group the intervals into \( K \) sets of intervals each with total length less than \( \delta \).
        \end{proof}
\end{lemma}

\begin{lemma}[\textbf{5.13}]

    If \( f \) is absolutely continuous on \( [a,b] \) and \( f'(x) = 0 \) almost everywhere, then \( f \) is constant. 
    
\end{lemma}

\begin{thm}[\textbf{5.14}]

    A function \( F \) is an indefinite integral if and only \( F \) is absolutely continuous. 

\end{thm}

\begin{remark}
    The above theorem tells us that there exists an integrable function \( f \) such that 
        \[
            F(x) = F(a) + \int_{a}^{x} f(t) \dif t.  
        \]
\end{remark}


\end{document}