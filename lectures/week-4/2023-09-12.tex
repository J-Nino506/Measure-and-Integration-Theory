\documentclass[12pt]{article}
\usepackage{amsmath,fullpage,graphicx,fancyhdr,enumerate,amsthm,amssymb,tikz,tikz-cd,color,pgfplots,tabularx,amsfonts,amscd}
\usepackage{enumitem}
\usetikzlibrary{arrows,chains,matrix,positioning,scopes}
\pgfplotsset{compat=1.13}
\setlength{\headheight}{15pt}
\setlength{\headsep}{7pt}
\newcommand{\bsnl}{\bigskip\newline}
\pagestyle{fancy}
\usepackage[paper=a4paper, margin = 1in]{geometry}
\usepackage{hyperref}% http://ctan.org/pkg/hyperref
\usepackage[capitalise, noabbrev]{cleveref}

\hypersetup{ % colors hyperlinks with colors to make noticeable but is not an ugly green box like default
    colorlinks,
    linkcolor={red!50!black},
    citecolor={blue!50!black},
    urlcolor={blue!80!black}
}
\usepackage[mathscr]{euscript}\usepackage{pifont}
\usepackage{cleveref}
\usepackage{parskip}
\usepackage[makeroom]{cancel}


\newcommand{\cmark}{\ding{51}}%
\newcommand{\xmark}{\ding{55}}%

\renewcommand{\aa}{\mathbf{a}}
\newcommand{\bb}{\mathbf{b}}
\newcommand{\cc}{\mathbf{c}}
\newcommand{\dd}{\mathbf{d}}
\newcommand{\FF}{\mathbf{F}}
\newcommand{\ii}{\mathbf{i}}
\newcommand{\jj}{\mathbf{j}}
\newcommand{\rr}{\mathbf{r}}
\newcommand{\kk}{\mathbf{k}}
\newcommand{\uu}{\mathbf{u}}
\newcommand{\vv}{\mathbf{v}}
\newcommand{\ww}{\mathbf{w}}
\newcommand{\yy}{\mathbf{y}}
\newcommand{\R}{\mathbb{R}}
\newcommand{\T}{\mathcal{T}}
\newcommand{\N}{\mathbb{N}}
\newcommand{\C}{\mathscr{C}}
\newcommand{\Z}{\mathbb{Z}}
\newcommand{\B}{\mathcal{B}}
\newcommand{\Q}{\mathbb{Q}}
\newcommand{\Sph}{\mathbb{S}}
\newcommand{\D}{\mathbb{D}}

\def\dim{\mathop{\rm dim}\nolimits}
\def\image{\mathop{\rm Im}\nolimits}  
\def\interior{\mathop{\rm Int}\nolimits}
\def\kernel{\mathop{\rm Ker}\nolimits}
\def\cokernel{\mathop{\rm Coker}\nolimits}
\def\bd{\mathop{\rm Bd}\nolimits}
\def\ext{\mathop{\rm Ext}\nolimits}
\def\num{\mathop{\#}\limits}
\def\cl{\mathop{\rm Cl}\nolimits}
\def\lub{\mathop{\rm lub}\nolimits}


\newcommand{\proj}{\operatorname{proj}}
\newcommand{\ds}{\displaystyle}
\newcommand{\pa}{\partial}
\newcommand{\ol}{\overline{}}


\newcommand{\degree}{^{\circ}}
\renewcommand{\epsilon}{\varepsilon}

\pagestyle{fancy}
\lhead{Math 6421: Measure Theory}
\chead{\bf Jose Nino: Lecture \#5}
\rhead{September 12, 2023}
\cfoot{Page \thepage}
% \setlength{\headheight}{10pt}

\theoremstyle{definition}
\newtheorem*{thm}{Theorem}
\newtheorem*{exercise}{Exercise}
\newtheorem*{definition}{Definition}
\newtheorem{problem}{Problem}
\newtheorem*{lemma}{Lemma}
\newtheorem*{cor}{Corollary}
\newtheorem*{prop}{Proposition}


\begin{document}

\section*{Section 3.5, Measurable Functions}

\begin{prop}[\textbf{3.18}]

    Let \( E \subset \R\), and Let \( f: E \to [-\infty, \infty] \) be an extended real-valued function whose domain is measurable. Let \( \alpha \in \R \) be any real number. Then the following statements are equivalent:

    \begin{enumerate}[label = (\roman{*})]
        
        \item The set \( \{ x: f(x) > \alpha \} \) is measurable.
        \item The set \( \{ x: f(x) \geq \alpha \} \) is measurable. 
        \item The set \( \{ x: f(x) < \alpha \} \) is measurable. 
        \item The set \( \{ x: f(x) \leq \alpha \} \) is measurable. 
        
        \noindent All together, these imply

        \item The set \( \{ x: f(x) = \alpha \} \) is measurable. 
    \end{enumerate}

        \begin{proof}\footnote{Proof is on page 67.}

        \end{proof}

    
\end{prop}

\begin{definition}
     An extended real-valued function \( f: E \to [-\infty, \infty] \) is \textbf{(Lebesgue) measurable} if its domain is measurable and satisfies one of the first four statements of Proposition 18.
\end{definition}

Note that this means that any continuous function is measurable since then the pre-image of any open set is still an open set. 

\begin{prop}[\textbf{3.19}]
    
    Let \(f \) and \( g \) be two measurable functions defined on the same domain, and let \( c \in \R \). Then the functions \( f + c \), \( cf \), \( f + g \), \( g - f \), and \( fg \) are measurable.

        \begin{proof}
            Let \( \alpha \in \R \) be any real number. Fix \( c \in \R \). For \( f(x) + c \), note that 
                \[
                    \{ x: f(x) + c < \alpha \} = \{ x: f(x) < \alpha - c \}
                \]
            and \( \alpha - c \) is a real number, this set is still measurable i.e., \( f + c \) is measurable. A similar argument shows that \( cf \) is measurable as well. 

            Take the set 
                \begin{equation}
                    \label{prop-3.19-addition}
                    \{ x : f(x) + g(x) < \alpha \}.    
                \end{equation}
            Observe that \( f(x) + g(x) < \alpha \Leftrightarrow  f(x) < \alpha - g(x) \). By the density of \( \Q \), there exists \( r \in \Q \) such that \( f(x) < r < \alpha - g(x) \). So we can write \cref{prop-3.19-addition} as
                \[
                    \{ x : f(x) + g(x) < \alpha \} = \bigcup_{r \in \Q}  \left( \{ x : f(x) < r \} \cap \{ x : g(x) < \alpha - r \} \right).
                \]
            Because this set is countable, this set is measurable and thus \( f + g \) is measurable. 

            To show that \( fg \) is measurable, we can show that \( f^2 \) is measurable since \
                \[
                 fg = \frac{1}{2} \left( (f+g)^2 - f^2 - g^2 \right).
                \]
            Take the set 
                \begin{equation}
                    \label{prop-3.19-product}
                    \{ x : f^{2}(x) < \alpha \}. 
                \end{equation}
            For \( \alpha \geq 0 \), note that \( f^{2} < \alpha \) is the same as saying \( f(x) > \sqrt{\alpha} \) and \( f(x) < -\sqrt{\alpha} \). Thus, \cref{prop-3.19-product} can rewritten as 
                \[
                    \{ x : f^{2}(x) < \alpha \} = \{ x: f(x) > \sqrt{\alpha} \} \cup \{x: f(x) < -\sqrt{\alpha} \}.
                \]
            This is a measurable set which completes the proof.
            \end{proof}
\end{prop}
            
\begin{thm}[\textbf{3.20, Limit of Measurable Functions is Measurable}]\footnote{Proof is on bottom of page 68 and top of page 69}
        \begin{proof}
            For \( f(x) + c \), note that 
                \[
                    \{ x: f(x) + c < \alpha \} = \{ x: f(x) < \alpha - c \}.    
                \]
        \end{proof}
\end{thm}

\begin{thm}[\textbf{3.20, Limit of Measurable Functions is Measurable}]


    Let \( \{ f_n \} \) be a sequence of measurable functions with the same domain. Then the functions \( \sup\{f_1(x), \ldots, f_n(x) \} \), \( \inf \{ f_1(x), \ldots, f_n(x)\}\), \( \sup_{n} f_n \), \( \inf_{n} f_n \), \( \varlimsup f_n \), and \( \varliminf f_n \) are measurable. 

        \begin{proof}
            Let \( \{ f_n\} \) be a sequence of measurable functions. Let \( h(x) = \sup\{f_1(x), \ldots, f_n(x) \} \) and we so must show that \( \{ x: h(x) < \alpha \}\) for all \( \alpha \in \R \). To that end, let \( \alpha \in \R \) be chosen. Then 
                \[
                     \{ x: h(x) < \alpha \} = \bigcup_{i=1}^{n} \, \{ x: f_i(x) > \alpha \}
                \]
            which, because the right-hand side is a union of measurable sets from the \( f_i \)'s being measurable, means that the set \( \{ x: h(x) < \alpha \}\)  is also measurable.

            Let \( g(x) = \sup_{n} f_n \). By a similar argument as above, 
                \[
                    \{ x: h(x) < \alpha \} = \bigcup_{i=1}^{\infty} \, \{ x: f_i(x) > \alpha \}  
                \]
            is a countable set so \( \{ x: g(x) < \alpha \}\) is measurable. 
        \end{proof}
\end{thm}

\begin{definition}
    A property is said to hold \textbf{almost everywhere} (a.e) if the set of points where it fails to hold is a set of measure zero. Thus \(f = g\) a.e if \( f \) and \( g \) have the same domain and \( m \{ x: f(x) \neq g(x) \} \). 
\end{definition}


\begin{prop}[\textbf{3.21}] If \( f \) is measurable and \( f =  g\) a.e, then \( g \) is measurable.

        \begin{proof}\footnote{Proof is on middle of page 69.}
            Let \( E = \{ x: f(x) \neq g(x) \} \). 
            
            This is equivalent to saying that     

            Let \( \{ x: g(x) > \alpha \} \). This is equivalent to saying that     

                \[
                    \{ x: g(x) > \alpha \} = \{ x: f(x) > \alpha \} \cup \{ x: g(x) > \alpha \}
                \]
        \end{proof}
    
\end{prop}


The following proposition essentially says that measurable functions are nearly continuous; or. in other words, we can ``nicely'' approximate measurable functions. 

\begin{prop}[\textbf{3.22, Littlewood's 2nd Principle}]


    Let \( f: [a,b] \to E \) be a measurable function with \( E \subset \R \) and is equal to \( \pm \infty \) only on sets with measure zero. Then for all \( \epsilon > 0 \), there exist a step function \( g \) and a continuous function \( f \) such 
        \[
            |f - g| < \epsilon \quad \text{and} \quad |f - h| < \epsilon  
        \]
    except on set of measure less than \( \epsilon \); i.e., \( m\{x: |f(x) - g(x)| \geq \epsilon\} <  \epsilon \) and \( m\{x: |f(x) - h(x)| \geq \epsilon\} <  \epsilon \). If in addition \( m \leq f \leq M \), then we may choose the functions \( g \) and \( h \) so that \( m \leq g \leq M\) and \( m \leq h \leq M \).
 
\end{prop}


\begin{prop}[\textbf{3.23, (Weak) Egonoff's Theorem}]

    Let \( E \) be a measurable set of finite measure, and \( \{ f_n \} \) be a sequence of measurable functions defined on \( E \). Let \( f \) be real-valued function such for each \( x \in E \) we have \( f_n(x) \to f(x) \). Then for all \( \epsilon > 0 \) and all \( \delta > 0 \), there is measurable set \( A \subset E \) with \( m(A) < \delta \) and \( N \in \N \) such that for all \( x \not\in A \) and all \( n \geq N\),
        \[
            |f_n(x) - f(x)| < \epsilon.    
        \]
    
\end{prop}

\end{document}